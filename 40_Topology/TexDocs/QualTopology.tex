\input{"preamble.tex"}

\addbibresource{QualTopology.bib}

\let\Begin\begin
\let\End\end
\newcommand\wrapenv[1]{#1}

\makeatletter
\def\ScaleWidthIfNeeded{%
 \ifdim\Gin@nat@width>\linewidth
    \linewidth
  \else
    \Gin@nat@width
  \fi
}
\def\ScaleHeightIfNeeded{%
  \ifdim\Gin@nat@height>0.9\textheight
    0.9\textheight
  \else
    \Gin@nat@width
  \fi
}
\makeatother

\setkeys{Gin}{width=\ScaleWidthIfNeeded,height=\ScaleHeightIfNeeded,keepaspectratio}%

\title{
\textbf{
    Qual Topology
  }
  }







\begin{document}

\date{}
\maketitle


\newpage

% Note: addsec only in KomaScript
\addsec{Table of Contents}
\tableofcontents
\newpage

\hypertarget{preface}{%
\section{Preface}\label{preface}}

A great deal of credit for this document goes to Mike Usher, who created
an initial PDF of past UGA qual questions organized by topic. Here is a
list of problems that Mike recommended reviewing during our problem
sessions in Spring 2020:

\begin{itemize}
\tightlist
\item
  Section 1, Point-Set: 8, 10, 11, 14, 16, 19, 22, 23, 27 (Bolzano
  Weierstrass), 30 (Standard, see counterexamples), 31, 32, 38, 42, 44
\item
  Section 2, Fundamental Group: 1
\item
  Section 3, Covering Spaces: 1b and c, 2, 3, 6, 7, 8, 10 (not really
  about surfaces per se), 11, 12, 13, 14, 16
\item
  Section 4, Homology/Degree Theory: 2, 4, 5, 8, 13, 19, 21, 22
\item
  Section 5, Cell Complexes/Attaching: 1, 5, 10, 16
\item
  Section 6, Surfaces: 3, 6, 7, 11, 14
\item
  Section 7, Fixed Points: 4, 6, 12, 13, 14
\item
  Section 8, Misc Algebraic Topology: 1, 3, 6, 8
\end{itemize}

\begin{warnings}

Usually 30\% of the problems on any given qual are related to
point-set/general Topology. Note that this material is not covered in
the course!

\end{warnings}

\hypertarget{general-topology}{%
\section{General Topology}\label{general-topology}}

\hypertarget{topologies-subspaces-closures-and-maps}{%
\subsection{Topologies, Subspaces, Closures, and
Maps}\label{topologies-subspaces-closures-and-maps}}

\hypertarget{fall-11}{%
\subsubsection{Fall '11}\label{fall-11}}

\begin{problem}[Fall 2011]

Let \(X\) be a topological space, and \(B \subset A \subset X\). Equip
\(A\) with the subspace topology, and write
\({ \operatorname{cl}} _X (B)\) or \({ \operatorname{cl}} _A (B)\) for
the closure of \(B\) as a subset of, respectively, \(X\) or \(A\).

Determine, with proof, the general relationship between
\({ \operatorname{cl}} _X (B) \cap A\) and
\({ \operatorname{cl}} _A (B)\)

\begin{quote}
I.e., are they always equal? Is one always contained in the other but
not conversely? Neither?
\end{quote}

\end{problem}

\emph{Concept review omitted.}

\emph{Strategy omitted.}

\emph{Solution omitted.}

\hypertarget{fall-05}{%
\subsubsection{6 (Fall '05)}\label{fall-05}}

\begin{problem}[Fall 2005]

Prove that the unit interval \(I\) is compact. Be sure to explicitly
state any properties of \({\mathbb{R}}\) that you use.

\end{problem}

\emph{Concept review omitted.}

\emph{Strategy omitted.}

\emph{Solution omitted.}

\hypertarget{fall-06.}{%
\subsubsection{7 (Fall '06).}\label{fall-06.}}

\begin{problem}[Fall 2006, 7]

A topological space is \textbf{sequentially compact} if every infinite
sequence in \(X\) has a convergent subsequence.

Prove that every compact metric space is sequentially compact.

\end{problem}

\hypertarget{fall-10.}{%
\subsubsection{8 (Fall '10).}\label{fall-10.}}

\begin{problem}[Fall 2010, 8]

Show that for any two topological spaces \(X\) and \(Y\) ,
\(X \times Y\) is compact if and only if both \(X\) and \(Y\) are
compact.

\end{problem}

\emph{Concept review omitted.}

\emph{Strategy omitted.}

\emph{Solution omitted.}

\hypertarget{spring-06.}{%
\subsubsection{12 (Spring '06).}\label{spring-06.}}

\begin{problem}[Spring 2006, 12]

Write \(Y\) for the interval \([0, \infty)\), equipped with the usual
topology.

Find, with proof, all subspaces \(Z\) of \(Y\) which are retracts of
\(Y\).

\end{problem}

\todo[inline]{Not finished. Add concepts}

\emph{Solution omitted.}

\hypertarget{fall-06.-1}{%
\subsubsection{13 (Fall '06).}\label{fall-06.-1}}

\begin{problem}[Fall 2006, 13]

\envlist

\begin{enumerate}
\def\labelenumi{\alph{enumi}.}
\item
  Prove that if the space \(X\) is connected and locally path connected
  then \(X\) is path connected.
\item
  Is the converse true? Prove or give a counterexample.
\end{enumerate}

\end{problem}

\hypertarget{fall-07}{%
\subsubsection{14 (Fall '07)}\label{fall-07}}

\begin{problem}[?]

Let \(\left\{{X_\alpha \mathrel{\Big|}\alpha \in A}\right\}\) be a
family of connected subspaces of a space \(X\) such that there is a
point \(p \in X\) which is in each of the \(X_\alpha\).

Show that the union of the \(X_\alpha\) is connected.

\end{problem}

\todo[inline]{Proof 2 not complete?}

\emph{Solution omitted.}

\hypertarget{fall-04.}{%
\subsubsection{5 (Fall '04).}\label{fall-04.}}

\begin{problem}[?]

Let \(X\) be a topological space.

\begin{enumerate}
\def\labelenumi{\alph{enumi}.}
\item
  Prove that \(X\) is connected if and only if there is no continuous
  nonconstant map to the discrete two-point space
  \(\left\{{0, 1}\right\}\).
\item
  Suppose in addition that \(X\) is compact and \(Y\) is a connected
  Hausdorff space. Suppose further that there is a continuous map
  \(f : X \to Y\) such that every preimage \(f^{-1}(y)\) for
  \(y \in Y\), is a connected subset of \(X\).
\end{enumerate}

Show that \(X\) is connected.

\begin{enumerate}
\def\labelenumi{\alph{enumi}.}
\setcounter{enumi}{2}
\tightlist
\item
  Give an example showing that the conclusion of (b) may be false if
  \(X\) is not compact.
\end{enumerate}

\end{problem}

\hypertarget{spring-10}{%
\subsubsection{? (Spring '10)}\label{spring-10}}

\begin{problem}[?]

If \(X\) is a topological space and \(S \subset X\), define in terms of
open subsets of \(X\) what it means for \(S\) \textbf{not} to be
connected.

Show that if \(S\) is not connected there are nonempty subsets
\(A, B \subset X\) such that
\begin{align*}
A \cup B = S {\quad \operatorname{and} \quad} A \cap \mkern 1.5mu\overline{\mkern-1.5muB\mkern-1.5mu}\mkern 1.5mu = \mkern 1.5mu\overline{\mkern-1.5muA\mkern-1.5mu}\mkern 1.5mu \cap B = \emptyset
\end{align*}

\begin{quote}
Here \(\mkern 1.5mu\overline{\mkern-1.5muA\mkern-1.5mu}\mkern 1.5mu\)
and \(\mkern 1.5mu\overline{\mkern-1.5muB\mkern-1.5mu}\mkern 1.5mu\)
denote closure with respect to the topology on the ambient space \(X\).
\end{quote}

\end{problem}

\emph{Concept review omitted.}

\emph{Concept review omitted.}

\emph{Solution omitted.}

\hypertarget{spring-11}{%
\subsubsection{? (Spring '11)}\label{spring-11}}

\begin{problem}[?]

A topological space is \textbf{totally disconnected} if its only
connected subsets are one-point sets.

Is it true that if \(X\) has the discrete topology, it is totally
disconnected?

Is the converse true? Justify your answers.

\end{problem}

\hypertarget{fall-14}{%
\subsubsection{21 (Fall '14)}\label{fall-14}}

\begin{problem}[?]

Let \(X\) and \(Y\) be topological spaces and let \(f : X \to Y\) be a
function.

Suppose that \(X = A \cup B\) where \(A\) and \(B\) are closed subsets,
and that the restrictions \(f \mathrel{\Big|}_A\) and
\(f \mathrel{\Big|}_B\) are continuous (where \(A\) and \(B\) have the
subspace topology).

Prove that \(f\) is continuous.

\end{problem}

\hypertarget{spring-15}{%
\subsubsection{23 (Spring '15)}\label{spring-15}}

\begin{problem}[?]

Define a family \({\mathcal{T}}\) of subsets of \({\mathbb{R}}\) by
saying that \(A \in T\) is \(\iff A = \emptyset\) or
\({\mathbb{R}}\setminus A\) is a finite set.

Prove that \({\mathcal{T}}\) is a topology on \({\mathbb{R}}\), and that
\({\mathbb{R}}\) is compact with respect to this topology.

\end{problem}

\emph{Concept review omitted.}

\emph{Solution omitted.}

\hypertarget{fall-16}{%
\subsubsection{25 (Fall '16)}\label{fall-16}}

\begin{problem}[?]

Let \({\mathcal{S}}, {\mathcal{T}}\) be topologies on a set \(X\). Show
that \({\mathcal{S}}\cap {\mathcal{T}}\) is a topology on \(X\).

Give an example to show that \({\mathcal{S}}\cup {\mathcal{T}}\) need
not be a topology.

\end{problem}

\hypertarget{spring-10-1}{%
\subsubsection{42 (Spring '10)}\label{spring-10-1}}

\begin{problem}[?]

Define an equivalence relation \(\sim\) on \({\mathbb{R}}\) by
\(x \sim y\) if and only if \(x - y \in {\mathbb{Q}}\). Let \(X\) be the
set of equivalence classes, endowed with the quotient topology induced
by the canonical projection \(\pi : {\mathbb{R}}\to X\).

Describe, with proof, all open subsets of \(X\) with respect to this
topology.

\end{problem}

\hypertarget{fall-12}{%
\subsubsection{43 (Fall '12)}\label{fall-12}}

\begin{problem}[?]

Let \(A\) denote a subset of points of \(S^2\) that looks exactly like
the capital letter A. Let \(Q\) be the quotient of \(S^2\) given by
identifying all points of \(A\) to a single point.

Show that \(Q\) is homeomorphic to a familiar topological space and
identify that space.

\end{problem}

\hypertarget{compactness-and-metric-spaces}{%
\subsection{Compactness and Metric
Spaces}\label{compactness-and-metric-spaces}}

\hypertarget{spring-06}{%
\subsubsection{1 (Spring '06)}\label{spring-06}}

\begin{problem}[?]

Suppose \((X, d)\) is a metric space. State criteria for continuity of a
function \(f : X \to X\) in terms of:

\begin{enumerate}
\def\labelenumi{\roman{enumi}.}
\item
  open sets;
\item
  \({\varepsilon}\)'s and \(\delta\)'s; and
\item
  convergent sequences.
\end{enumerate}

Then prove that (iii) implies (i).

\end{problem}

\hypertarget{fall-17}{%
\subsubsection{26 (Fall '17)}\label{fall-17}}

\begin{problem}[?]

Let \(f : X \to Y\) be a continuous function between topological spaces.

Let \(A\) be a subset of \(X\) and let \(f (A)\) be its image in \(Y\) .

One of the following statements is true and one is false. Decide which
is which, prove the true statement, and provide a counterexample to the
false statement:

\begin{enumerate}
\def\labelenumi{\arabic{enumi}.}
\item
  If \(A\) is closed then \(f (A)\) is closed.
\item
  If \(A\) is compact then \(f (A)\) is compact.
\end{enumerate}

\end{problem}

\hypertarget{spring-12}{%
\subsubsection{2 (Spring '12)}\label{spring-12}}

\begin{problem}[?]

Let \(X\) be a topological space.

\begin{enumerate}
\def\labelenumi{\alph{enumi}.}
\item
  State what it means for \(X\) to be compact.
\item
  Let
  \(X = \left\{{0}\right\} \cup \left\{{{1\over n} \mathrel{\Big|}n \in {\mathbb{Z}}^+ }\right\}\).
  Is \(X\) compact?
\item
  Let \(X = (0, 1]\). Is \(X\) compact?
\end{enumerate}

\end{problem}

\todo[inline]{Incomplete proof for part 3.}

\emph{Concept review omitted.}

\emph{Solution omitted.}

\hypertarget{spring-09}{%
\subsubsection{3 (Spring '09)}\label{spring-09}}

\begin{problem}[?]

Let \((X, d)\) be a compact metric space, and let \(f : X \to X\) be an
isometry:
\begin{align*}
\forall~ x, y \in X, \qquad d(f (x), f (y)) = d(x, y)
.\end{align*}
Prove that \(f\) is a bijection.

\end{problem}

\hypertarget{spring-05}{%
\subsubsection{4 (Spring '05)}\label{spring-05}}

\begin{problem}[?]

Suppose \((X, d)\) is a compact metric space and \(U\) is an open
covering of \(X\).

Prove that there is a number \(\delta > 0\) such that for every
\(x \in X\), the ball of radius \(\delta\) centered at \(x\) is
contained in some element of \(U\).

\end{problem}

\emph{Solution omitted.}

\hypertarget{spring-15-1}{%
\subsubsection{44 (Spring '15)}\label{spring-15-1}}

\begin{problem}[?]

\begin{enumerate}
\def\labelenumi{\alph{enumi}.}
\item
  Prove that a topological space that has a countable base for its
  topology also contains a countable dense subset.
\item
  Prove that the converse to (a) holds if the space is a metric space.
\end{enumerate}

\end{problem}

\hypertarget{fall-07-1}{%
\subsubsection{18 (Fall '07)}\label{fall-07-1}}

\begin{problem}[?]

Prove that if \((X, d)\) is a compact metric space, \(f : X \to X\) is a
continuous map, and \(C\) is a constant with \(0 < C < 1\) such that
\begin{align*}
d(f (x), f (y)) \leq C \cdot d(x, y) \quad \forall x, y
,\end{align*}
then \(f\) has a fixed point.

\end{problem}

\emph{Solution omitted.}

\hypertarget{spring-15-2}{%
\subsubsection{19 (Spring '15)}\label{spring-15-2}}

\begin{problem}[?]

Prove that the product of two connected topological spaces is connected.

\end{problem}

\emph{Solution omitted.}

\hypertarget{fall-14-1}{%
\subsubsection{20 (Fall '14)}\label{fall-14-1}}

\begin{problem}[?]

\begin{enumerate}
\def\labelenumi{\alph{enumi}.}
\tightlist
\item
\end{enumerate}

Define what it means for a topological space to be:

\begin{enumerate}
\def\labelenumi{\roman{enumi}.}
\item
  \textbf{Connected}
\item
  \textbf{Locally connected}
\end{enumerate}

\begin{enumerate}
\def\labelenumi{\alph{enumi}.}
\setcounter{enumi}{1}
\tightlist
\item
  Give, with proof, an example of a space that is connected but not
  locally connected.
\end{enumerate}

\end{problem}

\todo[inline]{What's the picture?}

\emph{Concept review omitted.}

\emph{Solution omitted.}

\hypertarget{fall-18}{%
\subsubsection{22 (Fall '18)}\label{fall-18}}

\begin{problem}[?]

Let \(X\) be a compact space and let \(f : X \times R \to R\) be a
continuous function such that \(f (x, 0) > 0\) for all \(x \in X\).

Prove that there is \({\varepsilon}> 0\) such that \(f (x, t) > 0\)
whenever \({\left\lvert {t} \right\rvert} < {\varepsilon}\).

Moreover give an example showing that this conclusion may not hold if
\(X\) is not assumed compact.

\end{problem}

\hypertarget{spring-16}{%
\subsubsection{24 (Spring '16)}\label{spring-16}}

\begin{problem}[?]

In each part of this problem \(X\) is a compact topological space. Give
a proof or a counterexample for each statement.

\begin{enumerate}
\def\labelenumi{\alph{enumi}.}
\item
  If \(\left\{{F_n }\right\}_{n=1}^\infty\) is a sequence of nonempty
  \emph{closed} subsets of \(X\) such that \(F_{n+1} \subset F_{n}\) for
  all \(n\) then
  \begin{align*}\cap^\infty_{n=1} F_n\neq \emptyset.\end{align*}
\item
  If \(\left\{{O_n}\right\}_{n=1}^\infty\) is a sequence of nonempty
  \emph{open} subsets of \(X\) such that \(O_{n+1} \subset O_n\) for all
  \(n\) then
  \begin{align*}\cap_{n=1}^\infty O_{n}\neq \emptyset.\end{align*}
\end{enumerate}

\end{problem}

\hypertarget{fall-17-1}{%
\subsubsection{27 (Fall '17)}\label{fall-17-1}}

\begin{problem}[?]

A metric space is said to be \textbf{totally bounded} if for every
\({\varepsilon}> 0\) there exists a finite cover of \(X\) by open balls
of radius \({\varepsilon}\).

\begin{enumerate}
\def\labelenumi{\alph{enumi}.}
\item
  Show: a metric space \(X\) is totally bounded iff every sequence in
  \(X\) has a Cauchy subsequence.
\item
  Exhibit a complete metric space \(X\) and a closed subset \(A\) of
  \(X\) that is bounded but not totally bounded.
\end{enumerate}

\begin{quote}
You are not required to prove that your example has the stated
properties.
\end{quote}

\end{problem}

\emph{Concept review omitted.}

\emph{Solution omitted.}

\hypertarget{spring-19-1}{%
\subsubsection{Spring '19 \#1}\label{spring-19-1}}

\begin{problem}[?]

Is every complete bounded metric space compact? If so, give a proof; if
not, give a counterexample.

\end{problem}

\todo[inline]{Review, from last year.}

\emph{Concept review omitted.}

\emph{Solution omitted.}

\hypertarget{spring-2019-2}{%
\subsubsection{Spring 2019 \#2}\label{spring-2019-2}}

\begin{problem}[?]

Let \(X\) be Hausdorff, and recall that the \emph{one-point
compactification} \(\tilde X\) is given by the following:

\begin{itemize}
\item
  As a set,
  \(\tilde X \coloneqq X{\textstyle\coprod}\left\{{\infty}\right\}\).
\item
  A subset \(U\subseteq \tilde X\) is open iff either \(U\) is open in
  \(X\) or is of the form
  \(U = V{\textstyle\coprod}\left\{{\infty}\right\}\) where
  \(V\subset X\) is arbitrary and \(X\setminus V\) is compact.
\end{itemize}

Prove that this description defines a topology on \(\tilde X\) making
\(\tilde X\) compact.

\end{problem}

\emph{Concept review omitted.}

\emph{Solution omitted.}

\hypertarget{spring-2021-3}{%
\subsubsection{Spring 2021 \#3}\label{spring-2021-3}}

\begin{problem}[Spring 2021, 3]

For nonempty subsets \(A, B\) of a metric space \((X, d)\), define the
\textbf{setwise distance} as
\begin{align*}
d(A, B) \coloneqq\inf \left\{{ d(a, b) {~\mathrel{\Big\vert}~}a\in A,\, b\in B }\right\} 
.\end{align*}

\begin{enumerate}
\def\labelenumi{\alph{enumi}.}
\item
  Suppose that \(A\) and \(B\) are compact. Show that there is an
  \(a\in A\) and \(b\in B\) such that \(d(A, B) = d(a, b)\).
\item
  Suppose that \(A\) is closed and \(B\) is compact. Show that if
  \(d(A, B) = 0\) then \(A \cap B = \emptyset\).
\item
  Give an example in which \(A\) is closed, \(B\) is compact, and
  \(d(a, b) > d(A, B)\) for all \(a\in A\) and \(b\in B\).
\end{enumerate}

\begin{quote}
Hint: take \(X = \left\{{ 0 }\right\} \cup(1, 2] \subset {\mathbb{R}}\).
Throughout this problem, you may use without proof that the map
\(d:X\times X\to {\mathbb{R}}\) is continuous.
\end{quote}

\end{problem}

\hypertarget{connectedness}{%
\subsection{Connectedness}\label{connectedness}}

\hypertarget{spring-13}{%
\subsubsection{9 (Spring '13)}\label{spring-13}}

\begin{problem}[?]

Recall that a topological space is said to be \textbf{connected} if
there does not exist a pair \(U, V\) of disjoint nonempty subsets whose
union is \(X\).

\begin{enumerate}
\def\labelenumi{\alph{enumi}.}
\item
  Prove that \(X\) is connected if and only if the only subsets of \(X\)
  that are both open and closed are \(X\) and the empty set.
\item
  Suppose that \(X\) is connected and let \(f : X \to {\mathbb{R}}\) be
  a continuous map. If \(a\) and \(b\) are two points of \(X\) and \(r\)
  is a point of \({\mathbb{R}}\) lying between \(f (a)\) and \(f (b)\)
  show that there exists a point \(c\) of \(X\) such that \(f (c) = r\).
\end{enumerate}

\end{problem}

\hypertarget{fall-05-1}{%
\subsubsection{10 (Fall '05)}\label{fall-05-1}}

\begin{problem}[?]

Let
\begin{align*}
X = \left\{{(0, y) \mathrel{\Big|}- 1 \leq y \leq 1}\right\} \cup \left\{{\qty{x, s = \sin\qty{1 \over x}} \mathrel{\Big|}0 < x \leq 1}\right\}
.\end{align*}

Prove that \(X\) is connected but not path connected.

\end{problem}

\emph{Solution omitted.}

\hypertarget{fall-18-1}{%
\subsubsection{11 (Fall '18)}\label{fall-18-1}}

\begin{problem}[?]

Let
\begin{align*} X=\left\{(x, y) \in \mathbb{R}^{2} | x>0, y \geq 0, \text { and } \frac{y}{x} \text { is rational }\right\} \end{align*}
and equip \(X\) with the subspace topology induced by the usual topology
on \({\mathbb{R}}^2\).

Prove or disprove that \(X\) is connected.

\end{problem}

\todo[inline]{Not convincing..}

\emph{Solution omitted.}

\hypertarget{spring-2021-2}{%
\subsubsection{Spring 2021 \#2}\label{spring-2021-2}}

\begin{problem}[Spring 2021, 2]

Let \(X \coloneqq\prod_{==1}^{\infty} \left\{{ 0, 1 }\right\}\) endowed
with the product topology.

\begin{enumerate}
\def\labelenumi{\alph{enumi}.}
\item
  Show that for all points \(x,y\in X\) with \(x\neq y\), there are open
  subsets \(U_x, U_y \subset X\) such that \(x\in U_x, y\in U_y\), with
  \(U_x \cup U_y = X\) and \(U_x \cap U_y = \emptyset\).
\item
  Show that \(X\) is totally disconnected, i.e.~the only nonempty
  connected subsets of \(X\) are singletons.
\end{enumerate}

\end{problem}

\hypertarget{hausdorff-spaces-and-separation}{%
\subsection{Hausdorff Spaces and
Separation}\label{hausdorff-spaces-and-separation}}

\hypertarget{fall-14-2}{%
\subsubsection{29 (Fall '14)}\label{fall-14-2}}

\begin{problem}[?]

Is every product (finite or infinite) of Hausdorff spaces Hausdorff? If
yes, prove it. If no, give a counterexample.

\end{problem}

\hypertarget{spring-18}{%
\subsubsection{30 (Spring '18)}\label{spring-18}}

\begin{problem}[?]

Suppose that \(X\) is a Hausdorff topological space and that
\(A \subset X\). Prove that if \(A\) is compact in the subspace topology
then \(A\) is closed as a subset of X.

\end{problem}

\emph{Solution omitted.}

\hypertarget{spring-09-1}{%
\subsubsection{31 (Spring '09)}\label{spring-09-1}}

\begin{problem}[Spring 2009, 31]

\envlist

\begin{enumerate}
\def\labelenumi{\alph{enumi}.}
\item
  Show that a continuous bijection from a compact space to a Hausdorff
  space is a homeomorphism.
\item
  Give an example that shows that the ``Hausdorff'' hypothesis in part
  (a) is necessary.
\end{enumerate}

\end{problem}

\emph{Concept review omitted.}

\emph{Solution omitted.}

\hypertarget{fall-14-3}{%
\subsubsection{32 (Fall '14)}\label{fall-14-3}}

\begin{problem}[?]

Let \(X\) be a topological space and let
\begin{align*}
\Delta = \left\{{(x, y) \in X \times X \mathrel{\Big|}x = y}\right\}
.\end{align*}

Show that \(X\) is a Hausdorff space if and only if \(\Delta\) is closed
in \(X \times X\).

\end{problem}

\emph{Solution omitted.}

\hypertarget{fall-06}{%
\subsubsection{33 (Fall '06)}\label{fall-06}}

\begin{problem}[?]

If \(f\) is a function from \(X\) to \(Y\) , consider the graph
\begin{align*}
G = \left\{{(x, y) \in X \times Y \mathrel{\Big|}f (x) = y}\right\}
.\end{align*}

\begin{enumerate}
\def\labelenumi{\alph{enumi}.}
\item
  Prove that if \(f\) is continuous and \(Y\) is Hausdorff, then \(G\)
  is a closed subset of \(X \times Y\).
\item
  Prove that if \(G\) is closed and \(Y\) is compact, then \(f\) is
  continuous.
\end{enumerate}

\end{problem}

\hypertarget{fall-04}{%
\subsubsection{34 (Fall '04)}\label{fall-04}}

\begin{problem}[?]

Let X be a noncompact locally compact Hausdorff space, with topology
\({\mathcal{T}}\). Let \(\tilde X = X \cup \left\{{\infty}\right\}\)
(\(X\) with one point adjoined), and consider the family
\({\mathcal{B}}\) of subsets of \(\tilde X\) defined by
\begin{align*}
{\mathcal{B}}= T \cup \left\{{S \cup \left\{{\infty}\right\}\mathrel{\Big|}S \subset X,~~ X \backslash S \text{ is compact}}\right\}
.\end{align*}

\begin{enumerate}
\def\labelenumi{\alph{enumi}.}
\item
  Prove that \({\mathcal{B}}\) is a topology on \(\tilde X\), that the
  resulting space is compact, and that \(X\) is dense in \(\tilde X\).
\item
  Prove that if \(Y \supset X\) is a compact space such that \(X\) is
  dense in \(Y\) and \(Y \backslash X\) is a singleton, then Y is
  homeomorphic to \(\tilde X\).
\end{enumerate}

\begin{quote}
The space \(\tilde X\) is called the \textbf{one-point compactification}
of \(X\).
\end{quote}

\begin{enumerate}
\def\labelenumi{\alph{enumi}.}
\setcounter{enumi}{2}
\item
  Find familiar spaces that are homeomorphic to the one point
  compactifications of
\item
  \(X = (0, 1)\) and
\end{enumerate}

\begin{enumerate}
\def\labelenumi{\roman{enumi}.}
\setcounter{enumi}{1}
\tightlist
\item
  \(X = {\mathbb{R}}^2\).
\end{enumerate}

\end{problem}

\hypertarget{fall-16-1}{%
\subsubsection{35 (Fall '16)}\label{fall-16-1}}

\begin{problem}[?]

Prove that a metric space \(X\) is \textbf{normal}, i.e.~if
\(A, B \subset X\) are closed and disjoint then there exist open sets
\(A \subset U \subset X, ~B \subset V \subset X\) such that
\(U \cap V = \emptyset\).

\end{problem}

\hypertarget{spring-06-1}{%
\subsubsection{36 (Spring '06)}\label{spring-06-1}}

\begin{problem}[?]

Prove that every compact, Hausdorff topological space is normal.

\end{problem}

\hypertarget{spring-09-2}{%
\subsubsection{37 (Spring '09)}\label{spring-09-2}}

\begin{problem}[?]

Show that a connected, normal topological space with more than a single
point is uncountable.

\end{problem}

\hypertarget{spring-08}{%
\subsubsection{38 (Spring '08)}\label{spring-08}}

\begin{problem}[?]

Give an example of a quotient map in which the domain is Hausdorff, but
the quotient is not.

\end{problem}

\emph{Solution omitted.}

\hypertarget{fall-04-1}{%
\subsubsection{39 (Fall '04)}\label{fall-04-1}}

\begin{problem}[?]

Let \(X\) be a compact Hausdorff space and suppose
\(R \subset X \times X\) is a closed equivalence relation. Show that the
quotient space \(X/R\) is Hausdorff.

\end{problem}

\hypertarget{spring-18-1}{%
\subsubsection{40 (Spring '18)}\label{spring-18-1}}

\begin{problem}[?]

Let \(U \subset {\mathbb{R}}^n\) be an open set which is bounded in the
standard Euclidean metric. Prove that the quotient space
\({\mathbb{R}}^n / U\) is not Hausdorff.

\end{problem}

\hypertarget{fall-09}{%
\subsubsection{41 (Fall '09)}\label{fall-09}}

\begin{problem}[?]

Let \(A\) be a closed subset of a normal topological space \(X\). Show
that both \(A\) and the quotient \(X/A\) are normal.

\end{problem}

\hypertarget{spring-11-1}{%
\subsubsection{45 (Spring '11)}\label{spring-11-1}}

\begin{problem}[?]

Recall that a topological space is \textbf{regular} if for every point
\(p \in X\) and for every closed subset \(F \subset X\) not containing
\(p\), there exist disjoint open sets \(U, V \subset X\) with
\(p \in U\) and \(F \subset V\).

Let \(X\) be a regular space that has a countable basis for its
topology, and let \(U\) be an open subset of \(X\).

\begin{enumerate}
\def\labelenumi{\alph{enumi}.}
\item
  Show that \(U\) is a countable union of closed subsets of \(X\).
\item
  Show that there is a continuous function \(f : X \to [0,1]\) such that
  \(f (x) > 0\) for \(x \in U\) and \(f (x) = 0\) for \(x \in U\).
\end{enumerate}

\end{problem}

\hypertarget{exercises}{%
\subsection{Exercises}\label{exercises}}

\hypertarget{basics}{%
\subsubsection{Basics}\label{basics}}

\hypertarget{exercise}{%
\paragraph{Exercise}\label{exercise}}

Show that for \(A\subseteq X\), \({ \operatorname{cl}} _X(A)\) is the
smallest closed subset containing \(A\).

\hypertarget{exercise-1}{%
\paragraph{Exercise}\label{exercise-1}}

Give an example of spaces \(A\subseteq B \subseteq X\) such that \(A\)
is open in \(B\) but \(A\) is \emph{not} open in \(X\).

\emph{Solution omitted.}

\hypertarget{exercise-2}{%
\paragraph{Exercise}\label{exercise-2}}

Show that the diagonal map \(\Delta(x) = (x, x)\) is continuous.

\hypertarget{exercise-3}{%
\paragraph{Exercise}\label{exercise-3}}

Show that if \(A_i \subseteq X\), then
\({ \operatorname{cl}} _X(\cup_i A_i) = \cup_i { \operatorname{cl}} _X(A_i)\).

\hypertarget{exercise-4}{%
\paragraph{Exercise}\label{exercise-4}}

Show that \({\mathbb{R}}\) is not homeomorphic to \([0, \infty)\).

\hypertarget{exercise-5}{%
\paragraph{Exercise}\label{exercise-5}}

Show that the set \((x, y) \in {\mathbb{R}}^2\) such that at least one
of \(x, y\) is rational with the subspace topology is a connected space.

\hypertarget{exercise-mathbbrmathbbq-is-indiscrete}{%
\paragraph{\texorpdfstring{Exercise: \({\mathbb{R}}/{\mathbb{Q}}\) is
indiscrete}{Exercise: \{\textbackslash mathbb\{R\}\}/\{\textbackslash mathbb\{Q\}\} is indiscrete}}\label{exercise-mathbbrmathbbq-is-indiscrete}}

\begin{problem}[?]

Show that \({\mathbb{R}}/{\mathbb{Q}}\) has the indiscrete topology.

\end{problem}

\emph{Solution omitted.}

\hypertarget{connectedness-1}{%
\subsubsection{Connectedness}\label{connectedness-1}}

\hypertarget{exercise-6}{%
\paragraph{Exercise}\label{exercise-6}}

Prove that \(X\) is connected iff the only clopen subsets are
\(\emptyset, X\).

\hypertarget{exercise-7}{%
\paragraph{Exercise}\label{exercise-7}}

Let \(A \subset X\) be a connected subspace.

Show that if \(B\subset X\) satisfies
\(A\subseteq B \subseteq \mkern 1.5mu\overline{\mkern-1.5muA\mkern-1.5mu}\mkern 1.5mu\),
then \(B\) is connected.

\hypertarget{exercise-8}{%
\paragraph{Exercise}\label{exercise-8}}

Show that: - Connected does not imply path connected - Connected and
locally path connected \emph{does} imply path connected - Path connected
implies connected

\hypertarget{exercise-9}{%
\paragraph{Exercise}\label{exercise-9}}

Use the fact that intervals are connected to prove the intermediate
value theorem.

\hypertarget{exercise-10}{%
\paragraph{Exercise}\label{exercise-10}}

Prove that the continuous image of a connected set is connected.

\hypertarget{exercise-11}{%
\paragraph{Exercise}\label{exercise-11}}

Show that if \(X\) is locally path connected, then

\begin{itemize}
\tightlist
\item
  Every open subset of \(X\) is again locally path-connected.
\item
  \(X\) is connected \(\iff X\) is path-connected.
\item
  Every path component of \(X\) is a connected component of \(X\).
\item
  Every connected component of \(X\) is open in \(X\).
\end{itemize}

\hypertarget{exercise-12}{%
\paragraph{Exercise}\label{exercise-12}}

Show that \([0, 1]\) is connected.

\emph{Solution omitted.}

\hypertarget{compactness}{%
\subsubsection{Compactness}\label{compactness}}

\hypertarget{star-exercise}{%
\paragraph{\texorpdfstring{\(\star\)
Exercise}{\textbackslash star Exercise}}\label{star-exercise}}

Let \(X\) be a compact space and let \(A\) be a closed subspace. Show
that \(A\) is compact.

\emph{Solution omitted.}

\hypertarget{star-exercise-1}{%
\paragraph{\texorpdfstring{\(\star\)
Exercise}{\textbackslash star Exercise}}\label{star-exercise-1}}

Let \(f : X \to Y\) be a continuous function, with \(X\) compact. Show
that \(f(X)\) is compact.

\emph{Solution omitted.}

Let \(A\) be a compact subspace of a Hausdorff space \(X\). Show that
\(A\) is closed.

\hypertarget{exercise-13}{%
\paragraph{Exercise}\label{exercise-13}}

Show that any infinite set with the cofinite topology is compact.

\hypertarget{exercise-14}{%
\paragraph{Exercise}\label{exercise-14}}

Show that every compact metric space is complete.

\hypertarget{exercise-15}{%
\paragraph{Exercise}\label{exercise-15}}

Show that if \(X\) is second countable and Hausdorff, or a metric space,
then TFAE:

\begin{itemize}
\tightlist
\item
  \(X\) is compact
\item
  Every infinite subset \(A\subseteq X\) has a limit point in \(X\).
\item
  Every sequence in \(X\) has a convergent subsequence in \(X\).
\end{itemize}

\hypertarget{exercise-16}{%
\paragraph{Exercise}\label{exercise-16}}

Show that if \(f: A\to B\) is a continuous map between metric spaces and
\(K\subset A\) is compact, then \({\left.{{f}} \right|_{{K}} }\) is
uniformly continuous.

\hypertarget{exercise-17}{%
\paragraph{Exercise}\label{exercise-17}}

Show that if \(f:X\to Y\) is continuous and \(X\) is compact then
\(f(X)\) is compact.

\hypertarget{exercise-18}{%
\paragraph{Exercise}\label{exercise-18}}

Show that if \(f:X\to {\mathbb{R}}\) and \(X\) is compact then \(f\) is
bounded and attains its min/max.

\hypertarget{exercise-19}{%
\paragraph{Exercise}\label{exercise-19}}

Show that a finite product or union compact spaces is again compact.

\hypertarget{exercise-20}{%
\paragraph{Exercise}\label{exercise-20}}

Show that a quotient of a compact space is again compact.

\hypertarget{exercise-21}{%
\paragraph{Exercise}\label{exercise-21}}

Show that if \(X\) is compact and \(A\subseteq X\) is closed then \(A\)
is compact.

\hypertarget{exercise-22}{%
\paragraph{Exercise}\label{exercise-22}}

Show that if \(X\) is Hausdorff and \(A\subseteq X\) is compact then
\(A\) is closed.

\hypertarget{exercise-23}{%
\paragraph{Exercise}\label{exercise-23}}

Show that if \(X\) is a metric space and \(A\subseteq X\) is compact
then \(A\) is bounded.

\hypertarget{exercise-24}{%
\paragraph{Exercise}\label{exercise-24}}

Show that a continuous map from a compact space to a Hausdorff space is
closed.

\hypertarget{exercise-25}{%
\paragraph{Exercise}\label{exercise-25}}

Show that an injective continuous map from a compact space to a
Hausdorff space is an embedding (a homeomorphism onto its image).

\hypertarget{exercise-26}{%
\paragraph{Exercise}\label{exercise-26}}

Show that \([0, 1]\) is compact.

\hypertarget{exercise-27}{%
\paragraph{Exercise}\label{exercise-27}}

Show that a compact Hausdorff space is is metrizable iff it is
second-countable.

\hypertarget{exercise-28}{%
\paragraph{Exercise}\label{exercise-28}}

Show that if \(X\) is metrizable, then \(X\) is compact

\hypertarget{exercise-29}{%
\paragraph{Exercise}\label{exercise-29}}

Give an example of a space that is compact but not sequentially compact,
and vice versa.

\hypertarget{exercise-30}{%
\paragraph{Exercise}\label{exercise-30}}

Show that a sequentially compact space is totally bounded.

\hypertarget{exercise-31}{%
\paragraph{Exercise}\label{exercise-31}}

Show that \({\mathbb{R}}\) with the cofinite topology is compact.

\hypertarget{exercise-32}{%
\paragraph{Exercise}\label{exercise-32}}

Show that \([0, 1]\) is compact without using the Heine-Borel theorem.

\hypertarget{separation}{%
\subsubsection{Separation}\label{separation}}

\hypertarget{exercise-33}{%
\paragraph{Exercise}\label{exercise-33}}

Show that \(X\) is Hausdorff iff \(\Delta(X)\) is closed in
\(X\times X\).

\hypertarget{exercise-34}{%
\paragraph{Exercise}\label{exercise-34}}

Prove that \(X, Y\) are Hausdorff iff \(X\times Y\) is Hausdorff.

\hypertarget{exercise-35}{%
\paragraph{Exercise}\label{exercise-35}}

Show that \({\mathbb{R}}\) is separable.

\hypertarget{exercise-36}{%
\paragraph{Exercise}\label{exercise-36}}

Show that any space with the indiscrete topology is separable.

\hypertarget{exercise-37}{%
\paragraph{Exercise}\label{exercise-37}}

Show that any countable space with the discrete topology is separable.

\hypertarget{exercise-38}{%
\paragraph{Exercise}\label{exercise-38}}

Show that the minimal uncountable order with the order topology is not
separable.

\hypertarget{exercise-39}{%
\paragraph{Exercise}\label{exercise-39}}

Show that every first countable space is second countable.

\hypertarget{exercise-40}{%
\paragraph{Exercise}\label{exercise-40}}

Show that every metric space is Hausdorff in its metric topology.

\hypertarget{hausdorff-spaces}{%
\subsubsection{Hausdorff Spaces}\label{hausdorff-spaces}}

\hypertarget{exercise-41}{%
\paragraph{Exercise}\label{exercise-41}}

Show that a compact set in a Hausdorff space is closed.

\hypertarget{exercise-42}{%
\paragraph{Exercise}\label{exercise-42}}

Let \(A\subset X\) with \(A\) closed and \(X\) compact, and show that
\(A\) is compact.

\emph{Concept review omitted.}

\emph{Solution omitted.}

\hypertarget{exercise-43}{%
\paragraph{Exercise}\label{exercise-43}}

Show that a continuous bijection from a compact space to a Hausdorff
space is a homeomorphism.

\emph{Solution omitted.}

\hypertarget{exercise-44}{%
\paragraph{Exercise}\label{exercise-44}}

Show that a closed subset of a Hausdorff space need not be compact.

\hypertarget{exercise-45}{%
\paragraph{Exercise}\label{exercise-45}}

Show that in a \emph{compact} Hausdorff space, \(A\) is closed iff \(A\)
is compact.

\hypertarget{exercise-46}{%
\paragraph{Exercise}\label{exercise-46}}

Show that a local homeomorphism between compact Hausdorff spaces is a
covering space.

\hypertarget{the-fundamental-group}{%
\section{The Fundamental Group}\label{the-fundamental-group}}

\hypertarget{spring-15-3}{%
\subsection{1 (Spring '15)}\label{spring-15-3}}

\begin{problem}[?]

Let \(S^1\) denote the unit circle in \(C\), \(X\) be any topological
space, \(x_0 \in X\), and
\begin{align*}\gamma_0, \gamma_1 : S^1 \to X\end{align*}
be two continuous maps such that \(\gamma_0 (1) = \gamma_1 (1) = x_0\).

Prove that \(\gamma_0\) is homotopic to \(\gamma_1\) if and only if the
elements represented by \(\gamma_0\) and \(\gamma_1\) in
\(\pi_1 (X, x_0 )\) are conjugate.

\end{problem}

\emph{Concept review omitted.}

\emph{Solution omitted.}

\hypertarget{spring-09spring-07fall-07fall-06}{%
\subsection{2 (Spring '09/Spring '07/Fall '07/Fall
'06)}\label{spring-09spring-07fall-07fall-06}}

\begin{enumerate}
\def\labelenumi{\alph{enumi}.}
\item
  State van Kampen's theorem.
\item
  Calculate the fundamental group of the space obtained by taking two
  copies of the torus \(T = S^1 \times S^1\) and gluing them along a
  circle \(S^1 \times {p}\) where \(p\) is a point in \(S^1\).
\item
  Calculate the fundamental group of the Klein bottle.
\item
  Calculate the fundamental group of the one-point union of
  \(S^1 \times S^1\) and \(S^1\).
\item
  Calculate the fundamental group of the one-point union of
  \(S^1 \times S^1\) and \({\mathbb{RP}}^2\).
\end{enumerate}

\begin{quote}
\textbf{Note: multiple appearances!!}
\end{quote}

\hypertarget{fall-18-2}{%
\subsection{3 (Fall '18)}\label{fall-18-2}}

Prove the following portion of van Kampen's theorem. If \(X = A\cup B\)
and \(A\), \(B\), and \(A \cap B\) are nonempty and path connected with
\({\operatorname{pt}}\in A \cap B\), then there is a surjection
\begin{align*}
\pi_1 (A, {\operatorname{pt}}) \ast \pi_1 (B, {\operatorname{pt}}) \to \pi_1 (X, {\operatorname{pt}})
.\end{align*}

\hypertarget{spring-15-4}{%
\subsection{4 (Spring '15)}\label{spring-15-4}}

Let \(X\) denote the quotient space formed from the sphere \(S^2\) by
identifying two distinct points.

Compute the fundamental group and the homology groups of \(X\).

\hypertarget{spring-06-2}{%
\subsection{5 (Spring '06)}\label{spring-06-2}}

Start with the unit disk \({\mathbb{D}}^2\) and identify points on the
boundary if their angles, thought of in polar coordinates, differ a
multiple of \(\pi/2\).

Let \(X\) be the resulting space. Use van Kampen's theorem to compute
\(\pi_1 (X, \ast)\).

\hypertarget{spring-08-1}{%
\subsection{6 (Spring '08)}\label{spring-08-1}}

Let \(L\) be the union of the \(z\)-axis and the unit circle in the
\(xy{\hbox{-}}\)plane. Compute
\(\pi_1 ({\mathbb{R}}^3 \backslash L, \ast)\).

\hypertarget{fall-16-2}{%
\subsection{7 (Fall '16)}\label{fall-16-2}}

Let \(A\) be the union of the unit sphere in \({\mathbb{R}}^3\) and the
interval
\(\left\{{(t, 0, 0) : -1 \leq t \leq 1}\right\} \subset {\mathbb{R}}^3\).

Compute \(\pi_1 (A)\) and give an explicit description of the universal
cover of \(X\).

\hypertarget{spring-13-1}{%
\subsection{8 (Spring '13)}\label{spring-13-1}}

\begin{enumerate}
\def\labelenumi{\alph{enumi}.}
\item
  Let \(S_1\) and \(S_2\) be disjoint surfaces. Give the definition of
  their connected sum \(S^1 \#S^2\).
\item
  Compute the fundamental group of the connected sum of the projective
  plane and the two-torus.
\end{enumerate}

\hypertarget{fall-15}{%
\subsection{9 (Fall '15)}\label{fall-15}}

Compute the fundamental group, using any technique you like, of
\({\mathbb{RP}}^2 \#{\mathbb{RP}}^2 \#{\mathbb{RP}}^2\).

\hypertarget{fall-11-1}{%
\subsection{10 (Fall '11)}\label{fall-11-1}}

Let
\begin{align*}
V = {\mathbb{D}}^2 \times S^1 = \left\{{ (z, e^{it}) {~\mathrel{\Big\vert}~}{\left\lVert {z} \right\rVert} \leq 1,~~ 0 \leq t < 2\pi}\right\}
\end{align*}
be the ``solid torus'' with boundary given by the torus
\(T = S^1 \times S^1\) .

For \(n \in {\mathbb{Z}}\) define
\begin{align*} \phi_n : T &\to T \\ (e^{is} , e^{it} ) &\mapsto (e^{is} , e^{i(ns+t)}) .\end{align*}

Find the fundamental group of the identification space
\begin{align*}
V_n = {V{\textstyle\coprod}V \over \sim n}
\end{align*}
where the equivalence relation \(\sim_n\) identifies a point \(x\) on
the boundary \(T\) of the first copy of \(V\) with the point
\(\phi_n (x)\) on the boundary of the second copy of \(V\).

\hypertarget{fall-16-3}{%
\subsection{11 (Fall '16)}\label{fall-16-3}}

Let \(S_k\) be the space obtained by removing \(k\) disjoint open disks
from the sphere \(S^2\). Form \(X_k\) by gluing \(k\) Möbius bands onto
\(S_k\) , one for each circle boundary component of \(S_k\) (by
identifying the boundary circle of a Möbius band homeomorphically with a
given boundary component circle).

Use van Kampen's theorem to calculate \(\pi_1 (X_k)\) for each \(k > 0\)
and identify \(X_k\) in terms of the classification of surfaces.

\hypertarget{spring-13-2}{%
\subsection{12 (Spring '13)}\label{spring-13-2}}

\begin{enumerate}
\def\labelenumi{\alph{enumi}.}
\item
  Let \(A\) be a subspace of a topological space \(X\). Define what it
  means for \(A\) to be a \textbf{deformation retract} of \(X\).
\item
  Consider \(X_1\) the ``planar figure eight'' and
  \begin{align*}X_2 = S^1 \cup ({0} \times [-1, 1])\end{align*}
  (the ``theta space''). Show that \(X_1\) and \(X_2\) have isomorphic
  fundamental groups.
\item
  Prove that the fundamental group of \(X_2\) is a free group on two
  generators.
\end{enumerate}

\hypertarget{spring-2021-4}{%
\subsection{Spring 2021 \#4}\label{spring-2021-4}}

\begin{problem}[Spring 2021, 4]

Suppose that \(X\) is a topological space and \(x_0\in X\), and suppose
that every continuous map \(\gamma: S^1 \to X\) is freely homotopic to
the constant map to \(x_0\). Prove that
\(\pi_1(X, x_0) = \left\{{ e }\right\}\).

\begin{quote}
Note that ``freely'' means there are no conditions on basepoints.
\end{quote}

\end{problem}

\hypertarget{covering-spaces}{%
\section{Covering Spaces}\label{covering-spaces}}

\hypertarget{spring-11spring-14}{%
\subsection{1 (Spring 11/Spring '14)}\label{spring-11spring-14}}

\begin{enumerate}
\def\labelenumi{\alph{enumi}.}
\item
  Give the definition of a \textbf{covering space} \(\widehat{X}\) (and
  \textbf{covering map} \(p : \widehat{X} \to X\)) for a topological
  space \(X\).
\item
  State the homotopy lifting property of covering spaces. Use it to show
  that a covering map \(p : \widehat{X} \to X\) induces an injection
  \begin{align*}
  p^\ast : \pi_1 (\widehat{X}, \widehat{x}) \to \pi_1 (X, p(\widehat{x}))
  \end{align*}
  on fundamental groups.
\item
  Let \(p : \widehat{X} \to X\) be a covering map with \(Y\) and \(X\)
  path-connected. Suppose that the induced map \(p^\ast\) on \(\pi_1\)
  is an isomorphism.
\end{enumerate}

Prove that \(p\) is a homeomorphism.

\todo[inline]{Not done?}

\emph{Solution omitted.}

\hypertarget{fall-06fall-09fall-15}{%
\subsection{2 (Fall '06/Fall '09/Fall
'15)}\label{fall-06fall-09fall-15}}

\begin{enumerate}
\def\labelenumi{\alph{enumi}.}
\item
  Give the definitions of \textbf{covering space} and \textbf{deck
  transformation} (or covering transformation).
\item
  Describe the universal cover of the Klein bottle and its group of deck
  transformations.
\item
  Explicitly give a collection of deck transformations on
  \begin{align*}\left\{{(x, y) \mathrel{\Big|}-1 \leq x \leq 1, -\infty < y < \infty}\right\}\end{align*}
  such that the quotient is a Möbius band.
\item
  Find the universal cover of \({\mathbb{RP}}^2 \times S^1\) and
  explicitly describe its group of deck transformations.
\end{enumerate}

\hypertarget{spring-2021-5}{%
\subsection{Spring 2021 \#5}\label{spring-2021-5}}

\begin{problem}[Spring 2021, 5]

Identify five mutually non-homeomorphic connected spaces \(X\) for which
there is a covering map \(p:X\to K\) where \(K\) is the Klein bottle.
Give an example of the covering in each case.

\end{problem}

\hypertarget{spring-06spring-07spring-12}{%
\subsection{3 (Spring '06/Spring '07/Spring
'12)}\label{spring-06spring-07spring-12}}

\begin{enumerate}
\def\labelenumi{\alph{enumi}.}
\item
  What is the definition of a \textbf{regular} (or Galois) covering
  space?
\item
  State, without proof, a criterion in terms of the fundamental group
  for a covering map \(p : \tilde X \to X\) to be regular.
\item
  Let \(\Theta\) be the topological space formed as the union of a
  circle and its diameter (so this space looks exactly like the letter
  \(\Theta\)). Give an example of a covering space of \(\Theta\) that is
  not regular.
\end{enumerate}

\hypertarget{spring-08-2}{%
\subsection{4 (Spring '08)}\label{spring-08-2}}

Let \(S\) be the closed orientable surface of genus 2 and let \(C\) be
the commutator subgroup of \(\pi_1 (S, \ast)\). Let \(\tilde S\) be the
cover corresponding to \(C\). Is the covering map \(\tilde S \to S\)
regular?

\begin{quote}
The term ``normal'' is sometimes used as a synonym for regular in this
context.
\end{quote}

What is the group of deck transformations?

Give an example of a nontrivial element of \(\pi_1 (S, \ast)\) which
lifts to a trivial deck transformation.

\hypertarget{fall-04-2}{%
\subsection{5 (Fall '04)}\label{fall-04-2}}

Describe the 3-fold connected covering spaces of \(S^1 \lor S^1\).

\hypertarget{spring-17}{%
\subsection{6 (Spring '17)}\label{spring-17}}

Find all three-fold covers of the wedge of two copies of
\({\mathbb{RP}}^2\) . Justify your answer.

\emph{Solution omitted.}

\hypertarget{fall-17-2}{%
\subsection{7 (Fall '17)}\label{fall-17-2}}

Describe, as explicitly as you can, two different (non-homeomorphic)
connected two-sheeted covering spaces of
\({\mathbb{RP}}^2 \lor {\mathbb{RP}}^3\), and prove that they are not
homeomorphic.

\todo[inline]{Expand solution.}

\emph{Solution omitted.}

\hypertarget{spring-19}{%
\subsection{8 (Spring '19)}\label{spring-19}}

Is there a covering map from
\begin{align*}
X_3 = \left\{{x^2 + y^2 = 1}\right\} \cup \left\{{(x - 2)^2 + y^2 = 1}\right\} \cup \left\{{(x + 2)^2 + y^2 = 1}\right\} \subset {\mathbb{R}}^2
\end{align*}
to \(S^1 \vee S^1\)? If there is, give an example; if not, give a proof.

\emph{Solution omitted.}

\hypertarget{spring-05-1}{%
\subsection{9 (Spring '05)}\label{spring-05-1}}

\begin{enumerate}
\def\labelenumi{\alph{enumi}.}
\item
  Suppose \(Y\) is an \(n\)-fold connected covering space of the torus
  \(S^1 \times S^1\). Up to homeomorphism, what is \(Y\)? Justify your
  answer.
\item
  Let \(X\) be the topological space obtained by deleting a disk from a
  torus. Suppose \(Y\) is a 3-fold covering space of \(X\).

  What surfaces could \(Y\) be? Justify your answer, but you need not
  exhibit the covering maps explicitly.
\end{enumerate}

\hypertarget{spring-07}{%
\subsection{10 (Spring '07)}\label{spring-07}}

Let \(S\) be a connected surface, and let \(U\) be a connected open
subset of \(S\). Let \(p : \tilde S \to S\) be the universal cover of
\(S\). Show that \(p^{-1}(U )\) is connected if and only if the
homeomorphism \(i_\ast : \pi_1 (U ) \to \pi_1 (S)\) induced by the
inclusion \(i : U \to S\) is onto.

\hypertarget{fall-10}{%
\subsection{11 (Fall '10)}\label{fall-10}}

Suppose that X has universal cover \(p : \tilde X \to X\) and let
\(A \subset X\) be a subspace with \(p(\tilde a) = a \in A\). Show that
there is a group isomorphism
\begin{align*}
\ker(\pi_1 (A, a) \to \pi_1 (X, a)) \cong \pi_1 (p^{-1}A, \mkern 1.5mu\overline{\mkern-1.5mua\mkern-1.5mu}\mkern 1.5mu)
.\end{align*}

\hypertarget{fall-14-4}{%
\subsection{12 (Fall '14)}\label{fall-14-4}}

Prove that every continuous map \(f : {\mathbb{RP}}^2 \to S^1\) is
homotopic to a constant.

\begin{quote}
Hint: think about covering spaces.
\end{quote}

\hypertarget{spring-16-1}{%
\subsection{13 (Spring '16)}\label{spring-16-1}}

Prove that the free group on two generators contains a subgroup
isomorphic to the free group on five generators by constructing an
appropriate covering space of \(S^1 \lor S^1\).

\hypertarget{fall-12-1}{%
\subsection{14 (Fall '12)}\label{fall-12-1}}

Use covering space theory to show that
\({\mathbb{Z}}_2 \ast {\mathbb{Z}}\) (that is, the free product of
\({\mathbb{Z}}_2\) and \({\mathbb{Z}}\)) has two subgroups of index 2
which are not isomorphic to each other.

\hypertarget{spring-17-1}{%
\subsection{15 (Spring '17)}\label{spring-17-1}}

\begin{enumerate}
\def\labelenumi{\alph{enumi}.}
\item
  Show that any finite index subgroup of a finitely generated free group
  is free. State clearly any facts you use about the fundamental groups
  of graphs.
\item
  Prove that if \(N\) is a nontrivial normal subgroup of infinite index
  in a finitely generated free group \(F\) , then \(N\) is not finitely
  generated.
\end{enumerate}

\hypertarget{spring-19-2}{%
\subsection{16 (Spring '19)}\label{spring-19-2}}

Let \(p : X \to Y\) be a covering space, where \(X\) is compact,
path-connected, and locally path-connected.

Prove that for each \(x \in X\) the set
\(p^{-1}(\left\{{p(x)}\right\})\) is finite, and has cardinality equal
to the index of \(p_* (\pi_1 (X, x))\) in \(\pi_1 (Y, p(x))\).

\hypertarget{cell-complexes-and-adjunction-spaces}{%
\section{Cell Complexes and Adjunction
Spaces}\label{cell-complexes-and-adjunction-spaces}}

\hypertarget{fall-07-2}{%
\subsection{1 (Fall '07)}\label{fall-07-2}}

Describe a cell complex structure on the torus \(T = S^1 \times S^1\)
and use this to compute the homology groups of \(T\).

\begin{quote}
To justify your answer you will need to consider the attaching maps in
detail.
\end{quote}

\hypertarget{fall-04-3}{%
\subsection{2 (Fall '04)}\label{fall-04-3}}

Let \(X\) be the space formed by identifying the boundary of a Möbius
band with a meridian of the torus \(T^2\).

Compute \(\pi_1 (X)\) and \(H_* (X)\).

\hypertarget{spring-06-3}{%
\subsection{3 (Spring '06)}\label{spring-06-3}}

Compute the homology of the space \(X\) obtained by attaching a Möbius
band to \({\mathbb{RP}}^2\) via a homeomorphism of its boundary circle
to the standard \({\mathbb{RP}}^1\) in \({\mathbb{RP}}^2\).

\hypertarget{spring-14}{%
\subsection{4 (Spring '14)}\label{spring-14}}

Let \(X\) be a space obtained by attaching two 2-cells to the torus
\(S^1 \times S^1\), one along a simple closed curve
\(\left\{{x}\right\} \times S^1\) and the other along
\(\left\{{y}\right\} \times S^1\) for two points \(x \neq y\) in \(S^1\)
.

\hypertarget{a}{%
\subsubsection{a}\label{a}}

Draw an embedding of \(X\) in \({\mathbb{R}}^3\) and calculate its
fundamental group.

\hypertarget{b}{%
\subsubsection{b}\label{b}}

Calculate the homology groups of \(X\).

\hypertarget{fall-07-3}{%
\subsection{5 (Fall '07)}\label{fall-07-3}}

Let \(X\) be the space obtained as the quotient of a disjoint union of a
2-sphere \(S^2\) and a torus \(T = S^1 \times S^1\) by identifying the
equator in \(S^2\) with a circle \(S^1 \times \left\{{p}\right\}\) in
\(T\).

Compute the homology groups of \(X\).

\hypertarget{spring-06-4}{%
\subsection{6 (Spring '06)}\label{spring-06-4}}

Let \(X = S^2 / \left\{{p_1 = \cdots = p_k }\right\}\) be the
topological space obtained from the 2-sphere by identifying \(k\)
distinct points on it (\(k \geq 2\)).

Find:

\begin{enumerate}
\def\labelenumi{\alph{enumi}.}
\item
  The fundamental group of \(X\).
\item
  The Euler characteristic of \(X\).
\item
  The homology groups of \(X\).
\end{enumerate}

\hypertarget{fall-16-4}{%
\subsection{7 (Fall '16)}\label{fall-16-4}}

Let \(X\) be the topological space obtained as the quotient of the
sphere
\(S^2 = \left\{{\mathbf{x} \in {\mathbb{R}}^3 {~\mathrel{\Big\vert}~}{\left\lVert {\mathbf{x}} \right\rVert} = 1}\right\}\)
under the equivalence relation \(\mathbf{x} \sim -\mathbf{x}\) for
\(\mathbf{x}\) in the equatorial circle, i.e.~for
\(\mathbf{x} = (x_1, x_2, 0)\).

Calculate \(H_* (X; {\mathbb{Z}})\) from a CW complex description of
\(X\).

\hypertarget{fall-17-3}{%
\subsection{8 (Fall '17)}\label{fall-17-3}}

Compute, by any means available, the fundamental group and all the
homology groups of the space obtained by gluing one copy \(A\) of
\(S^2\) to another copy \(B\) of \(S^2\) via a two-sheeted covering
space map from the equator of \(A\) onto the equator of \(B\).

\hypertarget{spring-14-1}{%
\subsection{9 (Spring '14)}\label{spring-14-1}}

Use cellular homology to calculate the homology groups of
\(S^n \times S^m\).

\hypertarget{fall-09fall-12}{%
\subsection{10 (Fall '09/Fall '12)}\label{fall-09fall-12}}

Denote the points of \(S^1 \times I\) by \((z, t)\) where \(z\) is a
unit complex number and \(0 \leq t \leq 1\). Let \(X\) denote the
quotient of \(S^1 \times I\) given by identifying \((z, 1)\) and
\((z_2 , 0)\) for all \(z \in S^1\).

Give a cell structure, with attaching maps, for \(X\), and use it to
compute \(\pi_1 (X, \ast)\) and \(H_1 (X)\).

\hypertarget{spring-15-5}{%
\subsection{11 (Spring '15)}\label{spring-15-5}}

Let \(X = S_1 \cup S_2 \subset {\mathbb{R}}^3\) be the union of two
spheres of radius 2, one about \((1, 0, 0)\) and the other about
\((-1, 0, 0)\), i.e.~
\begin{align*}  
S_1 &= \left\{{(x, y,z) \mathrel{\Big|}(x-1)^2 + y^2 +z^2 = 4}\right\} \\
S_2 &= \left\{{(x, y, z) \mathrel{\Big|}(x + 1)^2 + y^2 + z^2 = 4}\right\}
.\end{align*}

\begin{enumerate}
\def\labelenumi{\alph{enumi}.}
\item
  Give a description of \(X\) as a CW complex.
\item
  Write out the cellular chain complex of \(X\).
\item
  Calculate \(H_* (X; Z)\).
\end{enumerate}

\hypertarget{spring-06-5}{%
\subsection{12 (Spring '06)}\label{spring-06-5}}

Let \(M\) and \(N\) be finite CW complexes.

\begin{enumerate}
\def\labelenumi{\alph{enumi}.}
\item
  Describe a cellular structure of \(M \times N\) in terms of the
  cellular structures of \(M\) and \(N\).
\item
  Show that the Euler characteristic of \(M \times N\) is the product of
  the Euler characteristics of \(M\) and \(N\).
\end{enumerate}

\hypertarget{spring-07-1}{%
\subsection{13 (Spring '07)}\label{spring-07-1}}

Suppose the space \(X\) is obtained by attaching a 2-cell to the torus
\(S^1 \times S^1\).

In other words, \(X\) is the quotient space of the disjoint union of the
closed disc \({\mathbb{D}}^2\) and the torus \(S^1 \times S^1\) by the
identification \(x \sim f(x)\) where \(S^1\) is the boundary of the unit
disc and \(f : S^1 \to S^1 \times S^1\) is a continuous map.

What are the possible homology groups of \(X\)? Justify your answer.

\hypertarget{spring-15-6}{%
\subsection{14 (Spring '15)}\label{spring-15-6}}

Let \(X\) be the topological space constructed by attaching a closed
2-disk \({\mathbb{D}}^2\) to the circle \(S^1\) by a continuous map
\(\partial{\mathbb{D}}^2 \to S^1\) of degree \(d > 0\) on the boundary
circle.

\begin{enumerate}
\def\labelenumi{\alph{enumi}.}
\item
  Show that every continuous map \(X \to X\) has a fixed point.
\item
  Explain how to obtain all the connected covering spaces of \(X\).
\end{enumerate}

\hypertarget{spring-11-2}{%
\subsection{15 (Spring '11)}\label{spring-11-2}}

Let \(X\) be a topological space obtained by attaching a 2-cell to
\({\mathbb{RP}}^2\) via some map \(f: S^1 \to {\mathbb{RP}}^2\) .

What are the possibilities for the homology \(H_* (X; Z)\)?

\hypertarget{spring-12-1}{%
\subsection{16 (Spring '12)}\label{spring-12-1}}

For any integer \(n \geq 2\) let \(X_n\) denote the space formed by
attaching a 2-cell to the circle \(S^1\) via the attaching map
\begin{align*}  
a_n: S^1 &\to S^1 \\
e^{i\theta} &\mapsto e^{in\theta}
.\end{align*}

\hypertarget{a-1}{%
\subsubsection{a}\label{a-1}}

Compute the fundamental group and the homology of \(X_n\).

\hypertarget{b-1}{%
\subsubsection{b}\label{b-1}}

Exactly one of the \(X_n\) (for \(n \geq 2\)) is homeomorphic to a
surface. Identify, with proof, both this value of \(n\) and the surface
that \(X_n\) is homeomorphic to (including a description of the
homeomorphism).

\hypertarget{spring-09-3}{%
\subsection{17 (Spring '09)}\label{spring-09-3}}

Let \(X\) be a CW complex and let \(\pi : Y \to X\) be a covering space.

\hypertarget{a-2}{%
\subsubsection{a}\label{a-2}}

Show that \(Y\) is compact iff \(X\) is compact and \(\pi\) has finite
degree.

\hypertarget{b-2}{%
\subsubsection{b}\label{b-2}}

Assume that \(\pi\) has finite degree \(d\). Show show that
\(\chi (Y ) = d \chi (X)\).

\hypertarget{c}{%
\subsubsection{c}\label{c}}

Let \(\pi :{\mathbb{RP}}^N \to X\) be a covering map. Show that if \(N\)
is even, \(\pi\) is a homeomorphism.

\hypertarget{spring-18-2}{%
\subsection{18 (Spring '18)}\label{spring-18-2}}

For topological spaces \(X, Y\) the \textbf{mapping cone} \(C(f )\) of a
map \(f : X \to Y\) is defined to be the quotient space
\begin{align*}  
(X \times [0, 1]){\textstyle\coprod}Y / \sim &{\quad \operatorname{where} \quad}  \\ 
(x, 0) &\sim (x', 0) {\quad \operatorname{for all} \quad} x, x' \in X \text{ and } \\ 
(x, 1) &\sim f (x) {\quad \operatorname{for all } \quad} x \in X
.\end{align*}

Let \(\phi_k : S^n \to S^n\) be a degree \(k\) map for some integer
\(k\).

Find \(H_i(C(\phi_k ))\) for all \(i\).

\hypertarget{spring-2019-7}{%
\subsection{Spring 2019 \#7}\label{spring-2019-7}}

For \(f:X\to Y\), the \emph{mapping cone} of \(f\) is defined as
\begin{align*}  
C_f \coloneqq\qty{X\times I} {\textstyle\coprod}Y/\sim \\
(x, 0) \sim (x', 0) \quad \text{for all }x, x'\in X\\
(x, 1) \sim f(x)
.\end{align*}

Let \(\phi_k: S^1\to S^1\) be a \(k{\hbox{-}}\)fold covering and find
\(\pi_1\qty{C_f}\).

\todo[inline]{Revisit, old. Maybe redo.}

\emph{Solution omitted.}

\hypertarget{fall-18-3}{%
\subsection{19 (Fall '18)}\label{fall-18-3}}

Prove that a finite CW complex must be Hausdorff.

\hypertarget{homology-and-degree-theory}{%
\section{Homology and Degree Theory}\label{homology-and-degree-theory}}

\hypertarget{spring-09-4}{%
\subsection{1 (Spring '09)}\label{spring-09-4}}

Compute the homology of the one-point union of \(S^1 \times S^1\) and
\(S^1\).

\hypertarget{fall-06-1}{%
\subsection{2 (Fall '06)}\label{fall-06-1}}

\begin{enumerate}
\def\labelenumi{\alph{enumi}.}
\item
  State the \textbf{Mayer-Vietoris theorem}.
\item
  Use it to compute the homology of the space \(X\) obtained by gluing
  two solid tori along their boundary as follows. Let \({\mathbb{D}}^2\)
  be the unit disk and let \(S^1\) be the unit circle in the complex
  plane \({\mathbb{C}}\). Let \(A = S^1 \times {\mathbb{D}}^2\) and
  \(B = {\mathbb{D}}^2 \times S^1\).
\end{enumerate}

Then \(X\) is the quotient space of the disjoint union
\(A {\textstyle\coprod}B\) obtained by identifying \((z, w) \in A\) with
\((zw^3 , w) \in B\) for all \((z, w) \in S^1 \times S^1\).

\hypertarget{fall-12-2}{%
\subsection{3 (Fall '12)}\label{fall-12-2}}

Let \(A\) and \(B\) be circles bounding disjoint disks in the plane
\(z = 0\) in \({\mathbb{R}}^3\). Let \(X\) be the subset of the upper
half-space of \({\mathbb{R}}^3\) that is the union of the plane
\(z = 0\) and a (topological) cylinder that intersects the plane in
\(\partial C = A \cup B\).

Compute \(H_* (X)\) using the Mayer--Vietoris sequence.

\hypertarget{fall-14-5}{%
\subsection{4 (Fall '14)}\label{fall-14-5}}

Compute the integral homology groups of the space \(X = Y \cup Z\) which
is the union of the sphere
\begin{align*}
Y = \left\{{x^2 + y^2 + z^2 = 1}\right\}
\end{align*}
and the ellipsoid
\begin{align*}
Z =  \left\{{x^2 + y^2 + {z^2 \over 4} = 1}\right\}
.\end{align*}

\hypertarget{spring-08-3}{%
\subsection{5 (Spring '08)}\label{spring-08-3}}

Let \(X\) consist of two copies of the solid torus
\({\mathbb{D}}^2 \times S^1\), glued together by the identity map along
the boundary torus \(S^1 \times S^1\). Compute the homology groups of
\(X\).

\hypertarget{spring-17-2}{%
\subsection{6 (Spring '17)}\label{spring-17-2}}

Use the circle along which the connected sum is performed and the
Mayer-Vietoris long exact sequence to compute the homology of
\({\mathbb{RP}}^2 \# {\mathbb{RP}}^2\).

\hypertarget{fall-15-1}{%
\subsection{7 (Fall '15)}\label{fall-15-1}}

Express a Klein bottle as the union of two annuli.

Use the Mayer Vietoris sequence and this decomposition to compute its
homology.

\hypertarget{spring-09-5}{%
\subsection{8 (Spring '09)}\label{spring-09-5}}

Let \(X\) be the topological space obtained by identifying three
distinct points on \(S^2\). Calculate \(H_* (X; Z)\).

\hypertarget{fall-05-2}{%
\subsection{9 (Fall '05)}\label{fall-05-2}}

Compute \(H_0\) and \(H_1\) of the complete graph \(K_5\) formed by
taking five points and joining each pair with an edge.

\hypertarget{fall-18-4}{%
\subsection{10 (Fall '18)}\label{fall-18-4}}

Compute the homology of the subset \(X \subset {\mathbb{R}}^3\) formed
as the union of the unit sphere, the \(z{\hbox{-}}\)axis, and the
\(xy{\hbox{-}}\)plane.

\hypertarget{spring-05fall-13}{%
\subsection{11 (Spring '05/Fall '13)}\label{spring-05fall-13}}

Let \(X\) be the topological space formed by filling in two circles
\(S^1 \times \left\{{p_1 }\right\}\) and
\(S^1 \times \left\{{p_2 }\right\}\) in the torus \(S^1 \times S^1\)
with disks.

Calculate the fundamental group and the homology groups of \(X\).

\hypertarget{spring-19-3}{%
\subsection{12 (Spring '19)}\label{spring-19-3}}

\begin{enumerate}
\def\labelenumi{\alph{enumi}.}
\tightlist
\item
  Consider the quotient space
  \begin{align*}
  T^2 = {\mathbb{R}}^2 / \sim {\quad \operatorname{where} \quad} (x, y) \sim (x + m, y + n) \text{ for } m, n \in {\mathbb{Z}}
  ,\end{align*}
  and let \(A\) be any \(2 \times 2\) matrix whose entries are integers
  such that \(\operatorname{det}A = 1\).
\end{enumerate}

Prove that the action of \(A\) on \({\mathbb{R}}^2\) descends via the
quotient \({\mathbb{R}}^2 \to T^2\) to induce a homeomorphism
\(T^2 \to T^2\).

\begin{enumerate}
\def\labelenumi{\alph{enumi}.}
\setcounter{enumi}{1}
\tightlist
\item
  Using this homeomorphism of \(T^2\), we define a new quotient space
  \begin{align*}
  T_A^3 \coloneqq{T^2\times{\mathbb{R}}\over \sim} {\quad \operatorname{where} \quad} ((x, y), t) \sim (A(x, y), t + 1)
  \end{align*}
\end{enumerate}

Compute \(H_1 (T_A^3 )\) if
\(A=\left(\begin{array}{ll} 1 & 1 \\ 0 & 1 \end{array}\right).\)

\hypertarget{spring-12-2}{%
\subsection{13 (Spring '12)}\label{spring-12-2}}

Give a self-contained proof that the zeroth homology \(H_0 (X)\) is
isomorphic to \({\mathbb{Z}}\) for every path-connected space \(X\).

\hypertarget{fall-18-5}{%
\subsection{14 (Fall '18)}\label{fall-18-5}}

It is a fact that if \(X\) is a single point then
\(H_1 (X) = \left\{{0}\right\}\).

One of the following is the correct justification of this fact in terms
of the singular chain complex.

Which one is correct and why is it correct?

\begin{enumerate}
\def\labelenumi{\alph{enumi}.}
\item
  \(C_1 (X) = \left\{{0}\right\}\).
\item
  \(C_1 (X) \neq \left\{{0}\right\}\) but \(\ker \partial_1 = 0\) with
  \(\partial_1 : C_1 (X) \to C_0 (X)\).
\item
  \(\ker \partial_1 \neq 0\) but
  \(\ker \partial_1 = \operatorname{im}\partial_2\) with
  \(\partial_2 : C_2 (X) \to C_1 (X)\).
\end{enumerate}

\hypertarget{fall-10-1}{%
\subsection{15 (Fall '10)}\label{fall-10-1}}

Compute the homology groups of \(S^2 \times S^2\).

\hypertarget{fall-16-5}{%
\subsection{16 (Fall '16)}\label{fall-16-5}}

Let \(\Sigma\) be a closed orientable surface of genus \(g\). Compute
\(H_i(S^1 \times \Sigma; Z)\) for \(i = 0, 1, 2, 3\).

\hypertarget{spring-07-2}{%
\subsection{17 (Spring '07)}\label{spring-07-2}}

Prove that if \(A\) is a retract of the topological space \(X\), then
for all nonnegative integers \(n\) there is a group \(G_n\) such that
\(H_{n} (X) \cong H_{n} (A) \oplus G_n\).

\begin{quote}
Here \(H_{n}\) denotes the \(n\)th singular homology group with integer
coefficients.
\end{quote}

\hypertarget{spring-13-3}{%
\subsection{18 (Spring '13)}\label{spring-13-3}}

Does there exist a map of degree 2013 from \(S^2 \to S^2\).

\hypertarget{fall-18-6}{%
\subsection{19 (Fall '18)}\label{fall-18-6}}

For each \(n \in {\mathbb{Z}}\) give an example of a map
\(f_n : S^2 \to S^2\).

For which \(n\) must any such map have a fixed point?

\hypertarget{spring-09-6}{%
\subsection{20 (Spring '09)}\label{spring-09-6}}

\begin{enumerate}
\def\labelenumi{\alph{enumi}.}
\tightlist
\item
  What is the degree of the antipodal map on the \(n\)-sphere?
\end{enumerate}

(No justification required)

\begin{enumerate}
\def\labelenumi{\alph{enumi}.}
\setcounter{enumi}{1}
\item
  Define a CW complex homeomorphic to the real projective
  \(n{\hbox{-}}\)space \({\mathbb{RP}}^n\).
\item
  Let \(\pi : {\mathbb{RP}}^n \to X\) be a covering map. Show that if
  \(n\) is even, \(\pi\) is a homeomorphism.
\end{enumerate}

\hypertarget{fall-17-4}{%
\subsection{21 (Fall '17)}\label{fall-17-4}}

Let \(A \subset X\). Prove that the relative homology group
\(H_0 (X, A)\) is trivial if and only if \(A\) intersects every path
component of \(X\).

\hypertarget{fall-18-7}{%
\subsection{22 (Fall '18)}\label{fall-18-7}}

Let \({\mathbb{D}}\) be a closed disk embedded in the torus
\(T = S^1 \times S^1\) and let \(X\) be the result of removing the
interior of \({\mathbb{D}}\) from \(T\) . Let \(B\) be the boundary of
\(X\), i.e.~the circle boundary of the original closed disk
\({\mathbb{D}}\).

Compute \(H_i (T, B)\) for all \(i\).

\hypertarget{fall-11-2}{%
\subsection{23 (Fall '11)}\label{fall-11-2}}

For any \(n \geq 1\) let
\(S^n = \left\{{(x_0 , \cdots , x_n )\mathrel{\Big|}\sum x_i^2 =1}\right\}\)
denote the \(n\) dimensional unit sphere and let
\begin{align*}E = \left\{{(x_0 , . . . , x_n )\mathrel{\Big|}x_n = 0}\right\}\end{align*}
denote the ``equator''.

Find, for all \(k\), the relative homology \(H_k (S^n , E)\).

\hypertarget{spring-12spring-15}{%
\subsection{24 (Spring '12/Spring '15)}\label{spring-12spring-15}}

Suppose that \(U\) and \(V\) are open subsets of a space \(X\), with
\(X = U \cup V\). Find, with proof, a general formula relating the Euler
characteristics of \(X, U, V\), and \(U \cap V\).

\begin{quote}
You may assume that the homologies of \(U, V, U \cap V, X\) are
finite-dimensional so that their Euler characteristics are well defined.
\end{quote}

\hypertarget{spring-2021-6}{%
\subsection{Spring 2021 \#6}\label{spring-2021-6}}

\begin{problem}[Spring 2021, 6]

For each of the following spaces, compute the fundamental group and the
homology groups.

\begin{enumerate}
\def\labelenumi{\alph{enumi}.}
\item
  The graph \(\Theta\) consisting of two edges and three vertices
  connecting them.
\item
  The 2-dimensional cell complex \(\Theta_2\) consisting of a closed
  circle and three 2-dimensional disks each having boundary running once
  around that circle.
\end{enumerate}

\end{problem}

\hypertarget{spring-2021-7}{%
\subsection{Spring 2021 \#7}\label{spring-2021-7}}

\begin{problem}[Spring 2021, 7]

Prove directly from the definition that the 0th singular homology of a
nonempty path-connected space is isomorphic to \({\mathbb{Z}}\).

\end{problem}

\hypertarget{spring-2021-9}{%
\subsection{Spring 2021 \#9}\label{spring-2021-9}}

\begin{problem}[Spring 2021, 9]

Prove that for every continuous map \(f: S^{2n} \to S^{2n}\) there is a
point \(x\in S^{2n}\) such that either \(f(x) = x\) or \(f(x) = -x\).

\begin{quote}
You may use standard facts about degrees of maps of spheres, including
that the antipodal map on \(S^{2n}\) has degree \(d=-1\).
\end{quote}

\end{problem}

\hypertarget{surfaces}{%
\section{Surfaces}\label{surfaces}}

\hypertarget{fall-05-3}{%
\subsection{1 (Fall '05)}\label{fall-05-3}}

State the classification theorem for surfaces (compact, without
boundary, but not necessarily orientable). For each surface in the
classification, indicate the structure of the first homology group and
the value of the Euler characteristic.

Also, explain briefly how the 2-holed torus and the connected sum
\({\mathbb{RP}}^2 \# {\mathbb{RP}}^2\) fit into the classification.

\hypertarget{spring-16-2}{%
\subsection{2 (Spring '16)}\label{spring-16-2}}

Give a list without repetitions of all compact surfaces (orientable or
non-orientable and with or without boundary) that have Euler
characteristic negative one.

Explain why there are no repetitions on your list.

\hypertarget{spring-07-3}{%
\subsection{3 (Spring '07)}\label{spring-07-3}}

Describe the topological classification of all compact connected
surfaces \(M\) without boundary having Euler characteristic
\(\chi(M )\geq -2\).

No proof is required.

\hypertarget{spring-09-7}{%
\subsection{4 (Spring '09)}\label{spring-09-7}}

How many surfaces are there, up to homeomorphism, which are:

\begin{itemize}
\tightlist
\item
  Connected,
\item
  Compact,
\item
  Possibly with boundary,
\item
  Possibly nonorientable, and
\item
  With Euler characteristic -3?
\end{itemize}

Describe one representative from each class.

\hypertarget{fall-13}{%
\subsection{5 (Fall '13)}\label{fall-13}}

Prove that the Euler characteristic of a compact surface with boundary
which has \(k\) boundary components is less than or equal to \(2 - k\).

\hypertarget{spring-13-4}{%
\subsection{6 (Spring '13)}\label{spring-13-4}}

What surface is represented by the \(6{\hbox{-}}\)gon with edges
identified according to the symbol \(xyzxy^{-1}z^{-1}\) ?

\hypertarget{spring-15-7}{%
\subsection{7 (Spring '15)}\label{spring-15-7}}

Let \(X\) be the topological space obtained as the quotient space of a
regular \(2n{\hbox{-}}\)gon (\(n \geq 2\)) in \({\mathbb{R}}^2\) by
identifying opposite edges via translations in the plane.

First show that X is a compact, orientable surface without boundary, and
then identify its genus as a function of \(n\).

\hypertarget{fall-10-2}{%
\subsection{8 (Fall '10)}\label{fall-10-2}}

\hypertarget{a-3}{%
\subsubsection{a}\label{a-3}}

Show that any compact connected surface with nonempty boundary is
homotopy equivalent to a wedge of circles

\begin{quote}
Hint: you may assume that any compact connected surface without boundary
is given by identifying edges of a polygon in pairs.
\end{quote}

\hypertarget{b-3}{%
\subsubsection{b}\label{b-3}}

For each surface appearing in the classification of compact surfaces
with nonempty boundary, say how many circles are needed in the wedge
from part (a).

\begin{quote}
Hint: you should be able to do this even if you have not done part (a).
\end{quote}

\hypertarget{fall-04-4}{%
\subsection{9 (Fall '04)}\label{fall-04-4}}

Let \(M_g^2\) be the compact oriented surface of genus \(g\).

Show that there exists a continuous map \(f : M_g^2 \to S^2\) which is
not homotopic to a constant map.

\hypertarget{spring-11-3}{%
\subsection{10 (Spring '11)}\label{spring-11-3}}

Show that \({\mathbb{RP}}^2 \lor S^1\) is \emph{not} homotopy equivalent
to a compact surface (possibly with boundary).

\hypertarget{fall-14-6}{%
\subsection{11 (Fall '14)}\label{fall-14-6}}

Identify (with proof, but of course you can appeal to the classification
of surfaces) all of the compact surfaces without boundary that have a
cell decomposition having exactly one 0-cell and exactly two 1-cells
(with no restriction on the number of cells of dimension larger than 1).

\hypertarget{fall-11-3}{%
\subsection{12 (Fall '11)}\label{fall-11-3}}

For any natural number \(g\) let \(\Sigma_g\) denote the (compact,
orientable) surface of genus \(g\).

Determine, with proof, all valued of \(g\) with the property that there
exists a covering space \(\pi : \Sigma_5 \to \Sigma_g\) .

\begin{quote}
Hint: How does the Euler characteristic behave for covering spaces?
\end{quote}

\hypertarget{spring-14-2}{%
\subsection{13 (Spring '14)}\label{spring-14-2}}

Find \emph{all} surfaces, orientable and non-orientable, which can be
covered by a closed surface (i.e.~compact with empty boundary) of genus
2. Prove that your answer is correct.

\hypertarget{spring-18-3}{%
\subsection{14 (Spring '18)}\label{spring-18-3}}

\hypertarget{a-4}{%
\subsubsection{a}\label{a-4}}

Write down (without proof) a presentation for \(\pi_1 (\Sigma_2 , p)\)
where \(\Sigma_2\) is a closed, connected, orientable genus 2 surface
and \(p\) is any point on \(\Sigma_2\) .

\hypertarget{b-4}{%
\subsubsection{b}\label{b-4}}

Show that \(\pi_1 (\Sigma_2 , p)\) is not abelian by showing that it
surjects onto a free group of rank 2.

\hypertarget{c-1}{%
\subsubsection{c}\label{c-1}}

Show that there is no covering space map from \(\Sigma_2\) to
\(S^1 \times S^1\) . You may use the fact that
\(\pi_1 (S^1 \times S^1 ) \cong {\mathbb{Z}}^2\) together with the
result in part (b) above.

\hypertarget{fall-16-6}{%
\subsection{15 (Fall '16)}\label{fall-16-6}}

Give an example, with explanation, of a closed curve in a surfaces which
is not nullhomotopic but is nullhomologous.

\hypertarget{fall-17-5}{%
\subsection{16 (Fall '17)}\label{fall-17-5}}

Let \(M\) be a compact orientable surface of genus \(2\) without
boundary.

Give an example of a pair of loops
\begin{align*}\gamma_0 , \gamma_1 : S^1 \to M\end{align*}
with \(\gamma_0 (1) = \gamma_1 (1)\) such that there is a continuous map
\(\Gamma: [0, 1] \times S^1 \to M\) such that
\begin{align*}
\Gamma(0, t) = \gamma_0 (t), \quad \Gamma(1, t) = \gamma_1 (t) {\quad \operatorname{for all} \quad} t \in S^1
,\end{align*}
but such that there is no such map \(\Gamma\) with the additional
property that \(\Gamma_s (1) = \gamma_0 (1)\) for all \(s \in [0, 1]\).

(You are not required to prove that your example satisfies the stated
property.)

\hypertarget{fall-18-8}{%
\subsection{17 (Fall '18)}\label{fall-18-8}}

Let \(C\) be cylinder. Let \(I\) and \(J\) be disjoint closed intervals
contained in \(\partial C\).

What is the Euler characteristic of the surface \(S\) obtained by
identifying \(I\) and \(J\)?

Can all surface with nonempty boundary and with this Euler
characteristic be obtained from this construction?

\hypertarget{spring-19-4}{%
\subsection{18 (Spring '19)}\label{spring-19-4}}

Let \(\Sigma\) be a compact connected surface and let
\(p_1, \cdots , p_k \in \Sigma\).

Prove that \(H_2 \qty{\Sigma \setminus \cup_{i=1}^k {p_i} } = 0\).

\hypertarget{spring-2021}{%
\subsection{Spring 2021}\label{spring-2021}}

\begin{problem}[Spring 2021, 8]

Let \(\Sigma_{g, n}\) denote the compact oriented surface of genus \(g\)
with \(n\) boundary components.

\begin{enumerate}
\def\labelenumi{\alph{enumi}.}
\item
  Show that \(\Sigma_{0, 3}\) and \(\Sigma_{1, 1}\) are both homotopy
  equivalent to \(S_1 \vee S^1\).
\item
  Give a complete classification of pairs \((g, n)\) and \((g', n')\)
  for which \(\Sigma_{g, n}\) is homotopy equivalent to
  \(\Sigma_{g', n'}\).
\end{enumerate}

\end{problem}

\hypertarget{fixed-points}{%
\section{Fixed Points}\label{fixed-points}}

\hypertarget{fall-14-7}{%
\subsection{1 (Fall '14)}\label{fall-14-7}}

Prove that, for every continuous map \(f : B^2 \to B^2\), there is a
point \(x\) such that \(f (x) = x\).

\begin{quote}
This is the \(n = 2\) case of the Brouwer fixed point theorem; your
proof shouldn't appeal to either of the Brouwer or the Lefschetz fixed
point theorems.
\end{quote}

\hypertarget{spring-18-4}{%
\subsection{2 (Spring '18)}\label{spring-18-4}}

Prove or disprove:

Every continuous map from \(S^2\) to \(S^2\) has a fixed point.

\hypertarget{spring-11-4}{%
\subsection{3 (Spring '11)}\label{spring-11-4}}

\begin{enumerate}
\def\labelenumi{\alph{enumi}.}
\item
  State the \textbf{Lefschetz Fixed Point Theorem} for a finite
  simplicial complex \(X\).
\item
  Use degree theory to prove this theorem in case \(X = S^n\).
\end{enumerate}

\hypertarget{spring-12-3}{%
\subsection{4 (Spring '12)}\label{spring-12-3}}

\hypertarget{a-5}{%
\subsubsection{a}\label{a-5}}

Prove that for every continuous map \(f : S^2 \to S^2\) there is some
\(x\) such that either \(f (x) = x\) or \(f (x) = -x\).

\begin{quote}
Hint: Where \(A : S^2 \to S^2\) is the antipodal map, you are being
asked to prove that either \(f\) or \(A \circ f\) has a fixed point.
\end{quote}

\hypertarget{b-5}{%
\subsubsection{b}\label{b-5}}

Exhibit a continuous map \(f : S^3 \to S^3\) such that for every
\(x \in S^3\), \(f (x)\) is equal to neither \(x\) nor \(-x\).

\begin{quote}
Hint: It might help to first think about how you could do this for a map
from \(S^1\) to \(S^1\).
\end{quote}

\hypertarget{spring-14-3}{%
\subsection{5 (Spring '14)}\label{spring-14-3}}

Show that a map \(S^n \to S^n\) has a fixed point unless its degree is
equal to the degree of the antipodal map \(a : x \to -x\).

\hypertarget{spring-08-4}{%
\subsection{6 (Spring '08)}\label{spring-08-4}}

Give an example of a homotopy class of maps of \(S^1 \lor S^1\) each
member of which must have a fixed point, and also an example of a map of
\(S^1 \lor S^1\) which doesn't have a fixed point.

\hypertarget{spring-17-3}{%
\subsection{7 (Spring '17)}\label{spring-17-3}}

Prove or disprove:

Every map from \({\mathbb{RP}}^2 \lor {\mathbb{RP}}^2\) to itself has a
fixed point.

\hypertarget{fall-09-1}{%
\subsection{8 (Fall '09)}\label{fall-09-1}}

Find all homotopy classes of maps from \(S^1 \times {\mathbb{D}}^2\) to
itself such that every element of the homotopy class has a fixed point.

\hypertarget{spring-10-2}{%
\subsection{9 (Spring '10)}\label{spring-10-2}}

Let \(X\) and \(Y\) be finite connected simplicial complexes and let
\(f : X \to Y\) and \(g : Y \to X\) be basepoint-preserving maps.

Show that no matter how you homotope
\(f \lor g : X \lor Y \to X \lor Y\), there will always be a fixed
point.

\hypertarget{fall-12-3}{%
\subsection{10 (Fall '12)}\label{fall-12-3}}

Let \(f = \operatorname{id}_{{\mathbb{RP}}^2} \lor \ast\) and
\(g = \ast \lor id_{S^1}\) be two maps of \({\mathbb{RP}}^2 \lor S^1\)
to itself where \(\ast\) denotes the constant map of a space to its
basepoint.

Show that one map is homotopic to a map with no fixed points, while the
other is not.

\hypertarget{spring-09-8}{%
\subsection{11 (Spring '09)}\label{spring-09-8}}

View the torus \(T\) as the quotient space
\({\mathbb{R}}^2 /{\mathbb{Z}}^2\).

Let \(A\) be a \(2 \times 2\) matrix with \({\mathbb{Z}}\) coefficients.

\hypertarget{a-6}{%
\subsubsection{a}\label{a-6}}

Show that the linear map \(A : {\mathbb{R}}^2 \to {\mathbb{R}}^2\)
descends to a continuous map \({\mathcal{A}}: T \to T\).

\hypertarget{b-6}{%
\subsubsection{b}\label{b-6}}

Show that, with respect to a suitable basis for
\(H_1 (T ; {\mathbb{Z}})\), the matrix \(A\) represents the map induced
on \(H_1\) by \({\mathcal{A}}\).

\hypertarget{c-2}{%
\subsubsection{c}\label{c-2}}

Find a necessary and sufficient condition on \(A\) for \({\mathcal{A}}\)
to be homotopic to the identity.

\hypertarget{d}{%
\subsubsection{d}\label{d}}

Find a necessary and sufficient condition on \(A\) for \({\mathcal{A}}\)
to be homotopic to a map with no fixed points.

\hypertarget{spring-19-5}{%
\subsection{12 (Spring '19)}\label{spring-19-5}}

\hypertarget{a-7}{%
\subsubsection{a}\label{a-7}}

Use the Lefschetz fixed point theorem to show that any degree-one map
\(f : S^2 \to S^2\) has at least one fixed point.

\hypertarget{b-7}{%
\subsubsection{b}\label{b-7}}

Give an example of a map \(f : {\mathbb{R}}^2 \to {\mathbb{R}}^2\)
having no fixed points.

\hypertarget{c-3}{%
\subsubsection{c}\label{c-3}}

Give an example of a degree-one map \(f : S^2 \to S^2\) having exactly
one fixed point.

\hypertarget{fall-10-3}{%
\subsection{13 (Fall '10)}\label{fall-10-3}}

For which compact connected surfaces \(\Sigma\) (with or without
boundary) does there exist a continuous map \(f : \Sigma \to \Sigma\)
that is homotopic to the identity and has no fixed point?

Explain your answer fully.

\hypertarget{spring-16-3}{%
\subsection{14 (Spring '16)}\label{spring-16-3}}

Use the Brouwer fixed point theorem to show that an \(n \times n\)
matrix with nonnegative entries has a real eigenvalue.

\hypertarget{miscellaneous-algebraic-topology}{%
\section{Miscellaneous Algebraic
Topology}\label{miscellaneous-algebraic-topology}}

\hypertarget{fall-14-8}{%
\subsection{1 (Fall '14)}\label{fall-14-8}}

Prove that \({\mathbb{R}}^2\) is not homeomorphic to \({\mathbb{R}}^n\)
for \(n > 2\).

\hypertarget{spring-12-4}{%
\subsection{2 (Spring '12)}\label{spring-12-4}}

Prove that any finite tree is contractible, where a \textbf{tree} is a
connected graph that contains no closed edge paths.

\hypertarget{spring-13-5}{%
\subsection{3 (Spring '13)}\label{spring-13-5}}

Show that any continuous map \(f : {\mathbb{RP}}^2 \to S^1 \times S^1\)
is necessarily null-homotopic.

\emph{Solution omitted.}

\hypertarget{fall-11-4}{%
\subsection{4 (Fall '11)}\label{fall-11-4}}

Prove that, for \(n \geq 2\), every continuous map
\(f: {\mathbb{RP}}^n \to S^1\) is null-homotopic.

\emph{Solution omitted.}

\hypertarget{spring-06-6}{%
\subsection{5 (Spring '06)}\label{spring-06-6}}

Let \(S^2 \to {\mathbb{RP}}^2\) be the universal covering map.

Is this map null-homotopic? Give a proof of your answer.

\hypertarget{spring-17-4}{%
\subsection{6 (Spring '17)}\label{spring-17-4}}

Suppose that a map \(f : S^3 \times S^3 \to {\mathbb{RP}}^3\) is not
surjective.

Prove that \(f\) is homotopic to a constant function.

\todo[inline]{Lost, redo.}

\hypertarget{fall-06-2}{%
\subsection{7 (Fall '06)}\label{fall-06-2}}

Prove that there does not exist a continuous map \(f : S^2 \to S^2\)
from the unit sphere in \({\mathbb{R}}^3\) to itself such that
\(f (\mathbf{x}) \perp \mathbf{x}\) (as vectors in \({\mathbb{R}}^3\)
for all \(\mathbf{x} \in S^2\)).

\hypertarget{spring-08-5}{%
\subsection{8 (Spring '08)}\label{spring-08-5}}

Let \(f\) be the map of \(S^1 \times [0, 1]\) to itself defined by
\begin{align*}
f (e^{i\theta} , s) = (e^{i(\theta+2\pi s)} , s)
,\end{align*}
so that \(f\) restricts to the identity on the two boundary circles of
\(S^1 \times [0, 1]\).

Show that \(f\) is homotopic to the identity by a homotopy \(f_t\) that
is stationary on one of the boundary circles, but not by any homotopy
that is stationary on both boundary circles.

\begin{quote}
Hint: Consider what \(f\) does to the path
\(s \mapsto (e^{i\theta_0} , s)\) for fixed \(e^{i\theta_0} \in S^1\).
\end{quote}

\hypertarget{spring-17-5}{%
\subsection{9 (Spring '17)}\label{spring-17-5}}

Show that \(S^1 \times S^1\) is not the union of two disks (where there
is no assumption that the disks intersect along their boundaries).

\hypertarget{spring-14-4}{%
\subsection{10 (Spring '14)}\label{spring-14-4}}

Suppose that \(X \subset Y\) and \(X\) is a deformation retract of
\(Y\).

Show that if \(X\) is a path connected space, then \(Y\) is path
connected.

\hypertarget{spring-05-2}{%
\subsection{11 (Spring '05)}\label{spring-05-2}}

Do one of the following:

\begin{enumerate}
\def\labelenumi{\alph{enumi}.}
\item
  Give (with justification) a contractible subset
  \(X \subset {\mathbb{R}}^2\) which is not a retract of
  \({\mathbb{R}}^2\) .
\item
  Give (with justification) two topological spaces that have the same
  homology groups but that are not homotopy equivalent.
\end{enumerate}

\hypertarget{spring-16-4}{%
\subsection{12 (Spring '16)}\label{spring-16-4}}

Recall that the \textbf{suspension} of a topological space, denoted
\(SX\), is the quotient space formed from \(X \times [-1, 1]\) by
identifying \((x, 1)\) with \((y, 1)\) for all \(x, y \in X\), and also
identifying \((x, -1)\) with \((y, -1)\) for all \(x, y \in X\).

\begin{enumerate}
\def\labelenumi{\alph{enumi}.}
\item
  Show that \(SX\) is the union of two contractible subspaces.
\item
  Prove that if \(X\) is path-connected then
  \(\pi_1 (SX) = \left\{{0}\right\}\).
\item
  For all \(n \geq 1\), prove that \(H_{n} (X) \cong H_{n+1} (SX)\).
\end{enumerate}

\hypertarget{extra-problems-algebraic-topology}{%
\section{Extra Problems: Algebraic
Topology}\label{extra-problems-algebraic-topology}}

\hypertarget{algebraic-topology}{%
\subsection{Algebraic Topology}\label{algebraic-topology}}

\hypertarget{fundamental-group}{%
\subsubsection{Fundamental Group}\label{fundamental-group}}

\begin{itemize}
\tightlist
\item
  Compute \(\pi_1(X)\) where \(X \coloneqq S^2/\sim\), where
  \(x\sim -x\) only for \(x\) on the equator
  \(S^1 \hookrightarrow S^2\).

  \begin{itemize}
  \tightlist
  \item
    Hint: try cellular homology. Should yield
    \([{\mathbb{Z}}, {\mathbb{Z}}/2{\mathbb{Z}}, {\mathbb{Z}}, 0, \cdots]\).
  \end{itemize}
\item
  Show that if \(X = S^2 {\textstyle\coprod}_{\operatorname{id}} S^2\)
  is a pushout along the equators, then
  \(H_n(X) = [{\mathbb{Z}}, 0, {\mathbb{Z}}^3, 0, \cdots]\).
\end{itemize}

\hypertarget{covering-spaces-1}{%
\subsubsection{Covering Spaces}\label{covering-spaces-1}}

\begin{itemize}
\tightlist
\item
  Describe all connected covering spaces of
  \({\mathbb{RP}}^2 \vee {\mathbb{RP}}^2\).
\end{itemize}

\hypertarget{homology}{%
\subsubsection{Homology}\label{homology}}

\begin{itemize}
\tightlist
\item
  Compute the homology of the Klein bottle using the Mayer-Vietoris
  sequence and a decomposition \(K = M {\textstyle\coprod}_f M\)
\item
  Use the Kunneth formula to compute
  \(H^*(S^2\times S^2; {\mathbb{Z}})\).

  \begin{itemize}
  \tightlist
  \item
    Known to be
    \([{\mathbb{Z}}, 0, {\mathbb{Z}}^2, 0, {\mathbb{Z}}, 0, 0, \cdots]\).
  \end{itemize}
\item
  Compute \(H^*(S^2 \vee S^2 \vee S^4)\)

  \begin{itemize}
  \tightlist
  \item
    Known to be
    \([{\mathbb{Z}}, 0, {\mathbb{Z}}^2, 0, {\mathbb{Z}}, 0, 0, \cdots]\).
  \end{itemize}
\item
  Show that
  \(\chi(\Sigma_g + \Sigma_h) = \chi(\Sigma_g) + \chi(\Sigma_h) - 2\).
\end{itemize}



\newpage
\printbibliography[title=Bibliography]


\end{document}
