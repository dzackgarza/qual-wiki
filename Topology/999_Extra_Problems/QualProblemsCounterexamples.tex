\documentclass[psamsfonts, 11pt, reqno]{amsart}
\setlength{\topmargin}{0pt}
\setlength{\footskip}{10pt}
\setlength{\oddsidemargin}{2.5cm}
\setlength{\evensidemargin}{2.5cm}
\setlength{\textwidth}{329pt}
\addtolength{\textheight}{50pt}
\usepackage{amsfonts,amsmath, amsthm, amssymb, latexsym, amscd, epsfig, pdfsync}
\usepackage[all]{xy}
\usepackage[mathscr]{eucal}
%\usepackage{layout}
%\addtolength{\topmargin}{-90pt}
%\addtolength{\textheight}{140pt}
\addtolength{\textwidth}{120pt}
\addtolength{\hoffset}{-60pt}
\everymath={\displaystyle}

\newcommand{\seriesk}{\sum\limits_{k=1}^{\infty}}
\newcommand{\seriesn}{\sum\limits_{n=1}^{\infty}}
\newcommand{\z}{\mathbb{Z}}
\newcommand{\n}{\mathbb{N}}
\newcommand{\R}{\mathbb{R}}
\newcommand{\q}{\mathbb{Q}}
\newcommand{\C}{\mathbb{C}}
\newcommand{\la}{\langle}
\newcommand{\ra}{\rangle}
\newcommand{\x}{\underline{\underline{x}}}
\renewcommand{\r}{\underline{\underline{r}}}
\renewcommand{\t}{\mathcal{T}}
\renewcommand{\d}{\mathcal{D}}
\newcommand{\D}{\mathcal{D}}
\newcommand{\pic}{\mathbb{P}}
\newcommand{\p}{\mathcal{P}}
\newcommand{\Q}{\mathcal{Q}}
\newcommand{\ds}{\displaystyle}
\renewcommand{\tilde}{\widetilde}
\newcommand{\inv}{^{-1}}
\renewcommand{\oplus}{\bigoplus}
\renewcommand{\Sigma}{\sum}
\renewcommand{\Pi}{\prod}
\renewcommand{\phi}{\varphi}

\newtheorem{definition}{Definition}
\newtheorem{thm}[definition]{Theorem}
\newtheorem{theorem}[definition]{Theorem}
%\theoremstyle{definition}
%\newtheorem{fact}[defn]{Fact}
%\newtheorem{note}[defn]{Note}
\newtheorem{lemma}[definition]{Lemma}
%\newtheorem{cor}[defn]{Corollary}
%\newtheorem{conjecture}[defn]{Conjecture}
%\newtheorem{problem}[defn]{Problem}
%\newtheorem{notation}[defn]{Notation}
%\newtheorem*{question}{Question}
\newtheorem{ex}[definition]{Example}
\newtheorem{example}[definition]{Example}
%\newtheorem*{results}{Known Results}

\title{QualProblems}
%\author(,\\
%}

\begin{document}
\begin{center}
\begin{large}
Topology Qual Workshop Day 4: Counterexamples
\end{large}
\end{center}
\vspace{.25in}

\begin{enumerate}
\item (June '09 \# 2A) For a topological space $X$ and $y \in X$, the \emph{path component} $P_y$ of $X$ containing
$y$ is the largest path-connected subset with $y \in P_y \subseteq X$.
\begin{enumerate}
\item Show that this concept is well defined (that is, show that every point $y$ is contained in a largest path-connected
subset).
\item Give an example of a space and a point $y \in X$ so that $P_y$ is neither an open nor a closed subset of $X$.
\end{enumerate}
\vfill


\item (June '08 \# A1) Let $A$ be a subset of a topological space $X$, and let $B$ be a subset of a topological space $Y$.
Let $X\times Y$ be the product space, and let int$_Z(C)$ denote the interior of the set $C$ in the space $Z$.  Prove
or give a counterexample to int$_{X \times Y}(A \times B) = $int$_X(A) \times$int$_Y(B)$.

\vfill

\item (Jan '04 \# A3) Let $(X, \tau)$ be a Hausdorff space and let $\tau ' = \{ U \subseteq X : X \setminus U \subseteq X
\mbox{ is compact} \} \cup \{ \emptyset \}$.  Show that $\tau '$ is a topology on $X$  and is coarser than $\tau$.
Show that, in general, they need not be equal.

\vfill

\item Let $X$ be a compact space.  If $X = A \cup B$ with both $A$ and $B$ Hausdorff, must $X$ be Hausdorff?

\vfill
\end{enumerate}
\textbf{Tips}: The following are common places to look for counterexamples: 
\begin{itemize}
\item Flea and Comb (Differentiates Connected and Path Connected)
\item Comb (Sans the flea) (Differentiates Path Connected and Locally Path Connected)
\item Topologists Sine Curve (Differentiates Connected and Path Connected)
\item 2 or 3 point sets (Especially useful if you assume your space must be compact)
\item Any space with the Discrete Topology
\item Any space with the Indiscrete Topology
\end{itemize}


\newpage

\begin{center}
\begin{large}
Topology Qual Workshop Day 4: Assorted Problems
\end{large}
\end{center}
\vspace{.25in}

\begin{enumerate}

\item (Jan '12 \# A1) Recall that a continuous function $f: X \rightarrow Y$ between two topological spaces is called
a \emph{closed mapping} if for every closed subset $C$ of $X$, $f(C)$ is a closed subset of $Y$.  Prove that if
$f: X \rightarrow Y$ is continuous, $X$ is compact and $Y$ is Hausdorff, then $f$ is a closed mapping.

\vfill

\item (June '10 \# A1) Let $X$ be a topological space and let $f, g: X \rightarrow \mathbb{R}$ be continuous functions.
\begin{enumerate}
\item Show that the set $L = \{ p \in X: f(p) \leq g(p) \}$ is a closed subset of $X$.
\item Show that the function $h: X \rightarrow \mathbb{R}$ given by $h = \min \{f(p), g(p) \}$ is continuous.
\end{enumerate}
\vfill

\item (June '08 \# A3) 
\begin{enumerate}
\item Prove that a map $f: X \rightarrow Y$ between topological spaces is continuous if and only if $f(\overline{A})
\subseteq \overline{f(A)}$ for all $A \subseteq X$.
\item Prove that if $f$ is continuous and $f(\overline{A})$ closed for some $A \subseteq X$, then $f(\overline{A})
= \overline{f(A)}$ 
\end{enumerate}

\vfill

\item (Jan '08 \# B8) Suppose that $X$ is an arbitrary topological space and $Y$ is a compact space.  Consider
the projection map $\pi: X \times Y \rightarrow X$ defined by $\pi (x, y ) = x$.  Prove that if $X \times Y$ has
the product topology, then $\pi$ is a closed map.

\vfill

\item (June '07 \# A1) A space $(X, \tau)$ is called \emph{extremally disconnected} if the closure of every open
set is open; that is, whenever $U \in \tau$ we have $cl(U) \in \tau$.  Show that in an extremally disconnected space, if
$U, V \in \tau$ and $U \cap V = \emptyset$, then $cl(U) \cap cl(V) = \emptyset$. (Hint: first show that
$U \cap cl(V) = \emptyset$.)

\vfill

\item (Jan '06 \# A2) A topological space $(X, \tau)$ is called \emph{limit-point compact} if every infinite
subset $A$ of $X$ has a limit point. Show that every closed subset of a limit-point compact space is limit-point compact.

\vfill

\item (June '05 \# A4) A topological space $X$ is called \emph{metacompact} if for every open cover $\mathcal{C}$ of $X$,
there is a subcover $\mathcal{C}'$ satisfying the property that for every point $p \in X$, there are only finintely many open
sets in $\mathcal{C}'$ containing $p$.

\begin{enumerate}
\item Show that metacompactness is a homeomorphism invariant.
\item Let $X$ be the integers with the topology $\tau := \{ U \subseteq X | 0 \in U \} \cup \{ \emptyset \}$.  Show
that this space is not metacompact.
\end{enumerate}

\vfill

\item (June '04 \# A2) Let $X$ be the unit sphere in $\mathbb{R}^3$ and define an equivalence relation on $X$ by 
$$(x, y, z) \sim (x', y', z') \Leftrightarrow z = z'$$
Let $Z = X / \sim$ be the quotient space under this equivalence relation, with the quotient topology.  Show that $Z$
is homeomorphic to the interval $[-1,1]$.

\vfill

\end{enumerate}

\end{document}