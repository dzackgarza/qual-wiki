\documentclass[psamsfonts, 11pt, reqno]{amsart}
\setlength{\topmargin}{0pt}
\setlength{\footskip}{10pt}
\setlength{\oddsidemargin}{2.5cm}
\setlength{\evensidemargin}{2.5cm}
\setlength{\textwidth}{329pt}
\addtolength{\textheight}{50pt}
\usepackage{amsfonts,amsmath, amsthm, amssymb, latexsym, amscd, epsfig, pdfsync}
\usepackage[all]{xy}
\usepackage[mathscr]{eucal}
%\usepackage{layout}
%\addtolength{\topmargin}{-90pt}
%\addtolength{\textheight}{140pt}
\addtolength{\textwidth}{120pt}
\addtolength{\hoffset}{-60pt}
\everymath={\displaystyle}

\newcommand{\seriesk}{\sum\limits_{k=1}^{\infty}}
\newcommand{\seriesn}{\sum\limits_{n=1}^{\infty}}
\newcommand{\z}{\mathbb{Z}}
\newcommand{\n}{\mathbb{N}}
\newcommand{\R}{\mathbb{R}}
\newcommand{\q}{\mathbb{Q}}
\newcommand{\C}{\mathbb{C}}
\newcommand{\la}{\langle}
\newcommand{\ra}{\rangle}
\newcommand{\x}{\underline{\underline{x}}}
\renewcommand{\r}{\underline{\underline{r}}}
\renewcommand{\t}{\mathcal{T}}
\renewcommand{\d}{\mathcal{D}}
\newcommand{\D}{\mathcal{D}}
\newcommand{\pic}{\mathbb{P}}
\newcommand{\p}{\mathcal{P}}
\newcommand{\Q}{\mathcal{Q}}
\newcommand{\ds}{\displaystyle}
\renewcommand{\tilde}{\widetilde}
\newcommand{\inv}{^{-1}}
\renewcommand{\oplus}{\bigoplus}
\renewcommand{\Sigma}{\sum}
\renewcommand{\Pi}{\prod}
\renewcommand{\phi}{\varphi}

\newtheorem{definition}{Definition}
\newtheorem{thm}[definition]{Theorem}
\newtheorem{theorem}[definition]{Theorem}
%\theoremstyle{definition}
%\newtheorem{fact}[defn]{Fact}
%\newtheorem{note}[defn]{Note}
\newtheorem{lemma}[definition]{Lemma}
%\newtheorem{cor}[defn]{Corollary}
%\newtheorem{conjecture}[defn]{Conjecture}
%\newtheorem{problem}[defn]{Problem}
%\newtheorem{notation}[defn]{Notation}
%\newtheorem*{question}{Question}
\newtheorem{ex}[definition]{Example}
\newtheorem{example}[definition]{Example}
%\newtheorem*{results}{Known Results}

\title{QualProblems}
%\author(,\\
%}

\begin{document}
\begin{center}
\begin{large}
Topology Qual Workshop Day 7: Fundamental Groups
\end{large}
\end{center}
\vspace{.25in}

\begin{enumerate}

\item (Michigan May '09) Let $X$ be the space obtained from $S^1 \times \mathbb{R}$ by removing the interior
of $k$ disjoint 2-disks.
\begin{enumerate}
\item Compute the fundamental group $\pi_1 (X)$.
\item What would be your answer to part (a) if $S^1 \times \mathbb{R}$ is replaced by $S^2 \times \mathbb{R}$ and
2-disks are replaced by 3-balls?
\item Let $Y$ be the union of two copies of the real projective plane $\mathbb{R}P^2$ having exactly one point $y$
in common.  Compute $\pi_1(Y, y)$.
\end{enumerate}

\vfill

\item State the Seifert-van Kampen Theorem and use it to show that $\pi_1 (S^n) = 0$ for $n >1$.  Why doesn't this argument work for $n=1$?

\vfill

\item (Wisconsin Aug '10) The graph $K$ has six vertices $a_1, a_2, a_3, b_1, b_2, b_3$ and nine edges $a_i b_j$ for $i, j = 1,2,3$.  The
space $X$ obtained from $K$ by attaching a 2-cell along each loop formed by a cycle of four edges in $K$.  Find $\pi_1(X)$.


\vfill

\item (Jan '02) Find the fundamental group of the space $X$ consisting of $\mathbb{R}^3$ with the three coordinate axes
removed.

\vfill

\item (June '05) Find a cell structure for the quotient space obtained by identifying two distinct points $a,b$ in
a 2-torus to a third point $c$ in a 2-sphere, and compute a presentation for the fundamental group of this space.

\vfill

\item (June '08) Let $X$ be the triangle parachute formed from the standard 2-simplex $\Delta ^2$ by identifying
the three vertices with one another.  Compute a presentation for $\pi_1(X)$ and show that $\pi_1(X)$ is isomorphic
to a free group $F_n$ (and identify which $n$!).

\vfill



\item ***(Wisconsin Jan '08) By definition, a topological group is a set $G$ with both a topology and a group structure
such that the map $G \rightarrow G$ sending $x$ to $x^{-1}$ and the map $G \times G \rightarrow G$ sending
$(x,y)$ to $xy$ are both continuous.  Let $1 \in G$ denote the identity of this topological group $G$.  Show that
$\pi_1(G,1)$ is abelian.

\vfill

*** Possibly not appropriate.  Try if you have time.
\end{enumerate}


\end{document}