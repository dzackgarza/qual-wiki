\documentclass[psamsfonts, 11pt, reqno]{amsart}
\setlength{\topmargin}{0pt}
\setlength{\footskip}{10pt}
\setlength{\oddsidemargin}{2.5cm}
\setlength{\evensidemargin}{2.5cm}
\setlength{\textwidth}{329pt}
\addtolength{\textheight}{50pt}
\usepackage{amsfonts,amsmath, amsthm, amssymb, latexsym, amscd, epsfig, pdfsync}
\usepackage[all]{xy}
\usepackage[mathscr]{eucal}
%\usepackage{layout}
%\addtolength{\topmargin}{-90pt}
%\addtolength{\textheight}{140pt}
\addtolength{\textwidth}{120pt}
\addtolength{\hoffset}{-60pt}
\everymath={\displaystyle}

\newcommand{\seriesk}{\sum\limits_{k=1}^{\infty}}
\newcommand{\seriesn}{\sum\limits_{n=1}^{\infty}}
\newcommand{\z}{\mathbb{Z}}
\newcommand{\n}{\mathbb{N}}
\newcommand{\R}{\mathbb{R}}
\newcommand{\q}{\mathbb{Q}}
\newcommand{\C}{\mathbb{C}}
\newcommand{\la}{\langle}
\newcommand{\ra}{\rangle}
\newcommand{\x}{\underline{\underline{x}}}
\renewcommand{\r}{\underline{\underline{r}}}
\renewcommand{\t}{\mathcal{T}}
\renewcommand{\d}{\mathcal{D}}
\newcommand{\D}{\mathcal{D}}
\newcommand{\pic}{\mathbb{P}}
\newcommand{\p}{\mathcal{P}}
\newcommand{\Q}{\mathcal{Q}}
\newcommand{\ds}{\displaystyle}
\renewcommand{\tilde}{\widetilde}
\newcommand{\inv}{^{-1}}
\renewcommand{\oplus}{\bigoplus}
\renewcommand{\Sigma}{\sum}
\renewcommand{\Pi}{\prod}
\renewcommand{\phi}{\varphi}

\newtheorem{definition}{Definition}
\newtheorem{thm}[definition]{Theorem}
\newtheorem{theorem}[definition]{Theorem}
%\theoremstyle{definition}
%\newtheorem{fact}[defn]{Fact}
%\newtheorem{note}[defn]{Note}
\newtheorem{lemma}[definition]{Lemma}
%\newtheorem{cor}[defn]{Corollary}
%\newtheorem{conjecture}[defn]{Conjecture}
%\newtheorem{problem}[defn]{Problem}
%\newtheorem{notation}[defn]{Notation}
%\newtheorem*{question}{Question}
\newtheorem{ex}[definition]{Example}
\newtheorem{example}[definition]{Example}
%\newtheorem*{results}{Known Results}

\title{QualProblems}
%\author(,\\
%}

\begin{document}
\begin{center}
\begin{large}
Topology Qual Workshop Day 1: Compactness.
\end{large}
\end{center}
\vspace{.25in}

\begin{enumerate}
\item (Jan '02 \# A1) Let $X$ be a compact space, and let
$$A_1 \supseteq A_2 \supseteq \cdots \supseteq A_n \supseteq \cdots$$
be a descending chain of non-empty closed subsets of $X$.  Show that their
intersection $\cap_{n=1}^{\infty} A_n$ is non-empty.
\vfill

\item (June '04 \# A1) A topological space $X$ is \emph{locally compact} if every point in $X$ has an open
neighborhood whose closure is compact.  Show that the Cartesian product of two locally compact spaces,
with the product topology, is also locally compact.
\vfill

\item (June '05 \# A2) Let $X$ be the set of integers, let $\mathcal{C}_1 := \{ A \subseteq X | X \setminus A \mbox{ is finite }\}$,
and let $\mathcal{C}_2 := \{ A \subseteq X | 0 \notin A \}.$  Show that the union $\mathcal{T}:=\mathcal{C}_1 \cup \mathcal{C}_2$ is
a topology on $X$, and show that this topological space is compact.
\vfill

\item (Jan '06 \# A4) Let $\mathcal{T}, \mathcal{T}'$ be two topologies on $X$.  Show that if $(X, \mathcal{T})$ is compact and
Hausdorff, $\mathcal{T} \subseteq \mathcal{T}'$, and $\mathcal{T} \neq \mathcal{T}'$, then $(X, \mathcal{T}')$ is
Hausdorff but \emph{not} compact.

\vfill

\item (June '07 \# A4) A space $(X, \mathcal{T})$ is called \emph{locally compact} if, for every $x \in X$ and $x \in \mathcal{U} \in 
\mathcal{T}$, there is a compact set $C \subseteq \mathcal{U}$ containing an open neighborhood of $x$.  Show that
the Cartesian product of two locally compact spaces is locally compact.
\vfill

\item (June '11 \# A1) Prove that if $f: X\rightarrow Y$ is a continuous map between topological spaces and $C$
is a compact subset of $X$, then $f(C)$ is a compact subset of $Y$.
\vfill

\item (June '11 \# A4) Suppose $A, B$ are disjoint, compact subspaces of the Hausdorff topological space $X$.  Prove that
there are open subsets $U, V$ of $X$ such that $A \subseteq U, B \subseteq V$ and $U \cap V = \emptyset$.
\vfill

\item Suppose that $X$ is compact and $Y$ is Hausdorff.  Prove that every one-to-one, onto, continuous map
$f: X \rightarrow Y$ is a homeomorphism.
\vfill

\item (Purdue '11) Let $X$ be a set with two elements $\{ a, b \}$.  Give $X$ the indiscrete topology.  Give $X \times \mathbb{R}$ the
product topology.  Let $A \subset X \times \mathbb{R}$ be $( \{ a \} \times [0,1]) \cup (\{ b \} \times (0,1))$.  Prove that
$A$ is compact.
\vfill

\item (Purdue '11) Let $X$ be a compact space and let $\{ C_{\alpha} \}_{\alpha \in A}$ be a collection of closed sets in $X$.
Let $C = \cap_{\alpha \in A} C_{\alpha}$ and let $U$ be an open set containing $C$.  Prove there is a finite set $\alpha_1, ... \alpha_n$
in $A$ with $C_{\alpha_1} \cap \cdots \cap C_{\alpha_n} \subset U$.
\vfill

\end{enumerate}


\end{document}