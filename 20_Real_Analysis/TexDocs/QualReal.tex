\input{"preamble.tex"}

\addbibresource{QualReal.bib}

\let\Begin\begin
\let\End\end
\newcommand\wrapenv[1]{#1}

\makeatletter
\def\ScaleWidthIfNeeded{%
 \ifdim\Gin@nat@width>\linewidth
    \linewidth
  \else
    \Gin@nat@width
  \fi
}
\def\ScaleHeightIfNeeded{%
  \ifdim\Gin@nat@height>0.9\textheight
    0.9\textheight
  \else
    \Gin@nat@width
  \fi
}
\makeatother

\setkeys{Gin}{width=\ScaleWidthIfNeeded,height=\ScaleHeightIfNeeded,keepaspectratio}%

\title{
\textbf{
    Qual Real Analysis
  }
  }







\begin{document}

\date{}
\maketitle


\newpage

% Note: addsec only in KomaScript
\addsec{Table of Contents}
\tableofcontents
\newpage

\hypertarget{preface}{%
\section{Preface}\label{preface}}

\begin{quote}
Note: linking directly to sections doesn't seem to work yet. Just ctrl-F
and search the page for the relevant year.
\end{quote}

I'd like to extend my gratitude to Peter Woolfitt for supplying many
solutions and checking many proofs of the rest in problem sessions. Many
other solutions contain input and ideas from other graduate students and
faculty members at UGA, along with questions and answers posted on Math
Stack Exchange or Math Overflow.

\hypertarget{undergraduate-analysis-uniform-convergence}{%
\section{Undergraduate Analysis: Uniform
Convergence}\label{undergraduate-analysis-uniform-convergence}}

\hypertarget{fall-2018.1}{%
\subsection{Fall 2018.1}\label{fall-2018.1}}

Let \(f(x) = \frac 1 x\). Show that \(f\) is uniformly continuous on
\((1, \infty)\) but not on \((0,\infty)\).

\emph{Concept review omitted.}

\emph{Strategy omitted.}

\emph{Solution omitted.}

\hypertarget{fall-2017.1}{%
\subsection{Fall 2017.1}\label{fall-2017.1}}

Let
\begin{align*}
f(x) = \sum _{n=0}^{\infty} \frac{x^{n}}{n !}.
\end{align*}
Describe the intervals on which \(f\) does and does not converge
uniformly.

\emph{Concept review omitted.}

\emph{Strategy omitted.}

\emph{Solution omitted.}

\hypertarget{spring-2017.4}{%
\subsection{Spring 2017.4}\label{spring-2017.4}}

Let \(f(x, y)\) on \([-1, 1]^2\) be defined by
\begin{align*}
f(x, y) = \begin{cases}
\frac{x y}{\left(x^{2}+y^{2}\right)^{2}} & (x, y) \neq (0, 0) \\
0 & (x, y) = (0, 0)
\end{cases}
\end{align*}
Determine if \(f\) is integrable.

\emph{Concept review omitted.}

\emph{Solution omitted.}

\hypertarget{fall-2014.1}{%
\subsection{Fall 2014.1}\label{fall-2014.1}}

Let \(\left\{{f_n}\right\}\) be a sequence of continuous functions such
that \(\sum f_n\) converges uniformly.

Prove that \(\sum f_n\) is also continuous.

\emph{Concept review omitted.}

\emph{Solution omitted.}

\hypertarget{spring-2015.1}{%
\subsection{Spring 2015.1}\label{spring-2015.1}}

Let \((X, d)\) and \((Y, \rho)\) be metric spaces, \(f: X\to Y\), and
\(x_0 \in X\).

Prove that the following statements are equivalent:

\begin{enumerate}
\def\labelenumi{\arabic{enumi}.}
\tightlist
\item
  For every \(\varepsilon > 0 \quad \exists \delta > 0\) such that
  \(\rho( f(x), f(x_0) ) < \varepsilon\) whenever
  \(d(x, x_0) < \delta\).
\item
  The sequence \(\left\{{f(x_n)}\right\}_{n=1}^\infty \to f(x_0)\) for
  every sequence \(\left\{{x_n}\right\} \to x_0\) in \(X\).
\end{enumerate}

\emph{Concept review omitted.}

\emph{Solution omitted.}

\hypertarget{fall-2014.2}{%
\subsection{Fall 2014.2}\label{fall-2014.2}}

Let \(I\) be an index set and \(\alpha: I \to (0, \infty)\).

\begin{enumerate}
\def\labelenumi{\alph{enumi}.}
\item
  Show that
  \begin{align*}
  \sum_{i \in I} a(i):=\sup _{\substack{ J \subset I \\ J \text { finite }}} \sum_{i \in J} a(i)<\infty \implies I \text{ is countable.}
  \end{align*}
\item
  Suppose \(I = {\mathbb{Q}}\) and
  \(\sum_{q \in \mathbb{Q}} a(q)<\infty\). Define
  \begin{align*}
  f(x):=\sum_{\substack{q \in \mathbb{Q}\\ q \leq x}} a(q).
  \end{align*}
  Show that \(f\) is continuous at \(x \iff x\not\in {\mathbb{Q}}\).
\end{enumerate}

\emph{Concept review omitted.}

\emph{Solution omitted.}

\hypertarget{general-analysis}{%
\section{General Analysis}\label{general-analysis}}

\hypertarget{fall-2021.1}{%
\subsection{Fall 2021.1}\label{fall-2021.1}}

\begin{problem}[?]

Let \(\left\{x_{n}\right\}_{n-1}^{\infty}\) be a sequence of real
numbers such that \(x_{1}>0\) and
\begin{align*}
x_{n+1}=1-\left(2+x_{n}\right)^{-1}=\frac{1+x_{n}}{2+x_{n}} \text {. }
\end{align*}
Prove that the sequence \(\left\{x_{n}\right\}\) converges, and find its
limit.

\end{problem}

\emph{Solution omitted.}

\hypertarget{fall-2020.1}{%
\subsection{Fall 2020.1}\label{fall-2020.1}}

\begin{problem}[?]

Show that if \(x_n\) is a decreasing sequence of positive real numbers
such that \(\sum_{n=1}^\infty x_n\) converges, then
\begin{align*}
\lim_{n\to\infty} n x_n = 0.
\end{align*}

\end{problem}

\emph{Solution omitted.}

\hypertarget{spring-2020.1}{%
\subsection{Spring 2020.1}\label{spring-2020.1}}

Prove that if \(f: [0, 1] \to {\mathbb{R}}\) is continuous then
\begin{align*}
\lim_{k\to\infty} \int_0^1 kx^{k-1} f(x) \,dx = f(1)
.\end{align*}

\emph{Concept review omitted.}

\emph{Solution omitted.}

\hypertarget{fall-2019.1}{%
\subsection{Fall 2019.1}\label{fall-2019.1}}

Let \(\{a_n\}_{n=1}^\infty\) be a sequence of real numbers.

\begin{enumerate}
\def\labelenumi{\alph{enumi}.}
\item
  Prove that if \(\displaystyle\lim_{n\to \infty } a_n = 0\), then
  \begin{align*}
  \lim _{n \rightarrow \infty} \frac{a_{1}+\cdots+a_{n}}{n}=0
  \end{align*}
\item
  Prove that if \(\displaystyle\sum_{n=1}^{\infty} \frac{a_{n}}{n}\)
  converges, then
  \begin{align*}
  \lim _{n \rightarrow \infty} \frac{a_{1}+\cdots+a_{n}}{n}=0
  \end{align*}
\end{enumerate}

\emph{Solution omitted.}

\hypertarget{fall-2018.4}{%
\subsection{Fall 2018.4}\label{fall-2018.4}}

Let \(f\in L^1([0, 1])\). Prove that
\begin{align*}
\lim_{n \to \infty} \int_{0}^{1} f(x) {\left\lvert {\sin n x} \right\rvert} ~d x= \frac{2}{\pi} \int_{0}^{1} f(x) ~d x
\end{align*}

\begin{quote}
Hint: Begin with the case that \(f\) is the characteristic function of
an interval.
\end{quote}

\todo[inline]{Ask someone to check the last approximation part.}

\#todo

\emph{Solution omitted.}

\hypertarget{fall-2017.4}{%
\subsection{Fall 2017.4}\label{fall-2017.4}}

Let
\begin{align*}
f_{n}(x) = n x(1-x)^{n}, \quad n \in {\mathbb{N}}.
\end{align*}

\begin{enumerate}
\def\labelenumi{\alph{enumi}.}
\item
  Show that \(f_n \to 0\) pointwise but not uniformly on \([0, 1]\).
\item
  Show that
  \begin{align*}
  \lim _{n \to \infty} \int _{0}^{1} n(1-x)^{n} \sin x \, dx = 0
  \end{align*}
\end{enumerate}

\begin{quote}
Hint for (a): Consider the maximum of \(f_n\).
\end{quote}

\emph{Solution omitted.}

\hypertarget{spring-2017.3}{%
\subsection{Spring 2017.3}\label{spring-2017.3}}

Let
\begin{align*}
f_{n}(x) = a e^{-n a x} - b e^{-n b x} \quad \text{ where } 0 < a < b.
\end{align*}

Show that

\begin{enumerate}
\def\labelenumi{\alph{enumi}.}
\tightlist
\item
  \(\sum_{n=1}^{\infty} \left|f_{n}\right|\) is not in
  \(L^{1}([0, \infty), m)\)
\end{enumerate}

\begin{quote}
Hint: \(f_n(x)\) has a root \(x_n\).
\end{quote}

\begin{enumerate}
\def\labelenumi{\alph{enumi}.}
\setcounter{enumi}{1}
\tightlist
\item

  \begin{align*}
  \sum_{n=1}^{\infty} f_{n} \text { is in } L^{1}([0, \infty), m) 
  {\quad \operatorname{and} \quad}
  \int _{0}^{\infty} \sum _{n=1}^{\infty} f_{n}(x) \,dm = \ln \frac{b}{a}
  \end{align*}
  \todo[inline]{Not complete.} \todo[inline]{Add concepts.}
  \todo[inline]{Walk through.}
\end{enumerate}

\emph{Solution omitted.}

\hypertarget{fall-2016.1}{%
\subsection{Fall 2016.1}\label{fall-2016.1}}

Define
\begin{align*}
f(x) = \sum_{n=1}^{\infty} \frac{1}{n^{x}}.
\end{align*}
Show that \(f\) converges to a differentiable function on
\((1, \infty)\) and that
\begin{align*}
f'(x)  =\sum_{n=1}^{\infty}\left(\frac{1}{n^{x}}\right)^{\prime}.
\end{align*}

\begin{quote}
Hint:
\begin{align*}
\left(\frac{1}{n^{x}}\right)' = -\frac{1}{n^{x}} \ln n
\end{align*}
\end{quote}

\todo[inline]{Add concepts.}

\emph{Solution omitted.}

\hypertarget{fall-2016.5}{%
\subsection{Fall 2016.5}\label{fall-2016.5}}

Let \(\phi\in L^\infty({\mathbb{R}})\). Show that the following limit
exists and satisfies the equality
\begin{align*}
\lim _{n \to \infty} \left(\int _{\mathbb{R}} \frac{|\phi(x)|^{n}}{1+x^{2}} \, dx \right) ^ {\frac{1}{n}} 
= {\left\lVert {\phi} \right\rVert}_\infty.
\end{align*}
\todo[inline]{Add concepts.}

\emph{Solution omitted.}

\hypertarget{fall-2016.6}{%
\subsection{Fall 2016.6}\label{fall-2016.6}}

Let \(f, g \in L^2({\mathbb{R}})\). Show that
\begin{align*}
\lim _{n \to \infty} \int _{{\mathbb{R}}} f(x) g(x+n) \,dx = 0
\end{align*}

\todo[inline]{Rewrite solution.}

\emph{Concept review omitted.}

\emph{Solution omitted.}

\hypertarget{spring-2016.1}{%
\subsection{Spring 2016.1}\label{spring-2016.1}}

For \(n\in {\mathbb{N}}\), define
\begin{align*}
e_{n} = \left (1+ {1\over n} \right)^{n} 
{\quad \operatorname{and} \quad}
E_{n} = \left( 1+ {1\over n} \right)^{n+1}
\end{align*}

Show that \(e_n < E_n\), and prove Bernoulli's inequality:
\begin{align*}
(1+x)^n \geq 1+nx && -1 < x < \infty  ,\,\, n\in {\mathbb{N}}
.\end{align*}

Use this to show the following:

\begin{enumerate}
\def\labelenumi{\arabic{enumi}.}
\tightlist
\item
  The sequence \(e_n\) is increasing.
\item
  The sequence \(E_n\) is decreasing.
\item
  \(2 < e_n < E_n < 4\).
\item
  \(\lim _{n \to \infty} e_{n} = \lim _{n \to \infty} E_{n}\).
\end{enumerate}

\hypertarget{fall-2015.1}{%
\subsection{Fall 2015.1}\label{fall-2015.1}}

Define
\begin{align*}
f(x)=c_{0}+c_{1} x^{1}+c_{2} x^{2}+\ldots+c_{n} x^{n} \text { with } n \text { even and } c_{n}>0.
\end{align*}

Show that there is a number \(x_m\) such that \(f(x_m) \leq f(x)\) for
all \(x\in {\mathbb{R}}\).

\hypertarget{spring-2014.2}{%
\subsection{Spring 2014.2}\label{spring-2014.2}}

Let \(\left\{{a_n}\right\}\) be a sequence of real numbers such that
\begin{align*}
\left\{{b_n}\right\} \in \ell^2({\mathbb{N}}) \implies \sum a_n b_n < \infty.
\end{align*}
Show that \(\sum a_n^2 < \infty\).

\begin{quote}
Note: Assume \(a_n, b_n\) are all non-negative.
\end{quote}

\todo[inline]{Have someone check!}

\emph{Solution omitted.}

\hypertarget{measure-theory-sets}{%
\section{Measure Theory: Sets}\label{measure-theory-sets}}

\hypertarget{fall-2021.3}{%
\subsection{Fall 2021.3}\label{fall-2021.3}}

Recall that a set \(E \subset \mathbb{R}^{d}\) is measurable if for
every \(c>0\) there is an open set \(U \subseteq {\mathbb{R}}^d\) such
that \(m^{*}(U \setminus E)<\epsilon\).

\begin{enumerate}
\def\labelenumi{\alph{enumi}.}
\item
  Prove that if \(E\) is measurable then for all \(\epsilon>0\) there
  exists an elementary \(\operatorname{set} F\), such that
  \(m(E \Delta F)<\epsilon\).

  Here \(m(E)\) denotes the Lebesgue measure of \(E\), a set \(F\) is
  called elementary if it is a finite union of rectangles and
  \(E \Delta F\) denotes the symmetric difference of the sets \(E\) and
  \(F\).
\item
  Let \(E \subset \mathbb{R}\) be a measurable set, such that
  \(0<m(E)<\infty\). Use part (a) to show that
  \begin{align*}
  \lim _{n \rightarrow \infty} \int_{E} \sin (n t) d t=0
  \end{align*}
\end{enumerate}

\hypertarget{spring-2020.2}{%
\subsection{Spring 2020.2}\label{spring-2020.2}}

Let \(m_*\) denote the Lebesgue outer measure on \({\mathbb{R}}\).

a.. Prove that for every \(E\subseteq {\mathbb{R}}\) there exists a
Borel set \(B\) containing \(E\) such that
\begin{align*}
m_*(B) = m_*(E)
.\end{align*}

b.. Prove that if \(E\subseteq {\mathbb{R}}\) has the property that
\begin{align*}
m_*(A) = m_*(A\displaystyle\bigcap E) + m_*(A\displaystyle\bigcap E^c)
\end{align*}
for every set \(A\subseteq {\mathbb{R}}\), then there exists a Borel set
\(B\subseteq {\mathbb{R}}\) such that \(E = B\setminus N\) with
\(m_*(N) = 0\).

Be sure to address the case when \(m_*(E) = \infty\).

\emph{Concept review omitted.}

\emph{Solution omitted.}

\hypertarget{fall-2019.3.}{%
\subsection{Fall 2019.3.}\label{fall-2019.3.}}

Let \((X, \mathcal B, \mu)\) be a measure space with \(\mu(X) = 1\) and
\(\{B_n\}_{n=1}^\infty\) be a sequence of \(\mathcal B\)-measurable
subsets of \(X\), and
\begin{align*}
B \coloneqq\left\{{x\in X {~\mathrel{\Big\vert}~}x\in B_n \text{ for infinitely many } n}\right\}.
\end{align*}

\begin{enumerate}
\def\labelenumi{\alph{enumi}.}
\item
  Argue that \(B\) is also a \(\mathcal{B} {\hbox{-}}\)measurable subset
  of \(X\).
\item
  Prove that if \(\sum_{n=1}^\infty \mu(B_n) < \infty\) then
  \(\mu(B)= 0\).
\item
  Prove that if \(\sum_{n=1}^\infty \mu(B_n) = \infty\) \textbf{and} the
  sequence of set complements \(\left\{{B_n^c}\right\}_{n=1}^\infty\)
  satisfies
  \begin{align*}
  \mu\left(\bigcap_{n=k}^{K} B_{n}^{c}\right)=\prod_{n=k}^{K}\left(1-\mu\left(B_{n}\right)\right)
  \end{align*}
  for all positive integers \(k\) and \(K\) with \(k < K\), then
  \(\mu(B) = 1\).
\end{enumerate}

\begin{quote}
Hint: Use the fact that \(1 - x ≤ e^{-x}\) for all \(x\).
\end{quote}

\emph{Concept review omitted.}

\emph{Solution omitted.}

\hypertarget{spring-2019.2}{%
\subsection{Spring 2019.2}\label{spring-2019.2}}

Let \(\mathcal B\) denote the set of all Borel subsets of
\({\mathbb{R}}\) and \(\mu : \mathcal B \to [0, \infty)\) denote a
finite Borel measure on \({\mathbb{R}}\).

\begin{enumerate}
\def\labelenumi{\alph{enumi}.}
\item
  Prove that if \(\{F_k\}\) is a sequence of Borel sets for which
  \(F_k \supseteq F_{k+1}\) for all \(k\), then
  \begin{align*}
  \lim _{k \rightarrow \infty} \mu\left(F_{k}\right)=\mu\left(\bigcap_{k=1}^{\infty} F_{k}\right)
  \end{align*}
\item
  Suppose \(\mu\) has the property that \(\mu (E) = 0\) for every
  \(E \in \mathcal B\) with Lebesgue measure \(m(E) = 0\). Prove that
  for every \(\epsilon > 0\) there exists \(\delta > 0\) so that if
  \(E \in \mathcal B\) with \(m(E) < δ\), then \(\mu(E) < ε\).
\end{enumerate}

\emph{Concept review omitted.}

\emph{Strategy omitted.}

\emph{Solution omitted.}

\hypertarget{fall-2018.2}{%
\subsection{Fall 2018.2}\label{fall-2018.2}}

Let \(E\subset {\mathbb{R}}\) be a Lebesgue measurable set. Show that
there is a Borel set \(B \subset E\) such that \(m(E\setminus B) = 0\).

\todo[inline]{Move this to review notes to clean things up.}

\todo[inline]{What a mess, redo!!}

\emph{Concept review omitted.}

\emph{Solution omitted.}

\hypertarget{spring-2018.1}{%
\subsection{Spring 2018.1}\label{spring-2018.1}}

Define
\begin{align*}
E:=\left\{x \in \mathbb{R}:\left|x-\frac{p}{q}\right|<q^{-3} \text { for infinitely many } p, q \in \mathbb{N}\right\}.
\end{align*}

Prove that \(m(E) = 0\).

\emph{Concept review omitted.}

\emph{Solution omitted.}

\hypertarget{fall-2017.2}{%
\subsection{Fall 2017.2}\label{fall-2017.2}}

Let \(f(x) = x^2\) and
\(E \subset [0, \infty) \coloneqq{\mathbb{R}}^+\).

\begin{enumerate}
\def\labelenumi{\arabic{enumi}.}
\item
  Show that
  \begin{align*}
  m^*(E) = 0 \iff m^*(f(E)) = 0.
  \end{align*}
\item
  Deduce that the map
\end{enumerate}

\begin{align*}
\phi: \mathcal{L}({\mathbb{R}}^+) &\to \mathcal{L}({\mathbb{R}}^+) \\
E &\mapsto f(E)
\end{align*}
is a bijection from the class of Lebesgue measurable sets of
\([0, \infty)\) to itself.

\todo[inline]{Walk through.}

\emph{Solution omitted.}

\hypertarget{spring-2017.1}{%
\subsection{Spring 2017.1}\label{spring-2017.1}}

Let \(K\) be the set of numbers in \([0, 1]\) whose decimal expansions
do not use the digit \(4\).

\begin{quote}
We use the convention that when a decimal number ends with 4 but all
other digits are different from 4, we replace the digit \(4\) with
\(399\cdots\). For example, \(0.8754 = 0.8753999\cdots\).
\end{quote}

Show that \(K\) is a compact, nowhere dense set without isolated points,
and find the Lebesgue measure \(m(K)\).

\emph{Concept review omitted.}

\emph{Solution omitted.}

\hypertarget{spring-2017.2}{%
\subsection{Spring 2017.2}\label{spring-2017.2}}

\begin{enumerate}
\def\labelenumi{\alph{enumi}.}
\tightlist
\item
  Let \(\mu\) be a measure on a measurable space \((X, \mathcal M)\) and
  \(f\) a positive measurable function.
\end{enumerate}

Define a measure \(\lambda\) by
\begin{align*}
\lambda(E):=\int_{E} f ~d \mu, \quad E \in \mathcal{M}
\end{align*}

Show that for \(g\) any positive measurable function,
\begin{align*}
\int_{X} g ~d \lambda=\int_{X} f g ~d \mu
\end{align*}

\begin{enumerate}
\def\labelenumi{\alph{enumi}.}
\setcounter{enumi}{1}
\tightlist
\item
  Let \(E \subset {\mathbb{R}}\) be a measurable set such that
  \begin{align*}
  \int_{E} x^{2} ~d m=0.
  \end{align*}
  Show that \(m(E) = 0\).
\end{enumerate}

\emph{Concept review omitted.}

\emph{Solution omitted.}

\hypertarget{fall-2016.4}{%
\subsection{Fall 2016.4}\label{fall-2016.4}}

Let \((X, \mathcal M, \mu)\) be a measure space and suppose
\(\left\{{E_n}\right\} \subset \mathcal M\) satisfies
\begin{align*}
\lim _{n \rightarrow \infty} \mu\left(X \backslash E_{n}\right)=0.
\end{align*}

Define
\begin{align*}
G \coloneqq\left\{{x\in X {~\mathrel{\Big\vert}~}x\in E_n \text{ for only finitely many  } n}\right\}.
\end{align*}

Show that \(G \in \mathcal M\) and \(\mu(G) = 0\).

\todo[inline]{Add concepts.}

\emph{Solution omitted.}

\hypertarget{spring-2016.3}{%
\subsection{Spring 2016.3}\label{spring-2016.3}}

Let \(f\) be Lebesgue measurable on \({\mathbb{R}}\) and
\(E \subset {\mathbb{R}}\) be measurable such that
\begin{align*}
0<A=\int_{E} f(x) d x<\infty.
\end{align*}

Show that for every \(0 < t < 1\), there exists a measurable set
\(E_t \subset E\) such that
\begin{align*}
\int_{E_{t}} f(x) d x=t A.
\end{align*}

\hypertarget{spring-2016.5}{%
\subsection{Spring 2016.5}\label{spring-2016.5}}

Let \((X, \mathcal M, \mu)\) be a measure space. For \(f\in L^1(\mu)\)
and \(\lambda > 0\), define
\begin{align*}
\phi(\lambda)=\mu(\{x \in X | f(x)>\lambda\}) 
\quad \text { and } \quad 
\psi(\lambda)=\mu(\{x \in X | f(x)<-\lambda\})
\end{align*}

Show that \(\phi, \psi\) are Borel measurable and
\begin{align*}
\int_{X}|f| ~d \mu=\int_{0}^{\infty}[\phi(\lambda)+\psi(\lambda)] ~d \lambda
\end{align*}

\hypertarget{spring-2016.2}{%
\subsection{Spring 2016.2}\label{spring-2016.2}}

Let \(0 < \lambda < 1\) and construct a Cantor set \(C_\lambda\) by
successively removing middle intervals of length \(\lambda\).

Prove that \(m(C_\lambda) = 0\).

\hypertarget{fall-2015.2}{%
\subsection{Fall 2015.2}\label{fall-2015.2}}

Let \(f: {\mathbb{R}}\to {\mathbb{R}}\) be Lebesgue measurable.

\begin{enumerate}
\def\labelenumi{\arabic{enumi}.}
\tightlist
\item
  Show that there is a sequence of simple functions \(s_n(x)\) such that
  \(s_n(x) \to f(x)\) for all \(x\in {\mathbb{R}}\).
\item
  Show that there is a Borel measurable function \(g\) such that
  \(g = f\) almost everywhere.
\end{enumerate}

\hypertarget{spring-2015.3}{%
\subsection{Spring 2015.3}\label{spring-2015.3}}

Let \(\mu\) be a finite Borel measure on \({\mathbb{R}}\) and
\(E \subset {\mathbb{R}}\) Borel. Prove that the following statements
are equivalent:

\begin{enumerate}
\def\labelenumi{\arabic{enumi}.}
\tightlist
\item
  \(\forall \varepsilon > 0\) there exists \(G\) open and \(F\) closed
  such that
  \begin{align*}
  F \subseteq E \subseteq G \quad \text{and} \quad \mu(G\setminus F) < \varepsilon.
  \end{align*}
\item
  There exists a \(V \in G_\delta\) and \(H \in F_\sigma\) such that
  \begin{align*}
  H \subseteq E \subseteq V \quad \text{and}\quad \mu(V\setminus H) = 0
  \end{align*}
\end{enumerate}

\hypertarget{spring-2014.3}{%
\subsection{Spring 2014.3}\label{spring-2014.3}}

Let \(f: {\mathbb{R}}\to {\mathbb{R}}\) and suppose
\begin{align*}
\forall x\in {\mathbb{R}},\quad f(x) \geq \limsup _{y \rightarrow x} f(y)
\end{align*}
Prove that \(f\) is Borel measurable.

\hypertarget{spring-2014.4}{%
\subsection{Spring 2014.4}\label{spring-2014.4}}

Let \((X, \mathcal M, \mu)\) be a measure space and suppose \(f\) is a
measurable function on \(X\). Show that
\begin{align*}
\lim _{n \rightarrow \infty} \int_{X} f^{n} ~d \mu =
\begin{cases}
\infty & \text{or} \\
\mu(f^{-1}(1)),
\end{cases}
\end{align*}
and characterize the collection of functions of each type.

\hypertarget{measure-theory-functions}{%
\section{Measure Theory: Functions}\label{measure-theory-functions}}

\hypertarget{spring-2021.1}{%
\subsection{Spring 2021.1}\label{spring-2021.1}}

\begin{problem}[Spring 2021, 1]

Let \((X, \mathcal{M},\mu)\) be a measure space and let
\(E_n \in \mathcal{M}\) be a measurable set for \(n\geq 1\). Let
\(f_n \coloneqq\chi_{E_n}\) be the indicator function of the set \(E\)
and show that

\begin{enumerate}
\def\labelenumi{\alph{enumi}.}
\item
  \(f_n \overset{n\to\infty}\to 1\) uniformly \(\iff\) there exists
  \(N\in {\mathbb{N}}\) such that \(E_n = X\) for all \(n\geq N\).
\item
  \(f_n(x) \overset{n\to\infty}\to 1\) for almost every \(x\) \(\iff\)
  \begin{align*}
  \mu \qty{ \displaystyle\bigcap_{n \geq 0} \displaystyle\bigcup_{k \geq n} (X \setminus E_k) } = 0
  .\end{align*}
\end{enumerate}

\end{problem}

\emph{Solution omitted.}

\hypertarget{spring-2021.3}{%
\subsection{Spring 2021.3}\label{spring-2021.3}}

Let \((X, \mathcal{M}, \mu)\) be a finite measure space and let
\(\left\{{ f_n}\right\}_{n=1}^{\infty } \subseteq L^1(X, \mu)\). Suppose
\(f\in L^1(X, \mu)\) such that \(f_n(x) \overset{n\to \infty }\to f(x)\)
for almost every \(x \in X\). Prove that for every \({\varepsilon}> 0\)
there exists \(M>0\) and a set \(E\subseteq X\) such that
\(\mu(E) \leq {\varepsilon}\) and
\({\left\lvert {f_n(x)} \right\rvert}\leq M\) for all
\(x\in X\setminus E\) and all \(n\in {\mathbb{N}}\).

\hypertarget{fall-2020.2}{%
\subsection{Fall 2020.2}\label{fall-2020.2}}

\begin{enumerate}
\def\labelenumi{\alph{enumi}.}
\item
  Let \(f: {\mathbb{R}}\to {\mathbb{R}}\). Prove that
  \begin{align*}
  f(x) \leq \liminf_{y\to x} f(y)~ \text{for each}~ x\in {{\mathbb{R}}} \iff \{ x\in {{\mathbb{R}}} \mathrel{\Big|}f(x) > a \}~\text{is open for all}~ a\in {{\mathbb{R}}}
  \end{align*}
\item
  Recall that a function \(f: {{\mathbb{R}}} \to {{\mathbb{R}}}\) is
  called \emph{lower semi-continuous} iff it satisfies either condition
  in part (a) above.
\end{enumerate}

Prove that if \(\mathcal{F}\) is any family of lower semi-continuous
functions, then
\begin{align*}
g(x) = \sup\{ f(x) \mathrel{\Big|}f\in \mathcal{F}\}
\end{align*}
is Borel measurable.

\begin{quote}
Note that \(\mathcal{F}\) need not be a countable family.
\end{quote}

\hypertarget{fall-2016.2}{%
\subsection{Fall 2016.2}\label{fall-2016.2}}

Let \(f, g: [a, b] \to {\mathbb{R}}\) be measurable with
\begin{align*}
\int_{a}^{b} f(x) ~d x=\int_{a}^{b} g(x) ~d x.
\end{align*}
Show that either

\begin{enumerate}
\def\labelenumi{\arabic{enumi}.}
\tightlist
\item
  \(f(x) = g(x)\) almost everywhere, or
\item
  There exists a measurable set \(E \subset [a, b]\) such that
  \begin{align*}
  \int _{E} f(x) \, dx > \int _{E} g(x) \, dx
  \end{align*}
\end{enumerate}

\emph{Concept review omitted.}

\emph{Strategy omitted.}

\emph{Solution omitted.}

\hypertarget{spring-2016.4}{%
\subsection{Spring 2016.4}\label{spring-2016.4}}

Let \(E \subset {\mathbb{R}}\) be measurable with \(m(E) < \infty\).
Define
\begin{align*}
f(x)=m(E \cap(E+x)).
\end{align*}

Show that

\begin{enumerate}
\def\labelenumi{\arabic{enumi}.}
\tightlist
\item
  \(f\in L^1({\mathbb{R}})\).
\item
  \(f\) is uniformly continuous.
\item
  \(\lim _{|x| \to \infty} f(x) = 0\).
\end{enumerate}

\begin{quote}
Hint:
\begin{align*}
\chi_{E \cap(E+x)}(y)=\chi_{E}(y) \chi_{E}(y-x)
\end{align*}
\end{quote}

\hypertarget{integrals-convergence}{%
\section{Integrals: Convergence}\label{integrals-convergence}}

\hypertarget{fall-2020.3}{%
\subsection{Fall 2020.3}\label{fall-2020.3}}

\begin{problem}[?]

Let \(f\) be a non-negative Lebesgue measurable function on
\([1, \infty)\).

\begin{enumerate}
\def\labelenumi{\alph{enumi}.}
\item
  Prove that
  \begin{align*}  
  1 \leq \qty{
  {1 \over b-a} \int_a^b f(x) \,dx
  }\qty{
  {1\over b-a} \int_a^b {1 \over f(x)}\, dx
  }
  \end{align*}
  for any \(1\leq a < b <\infty\).
\item
  Prove that if \(f\) satisfies
  \begin{align*}  
  \int_1^t f(x) \, dx \leq t^2 \log(t)
  \end{align*}
  for all \(t\in [1, \infty)\), then
  \begin{align*}  
  \int_1^\infty {1\over f(x)}\,dx= \infty
  .\end{align*}
\end{enumerate}

\begin{quote}
Hint: write
\begin{align*}  
\int_1^\infty {1\over f(x) }\,dx= \sum_{k=0}^\infty \int_{2^k}^{2^{k+1}} {1 \over f(x)}\,dx
.\end{align*}
\end{quote}

\end{problem}

\emph{Solution omitted.}

\hypertarget{spring-2021.2}{%
\subsection{Spring 2021.2}\label{spring-2021.2}}

\begin{problem}[?]

Calculate the following limit, justifying each step of your calculation:
\begin{align*}
L \coloneqq\lim_{n\to \infty} \int_0^n { \cos\qty{x\over n} \over x^2 + \cos\qty{x\over n} }\,dx
.\end{align*}

\end{problem}

\emph{Solution omitted.}

\hypertarget{spring-2021.5}{%
\subsection{Spring 2021.5}\label{spring-2021.5}}

\begin{problem}[?]

Let \(f_n \in L^2([0, 1])\) for \(n\in {\mathbb{N}}\), and assume that

\begin{itemize}
\item
  \({\left\lVert {f_n} \right\rVert}_2 \leq n^{-51 \over 100}\) for all
  \(n\in {\mathbb{N}}\),
\item
  \(\widehat{f}_n\) is supported in the interval \([2^n, 2^{n+1}]\), so
  \begin{align*}
  \widehat{f}_n(\xi) \coloneqq\int_0^1 f_n(x) e^{2\pi i \xi \cdot x} \,dx= 0 && \text{for } \xi \not\in [2^n, 2^{n+1}]
  .\end{align*}
\end{itemize}

Prove that \(\sum_{n\in {\mathbb{N}}} f_n\) converges in the Hilbert
space \(L^2([0, 1])\).

\begin{quote}
Hint: Plancherel's identity may be helpful.
\end{quote}

\end{problem}

\begin{warnings}

Although this mentions Plancherel, probably what is needed is Parseval's
identity:
\begin{align*}
\sum_{k\in {\mathbb{Z}}} {\left\lvert {\widehat{f}(k)} \right\rvert}^2 = \int_0^1 {\left\lvert {f(x)} \right\rvert}^2\,dx
.\end{align*}

\end{warnings}

\hypertarget{fall-2019.2}{%
\subsection{Fall 2019.2}\label{fall-2019.2}}

Prove that
\begin{align*}
\left| \frac{d^{n}}{d x^{n}} \frac{\sin x}{x}\right| \leq \frac{1}{n}
\end{align*}

for all \(x \neq 0\) and positive integers \(n\).

\begin{quote}
Hint: Consider \(\displaystyle\int_0^1 \cos(tx) dt\)
\end{quote}

\emph{Solution omitted.}

\hypertarget{spring-2020.5}{%
\subsection{Spring 2020.5}\label{spring-2020.5}}

Compute the following limit and justify your calculations:
\begin{align*}
\lim_{n\to\infty} \int_0^n \qty{1 + {x^2 \over n}}^{-(n+1)} \,dx
.\end{align*}

\todo[inline]{Not finished, flesh out.}
\todo[inline]{Walk through.}

\emph{Solution omitted.}

\hypertarget{spring-2019.3}{%
\subsection{Spring 2019.3}\label{spring-2019.3}}

Let \(\{f_k\}\) be any sequence of functions in \(L^2([0, 1])\)
satisfying \({\left\lVert {f_k} \right\rVert}_2 ≤ M\) for all
\(k ∈ {\mathbb{N}}\).

Prove that if \(f_k \to f\) almost everywhere, then \(f ∈ L^2([0, 1])\)
with \({\left\lVert {f} \right\rVert}_2 ≤ M\) and
\begin{align*}
\lim _{k \rightarrow \infty} \int_{0}^{1} f_{k}(x) dx = \int_{0}^{1} f(x) d x
\end{align*}

\begin{quote}
Hint: Try using Fatou's Lemma to show that
\({\left\lVert {f} \right\rVert}_2 ≤ M\) and then try applying Egorov's
Theorem.
\end{quote}

\emph{Solution omitted.}

\hypertarget{fall-2018.6}{%
\subsection{Fall 2018.6}\label{fall-2018.6}}

Compute the following limit and justify your calculations:
\begin{align*}
\lim_{n \rightarrow \infty} \int_{1}^{n} \frac{d x}{\left(1+\frac{x}{n}\right)^{n} \sqrt[n]{x}}
\end{align*}

\todo[inline]{Add concepts.}

\emph{Solution omitted.}

\hypertarget{fall-2018.3}{%
\subsection{Fall 2018.3}\label{fall-2018.3}}

Suppose \(f(x)\) and \(xf(x)\) are integrable on \({\mathbb{R}}\).
Define \(F\) by
\begin{align*}
F(t)\coloneqq\int _{-\infty}^{\infty} f(x) \cos (x t) dx
\end{align*}
Show that
\begin{align*}
F'(t)=-\int _{-\infty}^{\infty} x f(x) \sin (x t) dx
.\end{align*}

\todo[inline]{Walk through.}

\emph{Solution omitted.}

\hypertarget{spring-2018.5}{%
\subsection{Spring 2018.5}\label{spring-2018.5}}

Suppose that

\begin{itemize}
\tightlist
\item
  \(f_n, f \in L^1\),
\item
  \(f_n \to f\) almost everywhere, and
\item
  \(\int\left|f_{n}\right| \rightarrow \int|f|\).
\end{itemize}

Show that \(\int f_{n} \rightarrow \int f\).

\emph{Solution omitted.}

\hypertarget{spring-2018.2}{%
\subsection{Spring 2018.2}\label{spring-2018.2}}

Let
\begin{align*}
f_{n}(x):=\frac{x}{1+x^{n}}, \quad x \geq 0.
\end{align*}

\begin{enumerate}
\def\labelenumi{\alph{enumi}.}
\item
  Show that this sequence converges pointwise and find its limit. Is the
  convergence uniform on \([0, \infty)\)?
\item
  Compute
  \begin{align*}
  \lim _{n \rightarrow \infty} \int_{0}^{\infty} f_{n}(x) d x
  \end{align*}
\end{enumerate}

\todo[inline]{Add concepts.}

\emph{Solution omitted.}

\hypertarget{fall-2016.3}{%
\subsection{Fall 2016.3}\label{fall-2016.3}}

Let \(f\in L^1({\mathbb{R}})\). Show that
\begin{align*}
\lim _{x \to 0} \int _{{\mathbb{R}}} {\left\lvert {f(y-x)-f(y)} \right\rvert} \, dy = 0
\end{align*}
\todo[inline]{Missing some stuff.}

\emph{Solution omitted.}

\hypertarget{fall-2015.3}{%
\subsection{Fall 2015.3}\label{fall-2015.3}}

\begin{problem}[?]

Compute the following limit:
\begin{align*}
\lim _{n \rightarrow \infty} \int_{1}^{n} \frac{n e^{-x}}{1+n x^{2}} \, \sin \left(\frac x n\right) \, dx
\end{align*}

\end{problem}

\emph{Solution omitted.}

\hypertarget{fall-2015.4}{%
\subsection{Fall 2015.4}\label{fall-2015.4}}

Let \(f: [1, \infty) \to {\mathbb{R}}\) such that \(f(1) = 1\) and
\begin{align*}
f^{\prime}(x)= \frac{1} {x^{2}+f(x)^{2}}
\end{align*}

Show that the following limit exists and satisfies the equality
\begin{align*}
\lim _{x \rightarrow \infty} f(x) \leq 1 + \frac \pi 4
\end{align*}

\hypertarget{integrals-approximation}{%
\section{Integrals: Approximation}\label{integrals-approximation}}

\hypertarget{fall-2021.2}{%
\subsection{Fall 2021.2}\label{fall-2021.2}}

\begin{problem}[?]

\begin{enumerate}
\def\labelenumi{\alph{enumi}.}
\item
  Let \(F \subset \mathbb{R}\) be closed, and define
  \begin{align*}
  \delta_{F}(y):=\inf _{x \in F}|x-y| .
  \end{align*}
  For \(y \notin F\), show that
  \begin{align*}
  \int_{F}|x-y|^{-2} d x \leq \frac{2}{\delta_F(y)},
  \end{align*}
\item
  Let \(F \subset \mathbb{R}\) be a closed set whose complement has
  finite measure, i.e.~\(m({\mathbb{R}}\setminus F)< \infty\). Define
  the function
  \begin{align*}
  I(x):=\int_{\mathbb{R}} \frac{\delta_{F}(y)}{|x-y|^{2}} d y
  \end{align*}
  Prove that \(I(x)=\infty\) if \(x \not\in F\), however \(I(x)<\infty\)
  for almost every \(x \in F\).

  \begin{quote}
  Hint: investigate \(\int_{F} I(x) d x\).
  \end{quote}
\end{enumerate}

\end{problem}

\emph{Solution omitted.}

\emph{Solution omitted.}

\hypertarget{spring-2018.3}{%
\subsection{Spring 2018.3}\label{spring-2018.3}}

Let \(f\) be a non-negative measurable function on \([0, 1]\).

Show that
\begin{align*}
\lim _{p \rightarrow \infty}\left(\int_{[0,1]} f(x)^{p} d x\right)^{\frac{1}{p}}=\|f\|_{\infty}.
\end{align*}

\emph{Concept review omitted.}

\emph{Solution omitted.}

\hypertarget{spring-2018.4}{%
\subsection{Spring 2018.4}\label{spring-2018.4}}

Let \(f\in L^2([0, 1])\) and suppose
\begin{align*}
\int _{[0,1]} f(x) x^{n} d x=0 \text { for all integers } n \geq 0.
\end{align*}
Show that \(f = 0\) almost everywhere.

\emph{Concept review omitted.}

\emph{Solution omitted.}

\hypertarget{spring-2015.2}{%
\subsection{Spring 2015.2}\label{spring-2015.2}}

Let \(f: {\mathbb{R}}\to {\mathbb{C}}\) be continuous with period 1.
Prove that
\begin{align*}
\lim _{N \rightarrow \infty} \frac{1}{N} \sum_{n=1}^{N} f(n \alpha)=\int_{0}^{1} f(t) d t \quad \forall \alpha \in {\mathbb{R}}\setminus{\mathbb{Q}}.
\end{align*}

\begin{quote}
Hint: show this first for the functions \(f(t) = e^{2\pi i k t}\) for
\(k\in {\mathbb{Z}}\).
\end{quote}

\hypertarget{fall-2014.4}{%
\subsection{Fall 2014.4}\label{fall-2014.4}}

\begin{problem}[?]

Let \(g\in L^\infty([0, 1])\) Prove that
\begin{align*}
\int _{[0,1]} f(x) g(x)\, dx = 0 
\quad\text{for all continuous } f:[0, 1] \to {\mathbb{R}}
\implies g(x) = 0 \text{ almost everywhere. }
\end{align*}

\end{problem}

\emph{Concept review omitted.}

\emph{Solution omitted.}

\hypertarget{l1}{%
\section{\texorpdfstring{\(L^1\)}{L\^{}1}}\label{l1}}

\hypertarget{spring-2021.4}{%
\subsection{Spring 2021.4}\label{spring-2021.4}}

Let \(f, g\) be Lebesgue integrable on \({\mathbb{R}}\) and let
\(g_n(x) \coloneqq g(x- n)\). Prove that
\begin{align*}
\lim_{n\to \infty } {\left\lVert {f + g_n} \right\rVert}_1 = {\left\lVert {f} \right\rVert}_1 + {\left\lVert {g} \right\rVert}_1
.\end{align*}

\emph{Concept review omitted.}

\emph{Solution omitted.}

\hypertarget{fall-2020.4}{%
\subsection{Fall 2020.4}\label{fall-2020.4}}

\begin{problem}[?]

Prove that if \(xf(x) \in L^1({\mathbb{R}})\), then
\begin{align*}  
F(y) \coloneqq\int f(x) \cos(yx)\,  dx
\end{align*}
defines a \(C^1\) function.

\end{problem}

\emph{Solution omitted.}

\hypertarget{spring-2020.3}{%
\subsection{Spring 2020.3}\label{spring-2020.3}}

\begin{enumerate}
\def\labelenumi{\alph{enumi}.}
\item
  Prove that if \(g\in L^1({\mathbb{R}})\) then
  \begin{align*}
  \lim_{N\to \infty} \int _{{\left\lvert {x} \right\rvert} \geq N} {\left\lvert {f(x)} \right\rvert} \, dx = 0
  ,\end{align*}
  and demonstrate that it is not necessarily the case that
  \(f(x) \to 0\) as \({\left\lvert {x} \right\rvert}\to \infty\).
\item
  Prove that if \(f\in L^1([1, \infty])\) and is decreasing, then
  \(\lim_{x\to\infty}f(x) =0\) and in fact
  \(\lim_{x\to \infty} xf(x) = 0\).
\item
  If \(f: [1, \infty) \to [0, \infty)\) is decreasing with
  \(\lim_{x\to \infty} xf(x) = 0\), does this ensure that
  \(f\in L^1([1, \infty))\)?
\end{enumerate}

\emph{Concept review omitted.}

\emph{Solution omitted.}

\emph{Solution omitted.}

\emph{Solution omitted.}

\hypertarget{fall-2019.5}{%
\subsection{Fall 2019.5}\label{fall-2019.5}}

\begin{enumerate}
\def\labelenumi{\alph{enumi}.}
\item
  Show that if \(f\) is continuous with compact support on
  \({\mathbb{R}}\), then
  \begin{align*}
  \lim _{y \rightarrow 0} \int_{\mathbb{R}}|f(x-y)-f(x)| d x=0
  \end{align*}
\item
  Let \(f\in L^1({\mathbb{R}})\) and for each \(h > 0\) let
  \begin{align*}
  \mathcal{A}_{h} f(x):=\frac{1}{2 h} \int_{|y| \leq h} f(x-y) d y
  \end{align*}
\end{enumerate}

\begin{itemize}
\item
  Prove that \(\left\|\mathcal{A}_{h} f\right\|_{1} \leq\|f\|_{1}\) for
  all \(h > 0\).
\item
  Prove that \(\mathcal{A}_h f \to f\) in \(L^1({\mathbb{R}})\) as
  \(h \to 0^+\).
\end{itemize}

\todo[inline]{Walk through.}

\emph{Concept review omitted.}

\emph{Solution omitted.}

\begin{remark}

This works for arbitrary \(f\in L^1\), using approximation by continuous
functions with compact support:

\begin{itemize}
\item
  Choose \(g\in C_c^0\) such that
  \({\left\lVert {f- g} \right\rVert}_1 \to 0\).
\item
  By translation invariance,
  \({\left\lVert {\tau_h f - \tau_h g} \right\rVert}_1 \to 0\).
\item
  Write
  \begin{align*}
  {\left\lVert {\tau f - f} \right\rVert}_1 
  &= {\left\lVert {\tau_h f - g + g - \tau_h g + \tau_h g - f} \right\rVert}_1 \\
  &\leq {\left\lVert {\tau_h f - \tau_h g} \right\rVert} + {\left\lVert {g - f} \right\rVert} + {\left\lVert {\tau_h g - g} \right\rVert} \\
  &\to {\left\lVert {\tau_h g - g} \right\rVert}
  ,\end{align*}

  so it suffices to show that
  \({\left\lVert {\tau_h g - g} \right\rVert} \to 0\).
\end{itemize}

\end{remark}

\hypertarget{fall-2017.3}{%
\subsection{Fall 2017.3}\label{fall-2017.3}}

Let
\begin{align*}
S = \mathop{\mathrm{span}}_{\mathbb{C}}\left\{{\chi_{(a, b)} {~\mathrel{\Big\vert}~}a, b \in {\mathbb{R}}}\right\},
\end{align*}
the complex linear span of characteristic functions of intervals of the
form \((a, b)\).

Show that for every \(f\in L^1({\mathbb{R}})\), there exists a sequence
of functions \(\left\{{f_n}\right\} \subset S\) such that
\begin{align*}
\lim _{n \rightarrow \infty}\left\|f_{n}-f\right\|_{1}=0
\end{align*}

\emph{Concept review omitted.}

\emph{Solution omitted.}

\hypertarget{spring-2015.4}{%
\subsection{Spring 2015.4}\label{spring-2015.4}}

\begin{problem}[?]

Define
\begin{align*}
f(x, y):=\left\{\begin{array}{ll}{\frac{x^{1 / 3}}{(1+x y)^{3 / 2}}} & {\text { if } 0 \leq x \leq y} \\ {0} & {\text { otherwise }}\end{array}\right.
\end{align*}

Carefully show that \(f \in L^1({\mathbb{R}}^2)\).

\end{problem}

\emph{Solution omitted.}

\hypertarget{fall-2014.3}{%
\subsection{Fall 2014.3}\label{fall-2014.3}}

\begin{problem}[?]

Let \(f\in L^1({\mathbb{R}})\). Show that
\begin{align*}
\forall\varepsilon > 0 \exists \delta > 0 \text{ such that } \qquad 
m(E) < \delta 
\implies 
\int _{E} |f(x)| \, dx < \varepsilon
\end{align*}

\end{problem}

\emph{Solution omitted.}

\emph{Solution omitted.}

\hypertarget{spring-2014.1}{%
\subsection{Spring 2014.1}\label{spring-2014.1}}

\begin{problem}[?]

\begin{enumerate}
\def\labelenumi{\arabic{enumi}.}
\item
  Give an example of a continuous \(f\in L^1({\mathbb{R}})\) such that
  \(f(x) \not\to 0\) as\({\left\lvert {x} \right\rvert} \to \infty\).
\item
  Show that if \(f\) is \emph{uniformly} continuous, then
  \begin{align*}
  \lim_{{\left\lvert {x} \right\rvert} \to \infty} f(x) = 0.
  \end{align*}
\end{enumerate}

\end{problem}

\emph{Solution omitted.}

\hypertarget{fubini-tonelli}{%
\section{Fubini-Tonelli}\label{fubini-tonelli}}

\hypertarget{spring-2021.6}{%
\subsection{Spring 2021.6}\label{spring-2021.6}}

\begin{warnings}

This problem may be much harder than expected. Recommended skip.

\end{warnings}

Let \(f: {\mathbb{R}}\times{\mathbb{R}}\to {\mathbb{R}}\) be a
measurable function and for \(x\in {\mathbb{R}}\) define the set
\begin{align*}
E_x \coloneqq\left\{{ y\in {\mathbb{R}}{~\mathrel{\Big\vert}~}\mu\qty{ z\in {\mathbb{R}}{~\mathrel{\Big\vert}~}f(x,z) = f(x, y) } > 0 }\right\} 
.\end{align*}
Show that the following set is a measurable subset of
\({\mathbb{R}}\times{\mathbb{R}}\):
\begin{align*}
E \coloneqq\displaystyle\bigcup_{x\in {\mathbb{R}}} \left\{{ x }\right\} \times E_x
.\end{align*}

\begin{quote}
Hint: consider the measurable function
\(h(x,y,z) \coloneqq f(x, y) - f(x, z)\).
\end{quote}

\hypertarget{fall-2021.4}{%
\subsection{Fall 2021.4}\label{fall-2021.4}}

\begin{problem}[?]

Let \(f\) be a measurable function on \(\mathbb{R}\). Show that the
graph of \(f\) has measure zero in \(\mathbb{R}^{2}\).

\end{problem}

\emph{Solution omitted.}

\hypertarget{spring-2020.4}{%
\subsection{Spring 2020.4}\label{spring-2020.4}}

Let \(f, g\in L^1({\mathbb{R}})\). Argue that
\(H(x, y) \coloneqq f(y) g(x-y)\) defines a function in
\(L^1({\mathbb{R}}^2)\) and deduce from this fact that
\begin{align*}
(f\ast g)(x) \coloneqq\int_{\mathbb{R}}f(y) g(x-y) \,dy
\end{align*}
defines a function in \(L^1({\mathbb{R}})\) that satisfies
\begin{align*}
{\left\lVert {f\ast g} \right\rVert}_1 \leq {\left\lVert {f} \right\rVert}_1 {\left\lVert {g} \right\rVert}_1
.\end{align*}

\emph{Strategy omitted.}

\emph{Concept review omitted.}

\emph{Solution omitted.}

\hypertarget{spring-2019.4}{%
\subsection{Spring 2019.4}\label{spring-2019.4}}

Let \(f\) be a non-negative function on \({\mathbb{R}}^n\) and
\(\mathcal A = \{(x, t) ∈ {\mathbb{R}}^n \times {\mathbb{R}}: 0 ≤ t ≤ f (x)\}\).

Prove the validity of the following two statements:

\begin{enumerate}
\def\labelenumi{\alph{enumi}.}
\item
  \(f\) is a Lebesgue measurable function on
  \({\mathbb{R}}^n \iff \mathcal A\) is a Lebesgue measurable subset of
  \({\mathbb{R}}^{n+1}\)
\item
  If \(f\) is a Lebesgue measurable function on \({\mathbb{R}}^n\), then
  \begin{align*}
  m(\mathcal{A})=\int _{{\mathbb{R}}^{n}} f(x) d x=\int_{0}^{\infty} m\left(\left\{x \in {\mathbb{R}}^{n}: f(x) \geq t\right\}\right) dt
  \end{align*}
\end{enumerate}

\emph{Concept review omitted.}

\emph{Solution omitted.}

\hypertarget{fall-2018.5}{%
\subsection{Fall 2018.5}\label{fall-2018.5}}

Let \(f \geq 0\) be a measurable function on \({\mathbb{R}}\). Show that
\begin{align*}
\int _{{\mathbb{R}}} f = \int _{0}^{\infty} m(\{x: f(x)>t\}) dt
\end{align*}

\emph{Concept review omitted.}

\emph{Solution omitted.}

\hypertarget{fall-2015.5}{%
\subsection{Fall 2015.5}\label{fall-2015.5}}

\begin{problem}[?]

Let \(f, g \in L^1({\mathbb{R}})\) be Borel measurable.

\begin{itemize}
\tightlist
\item
  Show that

  \begin{itemize}
  \tightlist
  \item
    The function
    \begin{align*}F(x, y) \coloneqq f(x-y) g(y)\end{align*}
    is Borel measurable on \({\mathbb{R}}^2\), and
  \item
    For almost every \(x\in {\mathbb{R}}\), the function \(f(x-y)g(y)\)
    is integrable with respect to \(y\) on \({\mathbb{R}}\).
  \end{itemize}
\item
  Show that \(f\ast g \in L^1({\mathbb{R}})\) and
  \begin{align*}
  \|f * g\|_{1} \leq \|f\|_{1} \|g\|_{1}
  \end{align*}
\end{itemize}

\end{problem}

\emph{Solution omitted.}

\hypertarget{spring-2014.5}{%
\subsection{Spring 2014.5}\label{spring-2014.5}}

\begin{problem}[?]

Let \(f, g \in L^1([0, 1])\) and for all \(x\in [0, 1]\) define
\begin{align*}
F(x) \coloneqq\int _{0}^{x} f(y) \, dy 
{\quad \operatorname{and} \quad}
G(x)\coloneqq\int _{0}^{x} g(y) \, dy.
\end{align*}

Prove that
\begin{align*}
\int _{0}^{1} F(x) g(x) \, dx = 
F(1) G(1) - \int _{0}^{1} f(x) G(x) \, dx
\end{align*}

\end{problem}

\hypertarget{l2-and-fourier-analysis}{%
\section{\texorpdfstring{\(L^2\) and Fourier
Analysis}{L\^{}2 and Fourier Analysis}}\label{l2-and-fourier-analysis}}

\hypertarget{fall-2020.5}{%
\subsection{Fall 2020.5}\label{fall-2020.5}}

\begin{problem}[?]

Suppose \(\varphi\in L^1({\mathbb{R}})\) with
\begin{align*}  
\int \varphi(x) \, dx = \alpha
.\end{align*}
For each \(\delta > 0\) and \(f\in L^1({\mathbb{R}})\), define
\begin{align*}  
A_\delta f(x) \coloneqq\int f(x-y) \delta^{-1} \varphi\qty{\delta^{-1} y}\, dy
.\end{align*}

\begin{enumerate}
\def\labelenumi{\alph{enumi}.}
\item
  Prove that for all \(\delta > 0\),
  \begin{align*}  
  {\left\lVert {A_\delta f} \right\rVert}_1 \leq {\left\lVert {\varphi} \right\rVert}_1 {\left\lVert {f} \right\rVert}_1
  .\end{align*}
\item
  Prove that
  \begin{align*}  
  A_\delta f \to \alpha f \text{ in } L^1({\mathbb{R}}) {\quad \operatorname{as} \quad} \delta\to 0^+
  .\end{align*}
\end{enumerate}

\begin{quote}
Hint: you may use without proof the fact that for all
\(f\in L^1({\mathbb{R}})\),
\begin{align*}  
\lim_{y\to 0} \int_{\mathbb{R}}{\left\lvert {f(x-y) - f(x)} \right\rvert}\, dx = 0
.\end{align*}
\end{quote}

\end{problem}

\begin{remark}

See Folland 8.14.

\end{remark}

\emph{Solution omitted.}

\emph{Solution omitted.}

\hypertarget{spring-2020.6}{%
\subsection{Spring 2020.6}\label{spring-2020.6}}

\begin{problem}[?]

\envlist

\begin{enumerate}
\def\labelenumi{\alph{enumi}.}
\item
  Show that
  \begin{align*}
  L^2([0, 1]) \subseteq L^1([0, 1]) {\quad \operatorname{and} \quad} \ell^1({\mathbb{Z}}) \subseteq \ell^2({\mathbb{Z}})
  .\end{align*}
\item
  For \(f\in L^1([0, 1])\) define
  \begin{align*}
  \widehat{f}(n) \coloneqq\int _0^1 f(x) e^{-2\pi i n x} \, dx
  .\end{align*}
  Prove that if \(f\in L^1([0, 1])\) and
  \(\left\{{\widehat{f}(n)}\right\} \in \ell^1({\mathbb{Z}})\) then
  \begin{align*}
  S_N f(x) \coloneqq\sum_{{\left\lvert {n} \right\rvert} \leq N} \widehat{f} (n) e^{2 \pi i n x}
  .\end{align*}
  converges uniformly on \([0, 1]\) to a continuous function \(g\) such
  that \(g = f\) almost everywhere.
\end{enumerate}

\begin{quote}
Hint: One approach is to argue that if \(f\in L^1([0, 1])\) with
\(\left\{{\widehat{f} (n)}\right\} \in \ell^1({\mathbb{Z}})\) then
\(f\in L^2([0, 1])\).
\end{quote}

\end{problem}

\emph{Concept review omitted.}

\emph{Solution omitted.}

\emph{Solution omitted.}

\emph{Solution omitted.}

\hypertarget{fall-2017.5}{%
\subsection{Fall 2017.5}\label{fall-2017.5}}

Let \(\varphi\) be a compactly supported smooth function that vanishes
outside of an interval \([-N, N]\) such that
\(\int _{{\mathbb{R}}} \varphi(x) \, dx = 1\).

For \(f\in L^1({\mathbb{R}})\), define
\begin{align*}
K_{j}(x) \coloneqq j \varphi(j x), 
\qquad 
f \ast K_{j}(x) \coloneqq\int_{{\mathbb{R}}} f(x-y) K_{j}(y) \, dy
\end{align*}
and prove the following:

\begin{enumerate}
\def\labelenumi{\arabic{enumi}.}
\item
  Each \(f\ast K_j\) is smooth and compactly supported.
\item

  \begin{align*}
  \lim _{j \to \infty} {\left\lVert {f * K_{j}-f} \right\rVert}_{1} = 0
  \end{align*}
\end{enumerate}

\begin{quote}
Hint:
\begin{align*}
\lim _{y \to 0} \int _{{\mathbb{R}}} |f(x-y)-f(x)| dy = 0
\end{align*}
\end{quote}

\todo[inline]{Add concepts.}

\emph{Solution omitted.}

\hypertarget{spring-2017.5}{%
\subsection{Spring 2017.5}\label{spring-2017.5}}

Let \(f, g \in L^2({\mathbb{R}})\). Prove that the formula
\begin{align*}
h(x) \coloneqq\int _{-\infty}^{\infty} f(t) g(x-t) \, dt
\end{align*}
defines a uniformly continuous function \(h\) on \({\mathbb{R}}\).

\hypertarget{spring-2015.6}{%
\subsection{Spring 2015.6}\label{spring-2015.6}}

Let \(f \in L^1({\mathbb{R}})\) and \(g\) be a bounded measurable
function on \({\mathbb{R}}\).

\begin{enumerate}
\def\labelenumi{\arabic{enumi}.}
\tightlist
\item
  Show that the convolution \(f\ast g\) is well-defined, bounded, and
  uniformly continuous on \({\mathbb{R}}\).
\item
  Prove that one further assumes that \(g \in C^1({\mathbb{R}})\) with
  bounded derivative, then \(f\ast g \in C^1({\mathbb{R}})\) and
  \begin{align*}
  \frac{d}{d x}(f * g)=f *\left(\frac{d}{d x} g\right)
  \end{align*}
\end{enumerate}

\hypertarget{fall-2014.5}{%
\subsection{Fall 2014.5}\label{fall-2014.5}}

\begin{enumerate}
\def\labelenumi{\arabic{enumi}.}
\item
  Let \(f \in C_c^0({\mathbb{R}}^n)\), and show
  \begin{align*}
  \lim _{t \to 0} \int_{{\mathbb{R}}^n} |f(x+t) - f(x)| \, dx = 0
  .\end{align*}
\item
  Extend the above result to \(f\in L^1({\mathbb{R}}^n)\) and show that
  \begin{align*}
  f\in L^1({\mathbb{R}}^n), \quad g\in L^\infty({\mathbb{R}}^n) \quad
  \implies f \ast g \text{ is bounded and uniformly continuous. }
  \end{align*}
\end{enumerate}

\hypertarget{functional-analysis-general}{%
\section{Functional Analysis:
General}\label{functional-analysis-general}}

\hypertarget{fall-2019.4}{%
\subsection{Fall 2019.4}\label{fall-2019.4}}

Let \(\{u_n\}_{n=1}^∞\) be an orthonormal sequence in a Hilbert space
\(\mathcal{H}\).

\begin{enumerate}
\def\labelenumi{\alph{enumi}.}
\item
  Prove that for every \(x ∈ \mathcal H\) one has
  \begin{align*}
  \displaystyle\sum_{n=1}^{\infty}\left|\left\langle x, u_{n}\right\rangle\right|^{2} \leq\|x\|^{2}
  \end{align*}
\item
  Prove that for any sequence
  \(\{a_n\}_{n=1}^\infty \in \ell^2({\mathbb{N}})\) there exists an
  element \(x\in\mathcal H\) such that
  \begin{align*}
  a_n = {\left\langle {x},~{u_n} \right\rangle} \text{ for all } n\in {\mathbb{N}}
  \end{align*}
  and
  \begin{align*}
  {\left\lVert {x} \right\rVert}^2 = \sum_{n=1}^{\infty}\left|\left\langle x, u_{n}\right\rangle\right|^{2}
  \end{align*}
\end{enumerate}

\emph{Concept review omitted.}

\emph{Solution omitted.}

\hypertarget{spring-2019.5}{%
\subsection{Spring 2019.5}\label{spring-2019.5}}

\begin{enumerate}
\def\labelenumi{\alph{enumi}.}
\item
  Show that \(L^2([0, 1]) ⊆ L^1([0, 1])\) and argue that \(L^2([0, 1])\)
  in fact forms a dense subset of \(L^1([0, 1])\).
\item
  Let \(Λ\) be a continuous linear functional on \(L^1([0, 1])\).
\end{enumerate}

Prove the Riesz Representation Theorem for \(L^1([0, 1])\) by following
the steps below:

\begin{enumerate}
\def\labelenumi{\roman{enumi}.}
\tightlist
\item
  Establish the existence of a function \(g ∈ L^2([0, 1])\) which
  represents \(Λ\) in the sense that
  \begin{align*}
    Λ(f ) = f (x)g(x) dx \text{ for all } f ∈ L^2([0, 1]).
    \end{align*}
\end{enumerate}

\begin{quote}
Hint: You may use, without proof, the Riesz Representation Theorem for
\(L^2([0, 1])\).
\end{quote}

\begin{enumerate}
\def\labelenumi{\roman{enumi}.}
\setcounter{enumi}{1}
\tightlist
\item
  Argue that the \(g\) obtained above must in fact belong to
  \(L^∞([0, 1])\) and represent \(Λ\) in the sense that
  \begin{align*}
    \Lambda(f)=\int_{0}^{1} f(x) \overline{g(x)} d x \quad \text { for all } f \in L^{1}([0,1])
    \end{align*}
  with
  \begin{align*}
    \|g\|_{L^{\infty}([0,1])} = \|\Lambda\|_{L^{1}([0,1]) {}^{ \vee }}
    \end{align*}
\end{enumerate}

\emph{Concept review omitted.}

\emph{Solution omitted.}

\hypertarget{spring-2016.6}{%
\subsection{Spring 2016.6}\label{spring-2016.6}}

Without using the Riesz Representation Theorem, compute
\begin{align*}
\sup \left\{\left|\int_{0}^{1} f(x) e^{x} d x\right| {~\mathrel{\Big\vert}~}f \in L^{2}([0,1], m),~~ \|f\|_{2} \leq 1\right\}
\end{align*}

\hypertarget{spring-2015.5}{%
\subsection{Spring 2015.5}\label{spring-2015.5}}

Let \(\mathcal H\) be a Hilbert space.

\begin{enumerate}
\def\labelenumi{\arabic{enumi}.}
\tightlist
\item
  Let \(x\in \mathcal H\) and \(\left\{{u_n}\right\}_{n=1}^N\) be an
  orthonormal set. Prove that the best approximation to \(x\) in
  \(\mathcal H\) by an element in
  \(\mathop{\mathrm{span}}_{\mathbb{C}}\left\{{u_n}\right\}\) is given
  by
  \begin{align*}
    \widehat{x} \coloneqq\sum_{n=1}^N {\left\langle {x},~{u_n} \right\rangle}u_n.
    \end{align*}
\item
  Conclude that finite dimensional subspaces of \(\mathcal H\) are
  always closed.
\end{enumerate}

\hypertarget{fall-2015.6}{%
\subsection{Fall 2015.6}\label{fall-2015.6}}

Let \(f: [0, 1] \to {\mathbb{R}}\) be continuous. Show that
\begin{align*}
\sup \left\{\|f g\|_{1} {~\mathrel{\Big\vert}~}g \in L^{1}[0,1],~~ \|g\|_{1} \leq 1\right\}=\|f\|_{\infty}
\end{align*}

\hypertarget{fall-2014.6}{%
\subsection{Fall 2014.6}\label{fall-2014.6}}

Let \(1 \leq p,q \leq \infty\) be conjugate exponents, and show that
\begin{align*}
f \in L^p({\mathbb{R}}^n) \implies \|f\|_{p} = \sup _{\|g\|_{q}=1}\left|\int f(x) g(x) d x\right|
\end{align*}

\hypertarget{banach-and-hilbert-spaces}{%
\section{Banach and Hilbert Spaces}\label{banach-and-hilbert-spaces}}

\hypertarget{fall-2021.5}{%
\subsection{Fall 2021.5}\label{fall-2021.5}}

Consider the Hilbert space \(\mathcal{H}=L^{2}([0,1])\).

\begin{enumerate}
\def\labelenumi{\alph{enumi}.}
\item
  Prove that of \(E \subset \mathcal{H}\) is closed and convex then
  \(E\) contains an element of smallest norm.

  \begin{quote}
  Hint: Show that if
  \(\left\|f_{n}\right\|_{2} \rightarrow \min \left\{f \in E:\|f\|_{2}\right\}\)
  then \(\left\{f_{n}\right\}\) is a Cauchy sequence.
  \end{quote}
\item
  Construct a non-empty closed subset \(E \subset \mathcal{H}\) which
  does not contain an element of smallest norm.
\end{enumerate}

\hypertarget{spring-2019.1}{%
\subsection{Spring 2019.1}\label{spring-2019.1}}

Let \(C([0, 1])\) denote the space of all continuous real-valued
functions on \([0, 1]\).

\begin{enumerate}
\def\labelenumi{\alph{enumi}.}
\item
  Prove that \(C([0, 1])\) is complete under the uniform norm
  \({\left\lVert {f} \right\rVert}_u := \displaystyle\sup_{x\in [0,1]} |f (x)|\).
\item
  Prove that \(C([0, 1])\) is not complete under the
  \(L^1{\hbox{-}}\)norm
  \({\left\lVert {f} \right\rVert}_1 = \displaystyle\int_0^1 |f (x)| ~dx\).
\end{enumerate}

\todo[inline]{Add concepts.}

\emph{Solution omitted.}

\hypertarget{spring-2017.6}{%
\subsection{Spring 2017.6}\label{spring-2017.6}}

Show that the space \(C^1([a, b])\) is a Banach space when equipped with
the norm
\begin{align*}
\|f\|:=\sup _{x \in[a, b]}|f(x)|+\sup _{x \in[a, b]}\left|f^{\prime}(x)\right|.
\end{align*}

\todo[inline]{Add concepts.}

\emph{Concept review omitted.}

\emph{Solution omitted.}

\hypertarget{fall-2017.6}{%
\subsection{Fall 2017.6}\label{fall-2017.6}}

Let \(X\) be a complete metric space and define a norm
\begin{align*}
\|f\|:=\max \{|f(x)|: x \in X\}.
\end{align*}

Show that \((C^0({\mathbb{R}}), {\left\lVert {{-}} \right\rVert} )\)
(the space of continuous functions \(f: X\to {\mathbb{R}}\)) is
complete.

\todo[inline]{Add concepts.}
\todo[inline]{Shouldn't this be a supremum? The max may not exist?}
\todo[inline]{Review and clean up.}

\emph{Solution omitted.}

\hypertarget{extras}{%
\section{Extras}\label{extras}}

\begin{exercise}[?]

Compute the following limits:

\begin{itemize}
\tightlist
\item
  \(\lim_{n\to\infty} \sum_{k\geq 1} {1\over k^2} \sin^n(k)\)
\item
  \(\lim_{n\to\infty} \sum_{k\geq 1} {1\over k} e^{-k/n}\)
\end{itemize}

\end{exercise}

\emph{Solution omitted.}

\begin{exercise}[?]

Let \((\Omega,{\mathcal{B}})\) be a measurable space with a Borel
\(\sigma{\hbox{-}}\)algebra and \(\mu_n: {\mathcal{B}}\to [0, \infty]\)
be a \(\sigma{\hbox{-}}\)additive measure for each \(n\). Show that the
following map is again a \(\sigma{\hbox{-}}\)additive measure on
\({\mathcal{B}}\):
\begin{align*}
\mu(B) \coloneqq\sum_{n\geq 1} \mu_n(B)
.\end{align*}

\end{exercise}

\emph{Solution omitted.}

\hypertarget{extra-problems-measure-theory}{%
\section{Extra Problems: Measure
Theory}\label{extra-problems-measure-theory}}

\hypertarget{greatest-hits}{%
\subsection{Greatest Hits}\label{greatest-hits}}

\begin{itemize}
\item
  \(\star\): Show that for \(E\subseteq {\mathbb{R}}^n\), TFAE:

  \begin{enumerate}
  \def\labelenumi{\arabic{enumi}.}
  \tightlist
  \item
    \(E\) is measurable
  \item
    \(E = H\cup Z\) here \(H\) is \(F_\sigma\) and \(Z\) is null
  \item
    \(E = V\setminus Z'\) where \(V\in G_\delta\) and \(Z'\) is null.
  \end{enumerate}
\item
  \(\star\): Show that if \(E\subseteq {\mathbb{R}}^n\) is measurable
  then
  \(m(E) = \sup \left\{{ m(K) {~\mathrel{\Big\vert}~}K\subset E\text{ compact}}\right\}\)
  iff for all \({\varepsilon}> 0\) there exists a compact
  \(K\subseteq E\) such that \(m(K) \geq m(E) - {\varepsilon}\).
\item
  \(\star\): Show that cylinder functions are measurable, i.e.~if \(f\)
  is measurable on \({\mathbb{R}}^s\), then \(F(x, y) \coloneqq f(x)\)
  is measurable on \({\mathbb{R}}^s\times{\mathbb{R}}^t\) for any \(t\).
\item
  \(\star\): Prove that the Lebesgue integral is translation invariant,
  i.e.~if \(\tau_h(x) = x+h\) then \(\int \tau_h f = \int f\).
\item
  \(\star\): Prove that the Lebesgue integral is dilation invariant,
  i.e.~if \(f_\delta(x) = {f({x\over \delta}) \over \delta^n}\) then
  \(\int f_\delta = \int f\).
\item
  \(\star\): Prove continuity in \(L^1\), i.e.
  \begin{align*}
  f \in L^{1} \Longrightarrow \lim _{h \rightarrow 0} \int|f(x+h)-f(x)|=0
  .\end{align*}
\item
  \(\star\): Show that
  \begin{align*}f,g \in L^1 \implies f\ast g \in L^1 {\quad \operatorname{and} \quad} {\left\lVert {f\ast g} \right\rVert}_1 \leq {\left\lVert {f} \right\rVert}_1 {\left\lVert {g} \right\rVert}_1.\end{align*}
\item
  \(\star\): Show that if \(X\subseteq {\mathbb{R}}\) with
  \(\mu(X) < \infty\) then
  \begin{align*}  
  {\left\lVert {f} \right\rVert}_p \overset{p\to\infty}\to {\left\lVert {f} \right\rVert}_\infty
  .\end{align*}
\end{itemize}

\hypertarget{topology}{%
\subsection{Topology}\label{topology}}

\begin{itemize}
\tightlist
\item
  Show that every compact set is closed and bounded.
\item
  Show that if a subset of a metric space is complete and totally
  bounded, then it is compact.
\item
  Show that if \(K\) is compact and \(F\) is closed with \(K, F\)
  disjoint then \(\operatorname{dist}(K, F) > 0\).
\end{itemize}

\hypertarget{continuity}{%
\subsection{Continuity}\label{continuity}}

\begin{itemize}
\tightlist
\item
  Show that a continuous function on a compact set is uniformly
  continuous.
\end{itemize}

\hypertarget{differentiation}{%
\subsection{Differentiation}\label{differentiation}}

\begin{itemize}
\tightlist
\item
  Show that if \(f\in C^1({\mathbb{R}})\) and both
  \(\lim_{x\to \infty} f(x)\) and \(\lim_{x\to \infty} f'(x)\) exist,
  then \(\lim_{x\to\infty} f'(x)\) must be zero.
\end{itemize}

\hypertarget{advanced-limitology}{%
\subsection{Advanced Limitology}\label{advanced-limitology}}

\begin{itemize}
\tightlist
\item
  If \(f\) is continuous, is it necessarily the case that \(f'\) is
  continuous?
\item
  If \(f_n \to f\), is it necessarily the case that \(f_n'\) converges
  to \(f'\) (or at all)?
\item
  Is it true that the sum of differentiable functions is differentiable?
\item
  Is it true that the limit of integrals equals the integral of the
  limit?
\item
  Is it true that a limit of continuous functions is continuous?
\item
  Show that a subset of a metric space is closed iff it is complete.
\item
  Show that if \(m(E) < \infty\) and \(f_n\to f\) uniformly, then
  \(\lim \int_E f_n = \int_E f\).
\end{itemize}

\hypertarget{uniform-convergence}{%
\subsection{Uniform Convergence}\label{uniform-convergence}}

\begin{itemize}
\tightlist
\item
  Show that a uniform limit of bounded functions is bounded.
\item
  Show that a uniform limit of continuous function is continuous.

  \begin{itemize}
  \tightlist
  \item
    I.e. if \(f_n\to f\) uniformly with each \(f_n\) continuous then
    \(f\) is continuous.
  \end{itemize}
\item
  Show that

  \begin{itemize}
  \tightlist
  \item
    \(f_n: [a, b]\to {\mathbb{R}}\) are continuously differentiable with
    derivatives \(f_n'\)
  \item
    The sequence of derivatives \(f_n'\) converges uniformly to some
    function \(g\)
  \item
    There exists \emph{at least one} point \(x_0\) such that
    \(\lim_n f_n(x_0)\) exists,
  \item
    Then \(f_n \to f\) uniformly to some differentiable \(f\), and
    \(f' = g\).
  \end{itemize}
\item
  Prove that uniform convergence implies pointwise convergence implies
  a.e. convergence, but none of the implications may be reversed.
\item
  Show that \(\sum {x^n \over n!}\) converges uniformly on any compact
  subset of \({\mathbb{R}}\).
\end{itemize}

\hypertarget{measure-theory}{%
\subsection{Measure Theory}\label{measure-theory}}

\begin{itemize}
\item
  Show that continuity of measure from above/below holds for outer
  measures.
\item
  Show that a countable union of null sets is null.
\end{itemize}

Measurability

\begin{itemize}
\tightlist
\item
  Show that \(f=0\) a.e. iff \(\int_E f = 0\) for every measurable set
  \(E\).
\end{itemize}

Integrability

\begin{itemize}
\tightlist
\item
  Show that if \(f\) is a measurable function, then \(f=0\) a.e. iff
  \(\int f = 0\).
\item
  Show that a bounded function is Lebesgue integrable iff it is
  measurable.
\item
  Show that simple functions are dense in \(L^1\).
\item
  Show that step functions are dense in \(L^1\).
\item
  Show that smooth compactly supported functions are dense in \(L^1\).
\end{itemize}

\hypertarget{convergence}{%
\subsection{Convergence}\label{convergence}}

\begin{itemize}
\tightlist
\item
  Prove Fatou's lemma using the Monotone Convergence Theorem.
\item
  Show that if \(\left\{{f_n}\right\}\) is in \(L^1\) and
  \(\sum \int {\left\lvert {f_n} \right\rvert} < \infty\) then
  \(\sum f_n\) converges to an \(L^1\) function and
  \begin{align*}\int \sum f_n = \sum \int f_n.\end{align*}
\end{itemize}

\hypertarget{convolution}{%
\subsection{Convolution}\label{convolution}}

\begin{itemize}
\tightlist
\item
  Show that if \(f, g\) are continuous and compactly supported, then so
  is \(f\ast g\).
\item
  Show that if \(f\in L^1\) and \(g\) is bounded, then \(f\ast g\) is
  bounded and uniformly continuous.
\item
  If \(f, g\) are compactly supported, is it necessarily the case that
  \(f\ast g\) is compactly supported?
\item
  Show that under any of the following assumptions, \(f\ast g\) vanishes
  at infinity:

  \begin{itemize}
  \tightlist
  \item
    \(f, g\in L^1\) are both bounded.
  \item
    \(f, g\in L^1\) with just \(g\) bounded.
  \item
    \(f, g\) smooth and compactly supported (and in fact \(f\ast g\) is
    smooth)
  \item
    \(f\in L^1\) and \(g\) smooth and compactly supported (and in fact
    \(f\ast g\) is smooth)
  \end{itemize}
\item
  Show that if \(f\in L^1\) and \(g'\) exists with
  \({\frac{\partial g}{\partial x_i}\,}\) all bounded, then
  \begin{align*}{\frac{\partial }{\partial x_i}\,}(f\ast g) = f \ast {\frac{\partial g}{\partial x_i}\,}\end{align*}
\end{itemize}

\hypertarget{fourier-analysis}{%
\subsection{Fourier Analysis}\label{fourier-analysis}}

\begin{itemize}
\tightlist
\item
  Show that if \(f\in L^1\) then \(\widehat{f}\) is bounded and
  uniformly continuous.
\item
  Is it the case that \(f\in L^1\) implies \(\widehat{f}\in L^1\)?
\item
  Show that if \(f, \widehat{f} \in L^1\) then \(f\) is bounded,
  uniformly continuous, and vanishes at infinity.

  \begin{itemize}
  \tightlist
  \item
    Show that this is not true for arbitrary \(L^1\) functions.
  \end{itemize}
\item
  Show that if \(f\in L^1\) and \(\widehat{f} = 0\) almost everywhere
  then \(f = 0\) almost everywhere.

  \begin{itemize}
  \tightlist
  \item
    Prove that \(\widehat{f} = \widehat{g}\) implies that \(f=g\) a.e.
  \end{itemize}
\item
  Show that if \(f, g \in L^1\) then
  \begin{align*}\int \widehat{f} g = \int f\widehat{g}.\end{align*}

  \begin{itemize}
  \tightlist
  \item
    Give an example showing that this fails if \(g\) is not bounded.
  \end{itemize}
\item
  Show that if \(f\in C^1\) then \(f\) is equal to its Fourier
  \emph{series}.
\end{itemize}

\hypertarget{approximate-identities}{%
\subsection{Approximate Identities}\label{approximate-identities}}

\begin{itemize}
\tightlist
\item
  Show that if \(\phi\) is an approximate identity, then
  \begin{align*}{\left\lVert {f\ast \phi_t - f} \right\rVert}_1 \overset{t\to 0}\to 0.\end{align*}

  \begin{itemize}
  \tightlist
  \item
    Show that if additionally
    \({\left\lvert {\phi(x)} \right\rvert} \leq c(1 + {\left\lvert {x} \right\rvert})^{-n-{\varepsilon}}\)
    for some \(c,{\varepsilon}>0\), then this converges is almost
    everywhere.
  \end{itemize}
\item
  Show that is \(f\) is bounded and uniformly continuous and \(\phi_t\)
  is an approximation to the identity, then \(f\ast \phi_t\) uniformly
  converges to \(f\).
\end{itemize}

\(L^p\) Spaces

\begin{itemize}
\tightlist
\item
  Show that if \(E\subseteq {\mathbb{R}}^n\) is measurable with
  \(\mu(E) < \infty\) and \(f\in L^p(X)\) then
  \begin{align*}{\left\lVert {f} \right\rVert}_{L^p(X)} \overset{p\to\infty}\to {\left\lVert {f} \right\rVert}_\infty.\end{align*}
\item
  Is it true that the converse to the DCT holds? I.e. if
  \(\int f_n \to \int f\), is there a \(g\in L^p\) such that \(f_n < g\)
  a.e. for every \(n\)?
\item
  Prove continuity in \(L^p\): If \(f\) is uniformly continuous then for
  all \(p\),
  \begin{align*}{\left\lVert {\tau_h f - f} \right\rVert}_p \overset{h\to 0}\to 0.\end{align*}
\item
  Prove the following inclusions of \(L^p\) spaces for
  \(m(X) < \infty\):
  \begin{align*}
  L^\infty(X) &\subset L^2(X) \subset L^1(X) \\
  \ell^2({\mathbb{Z}}) &\subset \ell^1({\mathbb{Z}}) \subset \ell^\infty({\mathbb{Z}})
  .\end{align*}
\end{itemize}

\hypertarget{unsorted}{%
\subsection{Unsorted}\label{unsorted}}

\begin{proposition}[Volumes of Rectangles]

If \(\left\{{R_j}\right\} \rightrightarrows R\) is a covering of \(R\)
by rectangles,
\begin{align*}
R = \overset{\circ}{\displaystyle\coprod_{j}} R_j &\implies {\left\lvert {R} \right\rvert} = \sum {\left\lvert {R} \right\rvert}_j \\
R \subseteq \displaystyle\bigcup_j R_j &\implies {\left\lvert {R} \right\rvert} \leq \sum {\left\lvert {R} \right\rvert}_j
.\end{align*}

\end{proposition}

\begin{itemize}
\tightlist
\item
  Show that any disjoint intervals is countable.
\item
  Show that every open \(U \subseteq {\mathbb{R}}\) is a countable union
  of disjoint open intervals.
\item
  Show that every open \(U \subseteq {\mathbb{R}}^n\) is a countable
  union of \emph{almost} disjoint closed cubes.
\item
  Show that that Cantor middle-thirds set is compact, totally
  disconnected, and perfect, with outer measure zero.
\item
  Prove the Borel-Cantelli lemma.
\end{itemize}

\hypertarget{extra-problems-from-problem-sets}{%
\section{Extra Problems from Problem
Sets}\label{extra-problems-from-problem-sets}}

\hypertarget{continuous-on-compact-implies-uniformly-continuous}{%
\subsection{Continuous on compact implies uniformly
continuous}\label{continuous-on-compact-implies-uniformly-continuous}}

\begin{problem}[?]

Show that a continuous function on a compact set is uniformly
continuous.

\end{problem}

\emph{Solution omitted.}

\hypertarget{section}{%
\subsection{2010 6.1}\label{section}}

\begin{problem}[?]

Show that
\begin{align*}
\int_{{\mathbb{B}}^n} {1 \over {\left\lvert {x} \right\rvert}^p } \,dx&< \infty \iff p < n \\ \\ \\ 
\int_{{\mathbb{R}}^n\setminus{\mathbb{B}}^n} {1 \over {\left\lvert {x} \right\rvert}^p } \,dx&< \infty \iff p > n 
.\end{align*}

\end{problem}

\emph{Solution omitted.}

\hypertarget{section-1}{%
\subsection{2010 6.2}\label{section-1}}

Show that
\begin{align*}
\int_{{\mathbb{R}}^n} {\left\lvert { f} \right\rvert} = \int_0^{\infty } m(A_t)\,dt&& A_t \coloneqq\left\{{x\in {\mathbb{R}}^n {~\mathrel{\Big\vert}~}{\left\lvert {f(x)} \right\rvert} > t}\right\}
.\end{align*}

\emph{Solution omitted.}

\hypertarget{section-2}{%
\subsection{2010 6.5}\label{section-2}}

Suppose \(F \subseteq {\mathbb{R}}\) with \(m(F^c) < \infty\) and let
\(\delta(x) \coloneqq d(x, F)\) and
\begin{align*}
I_F(x) \coloneqq\int_{\mathbb{R}}{ \delta(y) \over {\left\lvert {x-y} \right\rvert}^2 } \,dy
.\end{align*}

\begin{enumerate}
\def\labelenumi{\alph{enumi}.}
\item
  Show that \(\delta\) is continuous.
\item
  Show that if \(x\in F^c\) then \(I_F(x) = \infty\).
\item
  Show that \(I_F(x) < \infty\) for almost every \(x\)
\end{enumerate}

\emph{Solution omitted.}

\hypertarget{section-3}{%
\subsection{2010 7.1}\label{section-3}}

Let \((X, \mathcal{M}, \mu)\) be a measure space and prove the following
properties of \(L^ \infty (X, \mathcal{M}, \mu)\):

\begin{itemize}
\item
  If \(f, g\) are measurable on \(X\) then
  \begin{align*}
  {\left\lVert {fg} \right\rVert}_1 \leq {\left\lVert {f} \right\rVert}_1 {\left\lVert {g} \right\rVert}_{\infty }
  .\end{align*}
\item
  \({\left\lVert {{-}} \right\rVert}_{\infty }\) is a norm on
  \(L^{\infty }\) making it a Banach space.
\item
  \({\left\lVert {f_n - f} \right\rVert}_{\infty } \overset{n\to \infty }\to 0 \iff\)
  there exists an \(E\in \mathcal{M}\) such that
  \(\mu(X\setminus E) = 0\) and \(f_n \to f\) uniformly on \(E\).
\item
  Simple functions are dense in \(L^{\infty }\).
\end{itemize}

\hypertarget{section-4}{%
\subsection{2010 7.2}\label{section-4}}

Show that for \(0 < p < q \leq \infty\),
\({\left\lVert {a} \right\rVert}_{\ell^q} \leq {\left\lVert {a} \right\rVert}_{\ell^p}\)
over \({\mathbb{C}}\), where
\({\left\lVert {a} \right\rVert}_{\infty } \coloneqq\sup_j {\left\lvert {a_j} \right\rvert}\).

\hypertarget{section-5}{%
\subsection{2010 7.3}\label{section-5}}

Let \(f, g\) be non-negative measurable functions on \([0, \infty)\)
with
\begin{align*}
A &\coloneqq\int_0^{\infty } f(y) y^{-1/2} \,dy< \infty \\
B &\coloneqq\qty{ \int_0^{\infty } {\left\lvert { g(y) } \right\rvert} }^2 \,dy< \infty  
.\end{align*}

Show that
\begin{align*}
\int_0^{\infty } \qty{ \int_0^{\infty } f(y) \,dy} {g(x) \over x} \,dx\leq AB
.\end{align*}

\hypertarget{section-6}{%
\subsection{2010 7.4}\label{section-6}}

Let \((X, \mathcal{M}, \mu)\) be a measure space and
\(0 < p < q< \infty\). Prove that if \(L^q(X) \subseteq L^p(X)\), then
\(X\) does not contain sets of arbitrarily large finite measure.

\hypertarget{section-7}{%
\subsection{2010 7.5}\label{section-7}}

Suppose \(0 < a < b \leq \infty\), and find examples of functions
\(f \in L^p((0, \infty ))\) if and only if:

\begin{itemize}
\item
  \(a < p < b\)
\item
  \(a \leq p \leq b\)
\item
  \(p = a\)
\end{itemize}

\emph{Hint: consider functions of the following form:}
\begin{align*}
f(x) \coloneqq x^{- \alpha} {\left\lvert { \log(x) } \right\rvert}^{ \beta}
.\end{align*}

\hypertarget{section-8}{%
\subsection{2010 7.6}\label{section-8}}

Define
\begin{align*}
F(x) &\coloneqq\qty{ \sin(\pi x) \over \pi x}^2 \\
G(x) &\coloneqq
\begin{cases}
1 - {\left\lvert {x} \right\rvert} & {\left\lvert {x} \right\rvert} \leq 1
\\
0 & \text{else}.
\end{cases}
\end{align*}

\begin{enumerate}
\def\labelenumi{\alph{enumi}.}
\item
  Show that \(\widehat{G}(\xi) = F(\xi)\)
\item
  Compute \(\widehat{F}\).
\item
  Give an example of a function \(g\not \in L^1({\mathbb{R}})\) which is
  the Fourier transform of an \(L^1\) function.
\end{enumerate}

\emph{Hint: write \(\widehat{G}(\xi) = H(\xi) + H(-\xi)\) where}
\begin{align*}
H(\xi) \coloneqq e^{2\pi i \xi} \int_0^1 y e^{2\pi i y \xi }\,dy
.\end{align*}

\hypertarget{section-9}{%
\subsection{2010 7.7}\label{section-9}}

Show that for each \(\epsilon>0\) the following function is the Fourier
transform of an \(L^1({\mathbb{R}}^n)\) function:
\begin{align*}
F(\xi) \coloneqq\qty{1 \over 1 + {\left\lvert {\xi} \right\rvert}^2}^{\epsilon}
.\end{align*}

\emph{Hint: show that}

\begin{align*}
K_\delta(x) &\coloneqq\delta^{-n/2} e^{-\pi {\left\lvert {x} \right\rvert}^2 \over \delta} \\
f(x) &\coloneqq\int_0^{\infty } K_{\delta}(x) e^{-\pi \delta} \delta^{\epsilon - 1} \,d \delta \\
\Gamma(s) &\coloneqq\int_0^{\infty } e^{-t} t^{s-1} \,dt\\
\implies \widehat{f}(\xi) &= \int_0^{\infty } e^{- \pi \delta {\left\lvert {\xi} \right\rvert}^2} e^{ -\pi \delta} \delta^{\epsilon - 1}
= \pi^{-s} \Gamma(\epsilon) F(\xi)
.\end{align*}

\hypertarget{challenge-1-generalized-holder}{%
\subsection{2010 7 Challenge 1: Generalized
Holder}\label{challenge-1-generalized-holder}}

Suppose that
\begin{align*}
1\leq p_j \leq \infty, && \sum_{j=1}^n {1\over p_j} = {1\over r} \leq 1
.\end{align*}

Show that if \(f_j \in L^{p_j}\) for each \(1\leq j \leq n\), then
\begin{align*}
\prod f_j \in L^r, && {\left\lVert { \prod f_j } \right\rVert}_r \leq \prod {\left\lVert {f_j} \right\rVert}_{p_j}
.\end{align*}

\hypertarget{challenge-2-youngs-inequality}{%
\subsection{2010 7 Challenge 2: Young's
Inequality}\label{challenge-2-youngs-inequality}}

Suppose \(1\leq p,q,r \leq \infty\) with
\begin{align*}
{1\over p } + {1 \over q} = 1 + {1 \over r}
.\end{align*}

Prove that
\begin{align*}
f \in L^p, g\in L^q \implies f \ast g \in L^r \text{ and } {\left\lVert {f \ast g} \right\rVert}_r \leq {\left\lVert {f} \right\rVert}_p {\left\lVert {g} \right\rVert}_q
.\end{align*}

\hypertarget{section-10}{%
\subsection{2010 9.1}\label{section-10}}

Show that the set
\(\left\{{ u_k(j) \coloneqq\delta_{ij} }\right\} \subseteq \ell^2({\mathbb{Z}})\)
and forms an orthonormal system.

\hypertarget{section-11}{%
\subsection{2010 9.2}\label{section-11}}

Consider \(L^2([0, 1])\) and define
\begin{align*}
e_0(x) &= 1 \\
e_1(x) &= \sqrt{3}(2x-1)
.\end{align*}

\begin{enumerate}
\def\labelenumi{\alph{enumi}.}
\item
  Show that \(\left\{{e_0, e_1}\right\}\) is an orthonormal system.
\item
  Show that the polynomial \(p(x)\) where \(\deg(p) = 1\) which is
  closest to \(f(x) = x^2\) in \(L^2([0, 1])\) is given by
  \begin{align*}
  h(x) = x - {1\over 6}
  .\end{align*}
\end{enumerate}

Compute \({\left\lVert {f - g} \right\rVert}_2\).

\hypertarget{section-12}{%
\subsection{2010 9.3}\label{section-12}}

Let \(E \subseteq H\) a Hilbert space.

\begin{enumerate}
\def\labelenumi{\alph{enumi}.}
\item
  Show that \(E\perp \subseteq H\) is a closed subspace.
\item
  Show that \((E^\perp)^\perp = { \operatorname{cl}} _H(E)\).
\end{enumerate}

\hypertarget{b}{%
\subsection{2010 9.5b}\label{b}}

Let \(f\in L^1((0, 2\pi))\).

\begin{enumerate}
\def\labelenumi{\roman{enumi}.}
\tightlist
\item
  Show that for an \(\epsilon>0\) one can write \(f = g+h\) where
  \(g\in L^2((0, 2\pi))\) and
  \({\left\lVert {H} \right\rVert}_1 < \epsilon\).
\end{enumerate}

\hypertarget{section-13}{%
\subsection{2010 9.6}\label{section-13}}

Prove that every closed convex \(K \subset H\) a Hilbert space has a
unique element of minimal norm.

\hypertarget{challenge}{%
\subsection{2010 9 Challenge}\label{challenge}}

Let \(U\) be a unitary operator on \(H\) a Hilbert space, let
\(M \coloneqq\left\{{x\in H {~\mathrel{\Big\vert}~}Ux = x}\right\}\),
let \(P\) be the orthogonal projection onto \(M\), and define
\begin{align*}
S_N \coloneqq{1\over N} \sum_{n=0}^{N-1} U^n
.\end{align*}
Show that for all \(x\in H\),
\begin{align*}
{\left\lVert { S_N x - Px} \right\rVert}_H \overset{N\to \infty } \to 0
.\end{align*}

\hypertarget{section-14}{%
\subsection{2010 10.1}\label{section-14}}

Let \(\nu, \mu\) be signed measures, and show that
\begin{align*}
\nu \perp \mu \text{ and } \nu \ll {\left\lvert { \mu} \right\rvert} \implies \nu = 0
.\end{align*}

\hypertarget{section-15}{%
\subsection{2010 10.2}\label{section-15}}

Let \(f\in L^1({\mathbb{R}}^n)\) with \(f\neq 0\).

\begin{enumerate}
\def\labelenumi{\alph{enumi}.}
\tightlist
\item
  Prove that there exists a \(c>0\) such that
  \begin{align*}
  Hf(x) \geq {c \over (1 + {\left\lvert {x} \right\rvert})^n }
  .\end{align*}
\end{enumerate}

\hypertarget{section-16}{%
\subsection{2010 10.3}\label{section-16}}

Consider the function
\begin{align*}
f(x) \coloneqq
\begin{cases}
{1\over {\left\lvert {x} \right\rvert} \qty{ \log\qty{1\over x}}^2 } &  {\left\lvert {x} \right\rvert} \leq {1\over 2}
\\
0 & \text{else}.
\end{cases}
\end{align*}

\begin{enumerate}
\def\labelenumi{\alph{enumi}.}
\item
  Show that \(f \in L^1({\mathbb{R}})\).
\item
  Show that there exists a \(c>0\) such that for all
  \({\left\lvert {x} \right\rvert} \leq 1/2\),
  \begin{align*}
  Hf(x) \geq {c \over {\left\lvert {x} \right\rvert} \log\qty{1\over {\left\lvert {x} \right\rvert}} }
  .\end{align*}
  Conclude that \(Hf\) is not locally integrable.
\end{enumerate}

\hypertarget{section-17}{%
\subsection{2010 10.4}\label{section-17}}

Let \(f\in L^1({\mathbb{R}})\) and let
\(\mathcal{U}\coloneqq\left\{{(x, y) \in {\mathbb{R}}^2 {~\mathrel{\Big\vert}~}y > 0}\right\}\)
denote the upper half plane. For \((x, y) \in \mathcal{U}\) define
\begin{align*}
u(x, y) \coloneqq f \ast P_y(x) && \text{where } P_y(x) \coloneqq{1\over \pi}\qty{y \over t^2 + y^2}
.\end{align*}

\begin{enumerate}
\def\labelenumi{\alph{enumi}.}
\item
  Prove that there exists a constant \(C\) independent of \(f\) such
  that for all \(x\in {\mathbb{R}}\),
  \begin{align*}
  \sup_{y > 0} {\left\lvert { u(x, y) } \right\rvert} \leq C\cdot Hf(x)
  .\end{align*}

  \emph{Hint: write the following and try to estimate each term:}
  \begin{align*}
  u(x, y) = \int_{{\left\lvert {t} \right\rvert} < y} f(x - t) P_y(t) \,dt+ \sum_{k=0}^{\infty } \int_{A_k} f(x-t) P_y(t)\,dt&& A_k \coloneqq\left\{{2^ky \leq {\left\lvert {t} \right\rvert} < 2^{k+1}y}\right\}
  .\end{align*}
\item
  Following the proof of the Lebesgue differentiation theorem, show that
  for \(f\in L^1({\mathbb{R}})\) and for almost every
  \(x\in {\mathbb{R}}\),
  \begin{align*}
  u(x, y) \overset{y\to 0} \to f(x)
  .\end{align*}
\end{enumerate}

\hypertarget{midterm-exam-2-december-2014}{%
\section{Midterm Exam 2 (December
2014)}\label{midterm-exam-2-december-2014}}

\hypertarget{fall-2014-midterm-1.1}{%
\subsection{Fall 2014 Midterm 1.1}\label{fall-2014-midterm-1.1}}

\begin{quote}
Note: (a) is a repeat.
\end{quote}

\begin{itemize}
\tightlist
\item
  Let \(\Lambda\in L^2(X) {}^{ \vee }\).

  \begin{itemize}
  \tightlist
  \item
    Show that
    \(M\coloneqq\left\{{f\in L^2(X) {~\mathrel{\Big\vert}~}\Lambda(f) = 0}\right\} \subseteq L^2(X)\)
    is a closed subspace, and \(L^2(X) = M \oplus M\perp\).
  \item
    Prove that there exists a unique \(g\in L^2(X)\) such that
    \(\Lambda(f) = \int_X g \mkern 1.5mu\overline{\mkern-1.5muf\mkern-1.5mu}\mkern 1.5mu\).
  \end{itemize}
\end{itemize}

\hypertarget{fall-2014-midterm-1.2}{%
\subsection{Fall 2014 Midterm 1.2}\label{fall-2014-midterm-1.2}}

\begin{enumerate}
\def\labelenumi{\alph{enumi}.}
\tightlist
\item
  In parts:
\end{enumerate}

\begin{itemize}
\tightlist
\item
  Given a definition of \(L^\infty({\mathbb{R}}^n)\).
\item
  Verify that \({\left\lVert {{-}} \right\rVert}_\infty\) defines a norm
  on \(L^\infty({\mathbb{R}}^n)\).
\item
  Carefully proved that
  \((L^\infty({\mathbb{R}}^n), {\left\lVert {{-}} \right\rVert}_\infty)\)
  is a Banach space.
\end{itemize}

\begin{enumerate}
\def\labelenumi{\alph{enumi}.}
\setcounter{enumi}{1}
\tightlist
\item
  Prove that for any measurable \(f:{\mathbb{R}}^n \to {\mathbb{C}}\),
  \begin{align*}
  L^1({\mathbb{R}}^n) \cap L^\infty({\mathbb{R}}^n) \subset L^2({\mathbb{R}}^n) {\quad \operatorname{and} \quad} {\left\lVert {f} \right\rVert}_2 \leq {\left\lVert {f} \right\rVert}_1^{1\over 2} \cdot {\left\lVert {f} \right\rVert}_\infty^{1\over 2}
  .\end{align*}
\end{enumerate}

\hypertarget{fall-2014-midterm-1.3}{%
\subsection{Fall 2014 Midterm 1.3}\label{fall-2014-midterm-1.3}}

\begin{enumerate}
\def\labelenumi{\alph{enumi}.}
\item
  Prove that if \(f, g: {\mathbb{R}}^n\to {\mathbb{C}}\) is both
  measurable then \(F(x, y) \coloneqq f(x)\) and
  \(h(x, y)\coloneqq f(x-y) g(y)\) is measurable on
  \({\mathbb{R}}^n\times{\mathbb{R}}^n\).
\item
  Show that if
  \(f\in L^1({\mathbb{R}}^n) \cap L^\infty({\mathbb{R}}^n)\) and
  \(g\in L^1({\mathbb{R}}^n)\), then
  \(f\ast g \in L^1({\mathbb{R}}^n) \cap L^\infty({\mathbb{R}}^n)\) is
  well defined, and carefully show that it satisfies the following
  properties:
  \begin{align*}
  {\left\lVert {f\ast g} \right\rVert}_\infty &\leq {\left\lVert {g} \right\rVert}_1 {\left\lVert {f} \right\rVert}_\infty
  {\left\lVert {f\ast g} \right\rVert}_1      &\leq {\left\lVert {g} \right\rVert}_1 {\left\lVert {f} \right\rVert}_1
  {\left\lVert {f\ast g} \right\rVert}_2      &\leq {\left\lVert {g} \right\rVert}_1 {\left\lVert {f} \right\rVert}_2
  .\end{align*}
\end{enumerate}

\begin{quote}
Hint: first show
\({\left\lvert {f\ast g} \right\rvert}^2 \leq {\left\lVert {g} \right\rVert}_1 \qty{ {\left\lvert {f} \right\rvert}^2 \ast {\left\lvert {g} \right\rvert}}\).
\end{quote}

\hypertarget{fall-2014-midterm-1.4}{%
\subsection{Fall 2014 Midterm 1.4}\label{fall-2014-midterm-1.4}}

\begin{quote}
Note: (a) is a repeat.
\end{quote}

Let \(f: [0, 1]\to {\mathbb{R}}\) be continuous, and prove the
Weierstrass approximation theorem: for any \({\varepsilon}> 0\) there
exists a polynomial \(P\) such that
\({\left\lVert {f - P} \right\rVert}_{\infty} < {\varepsilon}\).

\hypertarget{midterm-exam-1-october-2018}{%
\section{Midterm Exam 1 (October
2018)}\label{midterm-exam-1-october-2018}}

\hypertarget{fall-2018-midterm-1.1}{%
\subsection{Fall 2018 Midterm 1.1}\label{fall-2018-midterm-1.1}}

\label{equivalence_of_approximating_measures} Let
\(E \subseteq {\mathbb{R}}^n\) be bounded. Prove the following are
equivalent:

\begin{enumerate}
\def\labelenumi{\arabic{enumi}.}
\item
  For any \(\epsilon>0\) there exists and open set \(G\) and a closed
  set \(F\) such that
  \begin{align*}
  F \subseteq E \subseteq G && m(G\setminus F) < \epsilon
  .\end{align*}
\item
  There exists a \(G_ \delta\) set \(V\) and an \(F_ \sigma\) set \(H\)
  such that
  \begin{align*}
  m(V\setminus H) = 0
  .\end{align*}
\end{enumerate}

\hypertarget{fall-2018-midterm-1.2}{%
\subsection{Fall 2018 Midterm 1.2}\label{fall-2018-midterm-1.2}}

Let \(\left\{{ f_k }\right\} _{k=1}^{\infty }\) be a sequence of
extended real-valued Lebesgue measurable functions.

\begin{enumerate}
\def\labelenumi{\alph{enumi}.}
\item
  Prove that \(\sup_k f_k\) is a Lebesgue measurable function.
\item
  Prove that if \(\lim_{k \to \infty } f_k(x)\) exists for every
  \(x \in {\mathbb{R}}^n\) then \(\lim_{k\to \infty } f_k\) is also a
  measurable function.
\end{enumerate}

\hypertarget{fall-2018-midterm-1.3}{%
\subsection{Fall 2018 Midterm 1.3}\label{fall-2018-midterm-1.3}}

\begin{enumerate}
\def\labelenumi{\alph{enumi}.}
\item
  Prove that if \(E \subseteq {\mathbb{R}}^n\) is a Lebesgue measurable
  set, then for any \(h \in {\mathbb{R}}\) the set
  \begin{align*}
  E+h \coloneqq\left\{{x + h {~\mathrel{\Big\vert}~}x\in E }\right\}
  \end{align*}
  is also Lebesgue measurable and satisfies \(m(E + h) = m(E)\).
\item
  Prove that if \(f\) is a non-negative measurable function on
  \({\mathbb{R}}^n\) and \(h\in {\mathbb{R}}^n\) then the function
  \begin{align*}
  \tau_h d(x) \coloneqq f(x-h)
  \end{align*}
  is a non-negative measurable function and
  \begin{align*}
  \int f(x) \,dx= \int f(x-h) \,dx
  .\end{align*}
\end{enumerate}

\hypertarget{fall-2018-midterm-1.4}{%
\subsection{Fall 2018 Midterm 1.4}\label{fall-2018-midterm-1.4}}

Let \(f: {\mathbb{R}}^n\to {\mathbb{R}}\) be a Lebesgue measurable
function.

\begin{enumerate}
\def\labelenumi{\alph{enumi}.}
\item
  Prove that for all \(\alpha> 0\) ,
  \begin{align*}
  A_ \alpha  \coloneqq\left\{{x\in {\mathbb{R}}^n {~\mathrel{\Big\vert}~}{\left\lvert { f(x) } \right\rvert} > \alpha}\right\} \implies m(A_ \alpha) \leq {1\over \alpha} \int {\left\lvert {f (x)} \right\rvert} \,dx
  .\end{align*}
\item
  Prove that
  \begin{align*}
  \int {\left\lvert { f(x) } \right\rvert} \,dx= 0 \iff f = 0 \text{ almost everywhere}
  .\end{align*}
\end{enumerate}

\hypertarget{fall-2018-midterm-1.5}{%
\subsection{Fall 2018 Midterm 1.5}\label{fall-2018-midterm-1.5}}

Let \(\left\{{ f_k }\right\}_{k=1}^{\infty } \subseteq L^2([0, 1])\) be
a sequence which \emph{converges in \(L^1\)} to a function \(f\).

\begin{enumerate}
\def\labelenumi{\alph{enumi}.}
\item
  Prove that \(f\in L^1([0, 1])\).
\item
  Give an example illustrating that \(f_k\) may not converge to \(f\)
  almost everywhere.
\item
  Prove that \(\left\{{f_k}\right\}\) must contain a subsequence that
  converges to \(f\) almost everywhere.
\end{enumerate}

\hypertarget{midterm-exam-2-november-2018}{%
\section{Midterm Exam 2 (November
2018)}\label{midterm-exam-2-november-2018}}

\hypertarget{fall-2018-midterm-2.1}{%
\subsection{Fall 2018 Midterm 2.1}\label{fall-2018-midterm-2.1}}

Let \(f, g\in L^1([0, 1])\), define \(F(x) = \int_0^x f(y)\,dy\) and
\(G(x) = \int_0^x g(y)\,dy\), and show
\begin{align*}
\int_0^1 F(x)g(x) \,dx = F(1)G(1) - \int_0^1 f(x) G(x) \, dx
.\end{align*}

\hypertarget{fall-2018-midterm-2.2}{%
\subsection{Fall 2018 Midterm 2.2}\label{fall-2018-midterm-2.2}}

Let \(\phi\in L^1({\mathbb{R}}^n)\) such that \(\int \phi = 1\) and
define \(\phi_t(x) = t^{-n}\phi(t^{-1}x)\). Show that if \(f\) is
bounded and uniformly continuous then
\(f\ast \phi_t \overset{t\to 0}\to f\) uniformly.

\hypertarget{fall-2018-midterm-2.3}{%
\subsection{Fall 2018 Midterm 2.3}\label{fall-2018-midterm-2.3}}

Let \(g\in L^\infty([0, 1])\).

\begin{enumerate}
\def\labelenumi{\alph{enumi}.}
\item
  Prove
  \begin{align*}
  {\left\lVert {g} \right\rVert}_{L^p([0, 1])}  \overset{p\to\infty}\to {\left\lVert {g} \right\rVert}_{L^\infty([0, 1])}
  .\end{align*}
\item
  Prove that the map
  \begin{align*}
  \Lambda_g: L^1([0, 1]) &\to {\mathbb{C}}\\
  f &\mapsto \int_0^1 fg
  \end{align*}
  defines an element of \(L^1([0, 1]) {}^{ \vee }\) with
  \({\left\lVert {\Lambda_g} \right\rVert}_{L^1([0, 1]) {}^{ \vee }}= {\left\lVert {g} \right\rVert}_{L^\infty([0, 1])}\).
\end{enumerate}

\hypertarget{fall-2018-midterm-2.4}{%
\subsection{Fall 2018 Midterm 2.4}\label{fall-2018-midterm-2.4}}

See \cref{hilbert_space_exam_question}

\hypertarget{practice-exam-november-2014}{%
\section{Practice Exam (November
2014)}\label{practice-exam-november-2014}}

\hypertarget{fall-2018-practice-midterm-1.1}{%
\subsection{Fall 2018 Practice Midterm
1.1}\label{fall-2018-practice-midterm-1.1}}

Let \(m_*(E)\) denote the Lebesgue outer measure of a set
\(E \subseteq {\mathbb{R}}^n\).

\begin{enumerate}
\def\labelenumi{\alph{enumi}.}
\item
  Prove using the definition of Lebesgue outer measure that
  \begin{align*}
  m \qty{ \displaystyle\bigcup_{j=1}^{\infty } E_j  } \leq \sum_{j=1}^{\infty } m_*(E_j) 
  .\end{align*}
\item
  Prove that for any \(E \subseteq {\mathbb{R}}^n\) and any
  \(\epsilon> 0\) there exists an open set \(G\) with \(E \subseteq G\)
  and
  \begin{align*}
  m_*(E) \leq m_*(G) \leq m_*(E) + \epsilon
  .\end{align*}
\end{enumerate}

\hypertarget{fall-2018-practice-midterm-1.2}{%
\subsection{Fall 2018 Practice Midterm
1.2}\label{fall-2018-practice-midterm-1.2}}

\begin{enumerate}
\def\labelenumi{\alph{enumi}.}
\item
  See \cref{equivalence_of_approximating_measures}
\item
  Let \(f_k\) be a sequence of extended real-valued Lebesgue measurable
  function.

  \begin{enumerate}
  \def\labelenumii{\roman{enumii}.}
  \item
    Prove that \(\inf_k f_k, \sup_k f_k\) are both Lebesgue measurable
    function.

    \emph{Hint: argue that}
    \begin{align*}
    \left\{{x {~\mathrel{\Big\vert}~}\inf_k f_k(x) < a}\right\} = \displaystyle\bigcup_k \left\{{x {~\mathrel{\Big\vert}~}f_k(x) < a}\right\}
    .\end{align*}
  \item
    Carefully state Fatou's Lemma and deduce the Monotone Converge
    Theorem from it.
  \end{enumerate}
\end{enumerate}

\hypertarget{fall-2018-practice-midterm-1.3}{%
\subsection{Fall 2018 Practice Midterm
1.3}\label{fall-2018-practice-midterm-1.3}}

\begin{enumerate}
\def\labelenumi{\alph{enumi}.}
\item
  Prove that if \(f, g\in L^+({\mathbb{R}})\) then
  \begin{align*}
  \int(f +g) = \int f + \int g
  .\end{align*}
  Extend this to establish that if
  \(\left\{{ f_k}\right\} \subseteq L^+({\mathbb{R}}^n)\) then
  \begin{align*}
    \int \sum_k f_k = \sum_k \int f_k
    .\end{align*}
\item
  Let
  \(\left\{{E_j}\right\}_{j\in {\mathbb{N}}} \subseteq \mathcal{M}({\mathbb{R}}^n)\)
  with \(E_j \nearrow E\). Use the countable additivity of \(\mu_f\) on
  \(\mathcal{M}({\mathbb{R}}^n)\) established above to show that
  \begin{align*}
    \mu_f(E) = \lim_{j\to \infty } \mu_f(E_j)
    .\end{align*}
\end{enumerate}

\hypertarget{fall-2018-practice-midterm-1.4}{%
\subsection{Fall 2018 Practice Midterm
1.4}\label{fall-2018-practice-midterm-1.4}}

\begin{enumerate}
\def\labelenumi{\alph{enumi}.}
\item
  Show that
  \(f\in L^1({\mathbb{R}}^n) \implies {\left\lvert {f(x)} \right\rvert} < \infty\)
  almost everywhere.
\item
  Show that if \(\left\{{f_k}\right\} \subseteq L^1({\mathbb{R}}^n)\)
  with \(\sum {\left\lVert {f_k} \right\rVert}_1 < \infty\) then
  \(\sum f_k\) converges almost everywhere and in \(L^1\).
\item
  Use the Dominated Convergence Theorem to evaluate
  \begin{align*}
  \lim_{t\to 0} \int_0^1 {e^{tx^2} - 1 \over t} \,dx
  .\end{align*}
\end{enumerate}

\hypertarget{practice-exam-november-2014-1}{%
\section{Practice Exam (November
2014)}\label{practice-exam-november-2014-1}}

\hypertarget{fall-2018-practice-midterm-2.1}{%
\subsection{Fall 2018 Practice Midterm
2.1}\label{fall-2018-practice-midterm-2.1}}

\begin{enumerate}
\def\labelenumi{\alph{enumi}.}
\item
  Carefully state Tonelli's theorem for a nonnegative function
  \(F(x, t)\) on \({\mathbb{R}}^n\times{\mathbb{R}}\).
\item
  Let \(f:{\mathbb{R}}^n\to [0, \infty]\) and define
  \begin{align*}
  {\mathcal{A}}\coloneqq\left\{{(x, t) \in {\mathbb{R}}^n\times{\mathbb{R}}{~\mathrel{\Big\vert}~}0\leq t \leq f(x)}\right\}
  .\end{align*}

  Prove the validity of the following two statements:

  \begin{enumerate}
  \def\labelenumii{\arabic{enumii}.}
  \tightlist
  \item
    \(f\) is Lebesgue measurable on
    \({\mathbb{R}}^{n} \iff {\mathcal{A}}\) is a Lebesgue measurable
    subset of \({\mathbb{R}}^{n+1}\).
  \item
    If \(f\) is Lebesgue measurable on \({\mathbb{R}}^n\) then
    \begin{align*}
    m(\mathcal{A})=\int_{\mathbb{R}^{n}} f(x) d x=\int_{0}^{\infty} m\left(\left\{x \in \mathbb{R}^{n}{~\mathrel{\Big\vert}~}f(x) \geq t\right\}\right) d t
    .\end{align*}
  \end{enumerate}
\end{enumerate}

\hypertarget{fall-2018-practice-midterm-2.2}{%
\subsection{Fall 2018 Practice Midterm
2.2}\label{fall-2018-practice-midterm-2.2}}

\begin{enumerate}
\def\labelenumi{\alph{enumi}.}
\item
  Let \(f, g\in L^1({\mathbb{R}}^n)\) and give a definition of
  \(f\ast g\).
\item
  Prove that if \(f, g\) are integrable and bounded, then
  \begin{align*}
  (f\ast g)(x) \overset{{\left\lvert {x} \right\rvert}\to\infty}\to 0
  .\end{align*}
\item
  In parts:

  \begin{enumerate}
  \def\labelenumii{\arabic{enumii}.}
  \tightlist
  \item
    Define the \emph{Fourier transform} of an integrable function \(f\)
    on \({\mathbb{R}}^n\).
  \item
    Give an outline of the proof of the Fourier inversion formula.
  \item
    Give an example of a function \(f\in L^1({\mathbb{R}}^n)\) such that
    \(\widehat{f}\) is not in \(L^1({\mathbb{R}}^n)\).
  \end{enumerate}
\end{enumerate}

\hypertarget{fall-2018-practice-midterm-2.3}{%
\subsection{Fall 2018 Practice Midterm
2.3}\label{fall-2018-practice-midterm-2.3}}

\label{hilbert_space_exam_question}

Let \(\left\{{u_n}\right\}_{n=1}^\infty\) be an orthonormal sequence in
a Hilbert space \(H\).

\begin{enumerate}
\def\labelenumi{\alph{enumi}.}
\item
  Let \(x\in H\) and verify that
  \begin{align*}
  \left\|x-\sum_{n=1}^{N}\left\langle x, u_{n}\right\rangle u_{n}\right\|_H^{2} =
  \|x\|_H^{2}-\sum_{n=1}^{N}\left|\left\langle x, u_{n}\right\rangle\right|^{2}
  .\end{align*}
  for any \(N\in {\mathbb{N}}\) and deduce that
  \begin{align*}
  \sum_{n=1}^{\infty}\left|\left\langle x, u_{n}\right\rangle\right|^{2} \leq\|x\|_H^{2}
  .\end{align*}
\item
  Let
  \(\left\{{a_n}\right\}_{n\in {\mathbb{N}}} \in \ell^2({\mathbb{N}})\)
  and prove that there exists an \(x\in H\) such that
  \(a_n = {\left\langle {x},~{u_n} \right\rangle}\) for all
  \(n\in {\mathbb{N}}\), and moreover \(x\) may be chosen such that
  \begin{align*}
  {\left\lVert {x} \right\rVert}_H = \qty{ \sum_{n\in {\mathbb{N}}} {\left\lvert {a_n} \right\rvert}^2}^{1\over 2}
  .\end{align*}
\item
  Prove that if \(\left\{{u_n}\right\}\) is \emph{complete}, Bessel's
  inequality becomes an equality.
\end{enumerate}

\emph{Solution omitted.}

\emph{Solution omitted.}

\hypertarget{fall-2018-practice-midterm-2.4}{%
\subsection{Fall 2018 Practice Midterm
2.4}\label{fall-2018-practice-midterm-2.4}}

\begin{enumerate}
\def\labelenumi{\alph{enumi}.}
\item
  Prove Holder's inequality: let \(f\in L^p, g\in L^q\) with \(p, q\)
  conjugate, and show that
  \begin{align*}
  {\left\lVert {fg} \right\rVert}_{p} \leq {\left\lVert {f} \right\rVert}_{p} \cdot {\left\lVert {g} \right\rVert}_{q}
  .\end{align*}
\item
  Prove Minkowski's Inequality:
  \begin{align*}
  1\leq p < \infty \implies {\left\lVert {f+g} \right\rVert}_{p} \leq {\left\lVert {f} \right\rVert}_{p}+ {\left\lVert {g} \right\rVert}_{p}
  .\end{align*}
  Conclude that if \(f, g\in L^p({\mathbb{R}}^n)\) then so is \(f+g\).
\item
  Let \(X = [0, 1] \subset {\mathbb{R}}\).

  \begin{enumerate}
  \def\labelenumii{\arabic{enumii}.}
  \item
    Give a definition of the Banach space \(L^\infty(X)\) of essentially
    bounded functions of \(X\).
  \item
    Let \(f\) be non-negative and measurable on \(X\), prove that
    \begin{align*}
     \int_X f(x)^p \,dx \overset{p\to\infty}\to
     \begin{dcases}
     \infty \quad\text{or} \\
     m\qty{\left\{{f^{-1}(1)}\right\}}
     \end{dcases}
     ,\end{align*}
    and characterize the functions of each type
  \end{enumerate}
\end{enumerate}

\emph{Solution omitted.}

\hypertarget{fall-2018-practice-midterm-2.5}{%
\subsection{Fall 2018 Practice Midterm
2.5}\label{fall-2018-practice-midterm-2.5}}

Let \(X\) be a normed vector space.

\begin{enumerate}
\def\labelenumi{\alph{enumi}.}
\item
  Give the definition of what it means for a map \(L:X\to {\mathbb{C}}\)
  to be a \emph{linear functional}.
\item
  Define what it means for \(L\) to be \emph{bounded} and show \(L\) is
  bounded \(\iff L\) is continuous.
\item
  Prove that
  \((X {}^{ \vee }, {\left\lVert {{-}} \right\rVert}_{^{\operatorname{op}}})\)
  is a Banach space.
\end{enumerate}

\begin{quote}
DZG: this comes from some tex file that I found when studying for quals,
so is definitely not my own content! I've just copied it here for extra
practice.
\end{quote}

\hypertarget{may-2016-qual}{%
\section{May 2016 Qual}\label{may-2016-qual}}

\hypertarget{may-2016-1}{%
\subsection{May 2016, 1}\label{may-2016-1}}

Consider the function \(f(x) = \frac{x}{1-x^2}\), \(x \in (0,1)\).

\begin{enumerate}
\def\labelenumi{\arabic{enumi}.}
\item
  By using the \(\epsilon-\delta\) definition of the limit only, prove
  that \(f\) is continuous on \((0,1)\). (Note: You are not allowed to
  trivialize the problem by using properties of limits).
\item
  Is \(f\) uniformly continuous on \((0,1)\)? Justify your answer.
\end{enumerate}

\emph{Proof omitted.}

\emph{Proof omitted.}

\hypertarget{may-2016-2}{%
\subsection{(May 2016, 2)}\label{may-2016-2}}

Let \(\{a_k\}_{k=1}^\infty\) be a bounded sequence of real numbers and
\(E\) given by:
\begin{align*}E:= \bigg\{s \in \mathbb{R}\, \colon \, \text{ the set } \{k \in \mathbb{N}\, \colon \, a_k \geq s\} \text{ has at most finitely many elements}\bigg\}.\end{align*}
Prove that \(\limsup_{k \to \infty} a_k = \inf E\).

\emph{Proof omitted.}

\hypertarget{may-2016-3}{%
\subsection{(May 2016, 3)}\label{may-2016-3}}

Assume \((X,d)\) is a compact metric space.

\begin{enumerate}
\def\labelenumi{\arabic{enumi}.}
\item
  Prove that \(X\) is both complete and separable.
\item
  Suppose \(\{x_k\}_{k=1}^\infty \subseteq X\) is a sequence such that
  the series \(\sum_{k=1}^\infty d(x_k, x_{k+1})\) converges. Prove that
  the sequence \(\{x_k\}_{k=1}^\infty\) converges in \(X\).
\end{enumerate}

\hypertarget{may-2016-4}{%
\subsection{(May 2016, 4)}\label{may-2016-4}}

Suppose that \(f \colon [0,2] \to \mathbb{R}\) is continuous on
\([0,2]\) , differentiable on \((0,2)\), and such that
\(f(0) = f(2) = 0\), \(f(c) = 1\) for some \(c \in (0,2)\). Prove that
there exists \(x \in (0,2)\) such that \(|f'(x)| >1.\)

\emph{Proof omitted.}

\hypertarget{may-2016-5}{%
\subsection{(May 2016, 5)}\label{may-2016-5}}

Let \(f_n(x) = n^\beta x(1-x^2)^n\), \(x \in [0,1]\),
\(n \in \mathbb{N}\).

\begin{enumerate}
\def\labelenumi{\arabic{enumi}.}
\item
  Prove that \(\{f_n\}_{n=1}^\infty\) converges pointwise on \([0,1]\)
  for every \(\beta \in \mathbb{R}\).
\item
  Show that the convergence in part (a) is uniform for all
  \(\beta < \frac{1}{2}\), but not uniform for any
  \(\beta \geq \frac{1}{2}\).
\end{enumerate}

\hypertarget{may-2016-6}{%
\subsection{(May 2016, 6)}\label{may-2016-6}}

\begin{enumerate}
\def\labelenumi{\arabic{enumi}.}
\item
  Suppose \(f \colon [-1,1] \to \mathbb{R}\) is a bounded function that
  is continuous at \(0\). Let \(\alpha(x) = -1\) for \(x \in [-1,0]\)
  and \(\alpha(x)=1\) for \(x \in (0,1]\). Prove that
  \(f \in \mathcal{R}(\alpha)[-1,1]\), i.e., \(f\) is Riemann integrable
  with respect to \(\alpha\) on \([-1,1]\), and
  \(\int_{-1}^1 f d\alpha = 2f(0)\).
\item
  Let \(g \colon [0,1] \to \mathbb{R}\) be a continuous function such
  that \(\int_0^1 g(x)x^{3k+2} dx = 0\) for all \(k = 0, 1, 2, \ldots\).
  Prove that \(g(x) =0\) for all \(x \in [0,1]\).
\end{enumerate}

\emph{Proof omitted.}

\emph{Proof omitted.}

\hypertarget{metric-spaces-and-topology}{%
\section{Metric Spaces and Topology}\label{metric-spaces-and-topology}}

\hypertarget{may-2019-1}{%
\subsection{(May 2019, 1)}\label{may-2019-1}}

Let \((M, d_M)\), \((N, d_N)\) be metric spaces. Define
\(d_{M \times N} \colon (M \times N) \times (M \times N) \to \mathbb{R}\)
by
\begin{align*}d_{M \times N}((x_1, y_1), (x_2, y_2)) := d_M(x_1, x_2) + d_N(y_1, y_2).\end{align*}

\begin{enumerate}
\def\labelenumi{\arabic{enumi}.}
\item
  Prove that \((M \times N, d_{M \times N})\) is a metric space.
\item
  Let \(S \subseteq M\) and \(T \subseteq N\) be compact sets in
  \((M, d_M)\) and \((N, d_N)\), respectively. Prove that \(S \times T\)
  is a compact set in \((M \times N, d_{M \times N})\).
\end{enumerate}

\emph{Proof omitted.}

\emph{Proof omitted.}

\hypertarget{june-2003-1bc}{%
\subsection{(June 2003, 1b,c)}\label{june-2003-1bc}}

\begin{enumerate}
\def\labelenumi{(\alph{enumi})}
\setcounter{enumi}{1}
\tightlist
\item
  Show by example that the union of infinitely many compact subsets of a
  metric space need not be compact. (c) If \((X,d)\) is a metric space
  and \(K\subset X\) is compact, define
  \(d(x_0,K)=\inf_{y\in K} d(x_0,y)\). Prove that there exists a point
  \(y_0\in K\) such that \(d(x_0,K)=d(x_0,y_0)\).
\end{enumerate}

\hypertarget{january-2009-4a}{%
\subsection{(January 2009, 4a)}\label{january-2009-4a}}

Consider the metric space \((\mathbb{Q},d)\) where \(\mathbb{Q}\)
denotes the rational numbers and \(d(x,y)=|x-y|\). Let
\(E=\{x\in\mathbb{Q}:x>0,\,2<x^2<3\}\). Is \(E\) closed and bounded in
\(\mathbb{Q}\)? Is \(E\) compact in \(\mathbb{Q}\)?

\hypertarget{january-2011-3a}{%
\subsection{(January 2011 3a)}\label{january-2011-3a}}

Let \((X,d)\) be a metric space, \(K\subset X\) be compact, and
\(F\subset X\) be closed. If \(K\cap F=\emptyset\), prove that there
exists an \(\epsilon>0\) so that \(d(k,f)\geq \epsilon\) for all
\(k\in K\) and \(f\in F\).

\emph{Proof omitted.}

\hypertarget{section-18}{%
\subsection{5?}\label{section-18}}

Let \((X,d)\) be an unbounded and connected metric space. Prove that for
each \(x_0 \in X\), the set \(\{x \in X \, \colon \, d(x,x_0) = r\}\) is
nonempty.

\hypertarget{sequences-and-series}{%
\section{Sequences and Series}\label{sequences-and-series}}

\hypertarget{june-2013-1a}{%
\subsection{(June 2013 1a)}\label{june-2013-1a}}

Let \(a_n =\sqrt{n}\left(\sqrt{n+1}-\sqrt{n}\right)\). Prove that
\(\lim_{n\to\infty}a_n=1/2\).

\hypertarget{january-2014-2}{%
\subsection{(January 2014 2)}\label{january-2014-2}}

\begin{enumerate}
\def\labelenumi{\arabic{enumi}.}
\item
  Produce sequences \(\{a_n\},\,\{b_n\}\) of positive real numbers such
  that
  \begin{align*}\liminf_{n\to\infty}(a_nb_n)>\left(\liminf_{n\to\infty} a_n\right) \left(\liminf_{n\to\infty} b_n\right).\end{align*}
\item
  If \(\{a_n\},\,\{b_n\}\) are sequences of positive real numbers and
  \(\{a_n\}\) converges, prove that
  \begin{align*}\liminf_{n\to\infty}(a_nb_n)=\left(\lim_{n\to\infty}a_n\right)\left(\liminf_{n\to\infty}b_n\right).\end{align*}
\end{enumerate}

\hypertarget{may-2011-4a}{%
\subsection{(May 2011 4a)}\label{may-2011-4a}}

Determine the values of \(x\in\mathbb{R}\) for which
\(\displaystyle\sum_{n=1}^\infty \frac{x^n}{1+n|x|^n}\) converges,
justifying your answer carefully.

\hypertarget{june-2005-3b}{%
\subsection{(June 2005 3b)}\label{june-2005-3b}}

If the series \(\sum_{n=0}^\infty a_n\) converges conditionally, show
that the radius of convergence of the power series
\(\sum_{n=0}^\infty a_nx^n\) is 1.

\hypertarget{january-2011-5}{%
\subsection{(January 2011 5)}\label{january-2011-5}}

Suppose \(\{a_n\}\) is a sequence of positive real numbers such that
\(\lim_{n\to\infty}a_n=0\) and \(\sum a_n\) diverges. Prove that for all
\(x>0\) there exist integers \(n(1)<n(2)<\ldots\) such that
\(\sum_{k=1}^\infty a_{n(k)}=x\).\\

\begin{quote}
(Note: Many variations on this problem are possible including more
general rearrangements. You may also wish to show that if \(\sum a_n\)
converges conditionally then given any \(x\in\mathbb{R}\) there is a
rearrangement of \(\{b_n\}\) of \(\{a_n\}\) such that \(\sum b_n=r\).
See Rudin Thm. 3.54 for a further generalization.)
\end{quote}

\hypertarget{june-2008-4b}{%
\subsection{(June 2008 \# 4b)}\label{june-2008-4b}}

Assume \(\beta >0\), \(a_n>0\), \(n=1,2,\ldots\), and the series
\(\sum a_n\) is divergent. Show that
\(\displaystyle \sum \frac{a_n}{\beta + a_n}\) is also divergent.

\hypertarget{continuity-of-functions}{%
\section{Continuity of Functions}\label{continuity-of-functions}}

\hypertarget{differential-calculus}{%
\section{Differential Calculus}\label{differential-calculus}}

\hypertarget{june-2005-1a}{%
\subsection{(June 2005 1a)}\label{june-2005-1a}}

Use the definition of the derivative to prove that if \(f\) and \(g\)
are differentiable at \(x\), then \(fg\) is differentiable at \(x\).

\hypertarget{january-2006-2b}{%
\subsection{(January 2006 2b)}\label{january-2006-2b}}

Assume that \(f\) is differentiable at \(a\). Evaluate
\begin{align*}\lim_{x\to a}\frac{a^nf(x)-x^nf(a)}{x-a},\quad n\in\mathbb{N}.\end{align*}

\hypertarget{june-2007-3a}{%
\subsection{(June 2007 3a)}\label{june-2007-3a}}

Suppose that \(f,g:\mathbb{R}\to\mathbb{R}\) are differentiable, that
\(f(x)\leq g(x)\) for all \(x\in\mathbb{R}\), and that \(f(x_0)=g(x_0)\)
for some \(x_0\). Prove that \(f'(x_0)=g'(x_0)\).

\hypertarget{june-2008-3a}{%
\subsection{(June 2008 3a)}\label{june-2008-3a}}

Prove that if \(f'\) exists and is bounded on \((a,b]\), then
\(\lim_{x\to a^+}f(x)\) exists.

\hypertarget{january-2012-4b-extended}{%
\subsection{(January 2012 4b,
extended)}\label{january-2012-4b-extended}}

Let \(f:\mathbb{R}\to\mathbb{R}\) be a differentiable function with
\(f'\in C(\mathbb{R})\). Assume that there are \(a,b\in\mathbb{R}\) with
\(\lim_{x\to\infty}f(x)=a\) and \(\lim_{x\to\infty}f'(x)=b\). Prove that
\(b=0\). Then, find a counterexample to show that the assumption
\(\lim_{x\to\infty}f'(x)\) exists is necessary.

\hypertarget{june-2012-1a}{%
\subsection{(June 2012 1a)}\label{june-2012-1a}}

Suppose that \(f:\mathbb{R}\to\mathbb{R}\) satisfies \(f(0)=0\). Prove
that \(f\) is differentiable at \(x=0\) if and only if there is a
function \(g:\mathbb{R}\to\mathbb{R}\) which is continuous at \(x=0\)
and satisfies \(f(x)=xg(x)\) for all \(x\in\mathbb{R}\).

\hypertarget{integral-calculus}{%
\section{Integral Calculus}\label{integral-calculus}}

\hypertarget{january-2006-4b}{%
\subsection{(January 2006 4b)}\label{january-2006-4b}}

Suppose that \(f\) is continuous and \(f(x)\geq 0\) on \([0,1]\). If
\(f(0)>0\), prove that \(\int_0^1 f(x)dx>0\).

\hypertarget{june-2005-1b}{%
\subsection{(June 2005 1b)}\label{june-2005-1b}}

Use the definition of the Riemann integral to prove that if \(f\) is
bounded on \([a,b]\) and is continuous everywhere except for finitely
many points in \((a,b)\), then \(f\in\mathscr{R}\) on \([a,b]\).

\hypertarget{january-2010-5}{%
\subsection{(January 2010 5)}\label{january-2010-5}}

Suppose that \(f:[a,b]\to\mathbb{R}\) is continuous, \(f\geq 0\) on
\([a,b]\), and put \(M=\sup\{f(x):x\in[a,b]\}\). Prove that
\begin{align*}\lim_{p\to\infty}\left(\int_a^b f(x)^p\,dx\right)^{1/p}=M.\end{align*}

\hypertarget{january-2009-4b}{%
\subsection{(January 2009 4b)}\label{january-2009-4b}}

Let \(f\) be a continuous real-valued function on \([0,1]\). Prove that
there exists at least one point \(\xi\in[0,1]\) such that
\(\int_0^1 x^4 f(x)\,dx=\frac{1}{5}f(\xi)\).

\emph{Proof omitted.}

\hypertarget{june-2009-5b}{%
\subsection{(June 2009 5b)}\label{june-2009-5b}}

Let \(\phi\) be a real-valued function defined on \([0,1]\) such that
\(\phi\), \(\phi'\), and \(\phi''\) are continuous on \([0,1]\). Prove
that
\begin{align*}\int_0^1 \cos x \frac{x\phi'(x)-\phi(x)+\phi(0)}{x^2}\,dx<\frac{3}{2}||\phi''||_\infty,\end{align*}
where \(||\phi''||_\infty = \sup_{[0,1]}|\phi''(x)|.\) Note that \(3/2\)
may not be the smallest possible constant.

\hypertarget{sequences-and-series-of-functions}{%
\section{Sequences and Series of
Functions}\label{sequences-and-series-of-functions}}

\hypertarget{june-2010-6a}{%
\subsection{(June 2010 6a)}\label{june-2010-6a}}

Let \(f:[0,1]\to\mathbb{R}\) be continuous with \(f(0)\neq f(1)\) and
define \(f_n(x)=f(x^n)\). Prove that \(f_n\) does not converge uniformly
on \([0,1]\).

\hypertarget{january-2008-5a}{%
\subsection{(January 2008 5a)}\label{january-2008-5a}}

Let \(f_n(x) = \frac{x}{1+nx^2}\) for \(n \in \mathbb{N}\). Let
\(\mathcal{F} := \{f_n \, \colon \, n = 1, 2, 3, \ldots\}\) and
\([a,b]\) be any compact subset of \(\mathbb{R}\). Is \(\mathcal{F}\)
equicontinuous? Justify your answer.

\hypertarget{january-2005-4-june-2010-6b}{%
\subsection{(January 2005 4, June 2010
6b)}\label{january-2005-4-june-2010-6b}}

If \(f:[0,1]\to\mathbb{R}\) is continuous, prove that
\begin{align*}\displaystyle\lim_{n\to\infty}\int_0^1 f(x^n)\,dx=f(0).\end{align*}

\hypertarget{january-2020-4a}{%
\subsection{(January 2020 4a)}\label{january-2020-4a}}

Let \(M<\infty\) and \(\mathcal{F} \subseteq C[a,b]\). Assume that each
\(f \in \mathcal{F}\) is differentiable on \((a,b)\) and satisfies
\(|f(a)| \leq M\) and \(|f'(x)| \leq M\) for all \(x \in (a,b)\). Prove
that \(\mathcal{F}\) is equicontinuous on \([a,b]\).

\hypertarget{june-2005-5}{%
\subsection{(June 2005 5)}\label{june-2005-5}}

Suppose that \(f\in C([0,1])\) and that
\(\displaystyle \int_0^1 f(x)x^n\,dx=0\) for all
\(n=99,100,101,\ldots\). Show that \(f\equiv 0\).\\

\begin{quote}
Note: Many variations on this problem exist. See June 2012 6b and
others.
\end{quote}

\hypertarget{january-2005-3b}{%
\subsection{(January 2005 3b)}\label{january-2005-3b}}

Suppose \(f_n:[0,1]\to\mathbb{R}\) are continuous functions converging
uniformly to \(f:[0,1]\to\mathbb{R}\). Either prove that
\(\displaystyle\lim_{n\to\infty}\int_{1/n}^1 f_n(x)\,dx=\int_0^1 f(x)\,dx\)
or give a counterexample.

\hypertarget{miscellaneous-topics}{%
\section{Miscellaneous Topics}\label{miscellaneous-topics}}

\hypertarget{bounded-variation}{%
\subsection*{Bounded Variation}\label{bounded-variation}}
\addcontentsline{toc}{subsection}{Bounded Variation}

\hypertarget{january-2018}{%
\subsection{(January 2018)}\label{january-2018}}

Let \(f \colon [a,b] \to \mathbb{R}\). Suppose \(f \in \text{BV}[a,b]\).
Prove \(f\) is the difference of two increasing functions.

\hypertarget{january-2007-6a}{%
\subsection{(January 2007, 6a)}\label{january-2007-6a}}

Let \(f\) be a function of bounded variation on \([a,b]\). Furthermore,
assume that for some \(c>0\), \(|f(x)| \geq c\) on \([a,b]\). Show that
\(g(x) = 1/f(x)\) is of bounded variation on \([a,b]\).

\hypertarget{january-2017-2a}{%
\subsection{(January 2017, 2a)}\label{january-2017-2a}}

Define \(f \colon [0,1] \to [-1,1]\) by
\begin{align*}f(x):= \begin{cases} x\sin\big({\frac{1}{x}}\big) & 0 < x \leq 1 \\ 0 & x = 0 \end{cases}\end{align*}
Determine, with justification, whether \(f\) is if bounded variation on
the interval \([0,1]\).

\hypertarget{january-2020-6a}{%
\subsection{(January 2020, 6a)}\label{january-2020-6a}}

Let \(\{a_n\}_{n=1}^\infty \subseteq \mathbb{R}\) and a strictly
increasing sequence \(\{x_n\}_{n=1}^\infty \subseteq (0,1)\) be given.
Assume that \(\sum_{n=1}^\infty a_n\) is absolutely convergent, and
define \(\alpha \colon [0,1] \to \mathbb{R}\) by
\begin{align*}\alpha(x):= \begin{cases} a_n &  x=x_n \\ 0 & \text{otherwise} \end{cases}.\end{align*}
Prove or disprove: \(\alpha\) has bounded variation on \([0,1]\).

\hypertarget{metric-spaces-and-topology-1}{%
\subsection*{Metric Spaces and
Topology}\label{metric-spaces-and-topology-1}}
\addcontentsline{toc}{subsection}{Metric Spaces and Topology}

\begin{enumerate}
\def\labelenumi{\arabic{enumi}.}
\tightlist
\item
  Find an example of a metric space \(X\) and a subset \(E \subseteq X\)
  such that \(E\) is closed and bounded but not compact.
\end{enumerate}

\hypertarget{may-2017-6}{%
\subsection{(May 2017 6)}\label{may-2017-6}}

Let \((X,d)\) be a metric space. A function
\(f \colon X \to \mathbb{R}\) is said to be lower semi-continuous
(l.s.c) if \(f^{-1}(a,\infty) = \{x \in X \, \colon \, f(x)> a\}\) is
open in \(X\) for every \(a \in \mathbb{R}\). Analogously, \(f\) is
upper semi-continuous (u.s.c) if
\(f^{-1}(-\infty, b) = \{x \in X \, \colon \, f(x)<b\}\) is open in
\(X\) for every \(b \in \mathbb{R}\).

\begin{enumerate}
\def\labelenumi{\arabic{enumi}.}
\item
  Prove that a function \(f \colon X \to \mathbb{R}\) is continuous if
  and only if \(f\) is both l.s.c. and u.s.c.
\item
  Prove that \(f\) is lower semi-continuous if and only if
  \(\liminf_{n \to \infty} f(x_n) \geq f(x)\) whenever
  \(\{x_n\}_{n=1}^\infty \subseteq X\) such that \(x_n \to x\) in \(X\).
\end{enumerate}

\hypertarget{january-2017-3}{%
\subsection{(January 2017 3)}\label{january-2017-3}}

Let \((X,d)\) be a compact metric space. Suppose that
\(f_n \colon X \to [0,\infty)\) is a sequence of continuous functions
with \(f_n(x) \geq f_{n+1}(x)\) for all \(n \in \mathbb{N}\) and
\(x \in X\), and such that \(f_n \to 0\) pointwise on \(X\). Prove that
\(\{f_n\}_{n=1}^\infty\) converges uniformly on \(X\).

\hypertarget{integral-calculus-1}{%
\section{Integral Calculus}\label{integral-calculus-1}}

\hypertarget{section-19}{%
\subsection{1.}\label{section-19}}

(June 2014 1)Define \(\alpha \colon [-1,1] \to \mathbb{R}\) by
\begin{align*}\alpha(x) := \begin{cases} -1 & x \in [-1,0] \\ 1 & x \in (0,1]. \end{cases}\end{align*}
Let \(f \colon [-1,1] \to \mathbb{R}\) be a function that is uniformly
bounded on \([-1,1]\) and continuous at \(x=0\), but not necessarily
continuous for \(x \neq 0\). Prove that \(f\) is Riemann-Stieltjes
integrable with respect to \(\alpha\) over \([-1,1]\) and that
\begin{align*}\int_{-1}^1 f(x)d\alpha(x) = 2f(0).\end{align*}

\hypertarget{june-2017-2}{%
\subsection{(June 2017 2)}\label{june-2017-2}}

Prove : \(f \in \mathcal{R}(\alpha)\) on \([a,b]\) if and only if for
any \(a <c<b\), \(f \in \mathcal{R}(\alpha)\) on \([a,c]\) and on
\([c,b]\). In addition, if either condition holds, then we have that
\begin{align*}\int_a^c fd\alpha + \int_c^b fd\alpha = \int_a^b fd\alpha.\end{align*}

\hypertarget{spring-2017-7}{%
\subsection{(Spring 2017 7)}\label{spring-2017-7}}

Prove that if \(f \in \mathcal{R}\) on \([a,b]\) and
\(\alpha \in C^1[a,b]\), then the Riemann integral
\(\int_a^b f(x)\alpha'(x)dx\) exists and
\begin{align*}\int_a^b f(x) d\alpha(x)= \int_a^b f(x)\alpha'(x)dx.\end{align*}

\hypertarget{sequences-and-series-and-of-functions}{%
\section{Sequences and Series (and of
Functions)}\label{sequences-and-series-and-of-functions}}

\hypertarget{january-2006-1}{%
\subsection{(January 2006 1)}\label{january-2006-1}}

Let the power series series \(\sum_{n=0}^\infty a_nx^n\) and
\(\sum_{n=0}^\infty b_nx^n\) have radii of convergence \(R_1\) and
\(R_2\), respectively.

\hypertarget{section-20}{%
\subsection{?}\label{section-20}}

If \(R_1 \neq R_2\), prove that the radius of convergence, \(R\), of the
power series \(\sum_{n=0}^\infty (a_n+b_n)x^n\) is \(\min\{R_1, R_2\}\).
What can be said about \(R\) when \(R_1 = R_2\)?

\hypertarget{section-21}{%
\subsection{?}\label{section-21}}

Prove that the radius of convergence, \(R\), of
\(\sum_{n=0}^\infty a_nb_nx^n\) satisfies \(R \geq R_1R_2\). Show by
means of example that this inequality can be strict.

\hypertarget{section-22}{%
\subsection{?}\label{section-22}}

Show that the infinite series \(\sum_{n=0}^\infty x^n2^{-nx}\) converges
uniformly on \([0,B]\) for any \(B > 0\). Does this series converge
uniformly on \([0,\infty)\)?

\hypertarget{january-2006-4a}{%
\subsection{(January 2006 4a)}\label{january-2006-4a}}

Let
\begin{align*}f_n(x) = \begin{cases} \frac{1}{n}  & x \in (\frac{1}{2^{n+1}}, \frac{1}{2^n}] \\ 0 & \text{ otherwise}.\end{cases}\end{align*}

Show that \(\sum_{n=1}^\infty f_n\) does not satisfy the Weierstrass
M-test but that it nevertheless converges uniformly on \(\mathbb{R}\).

\hypertarget{section-23}{%
\subsection{?}\label{section-23}}

Let \(f_n \colon [0,1) \to \mathbb{R}\) be the function defined by
\begin{align*}f_n(x):= \sum_{k=1}^n \frac{x^k}{1+x^k}.\end{align*}

\begin{enumerate}
\def\labelenumi{\arabic{enumi}.}
\item
  Prove that \(f_n\) converges to a function
  \(f \colon [0,1) \to \mathbb{R}\).
\item
  Prove that for every \(0 < a < 1\) the convergence is uniform on
  \([0,a]\).
\item
  Prove that \(f\) is differentiable on \((0,1)\).
\end{enumerate}

\hypertarget{january-2019-qualifying-exam}{%
\subsection*{January 2019 Qualifying
Exam}\label{january-2019-qualifying-exam}}
\addcontentsline{toc}{subsection}{January 2019 Qualifying Exam}

\begin{enumerate}
\def\labelenumi{\arabic{enumi}.}
\tightlist
\item
  Suppose that \(f: [0,1] \to \mathbb{R}\) is differentiable and
  \(f(0) = 0\). Assume that there is a \(k > 0\) such that
  \begin{align*}|f'(x)| \leq k|f(x)|\end{align*}
  for all \(x\in [0,1]\). Prove that \(f(x) = 0\) for all
  \(x\in [0,1]\).
\end{enumerate}

\emph{Proof omitted.}



\newpage
\printbibliography[title=Bibliography]


\end{document}
